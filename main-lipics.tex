% !TEX root = main.tex

% https://submission.dagstuhl.de/documentation/authors
\documentclass[a4paper,UKenglish,cleveref, autoref, thm-restate]{lipics-v2021}
%This is a template for producing LIPIcs articles.
%See lipics-v2021-authors-guidelines.pdf for further information.
%for A4 paper format use option "a4paper", for US-letter use option "letterpaper"
%for british hyphenation rules use option "UKenglish", for american hyphenation rules use option "USenglish"
%for section-numbered lemmas etc., use "numberwithinsect"
%for enabling cleveref support, use "cleveref"
%for enabling autoref support, use "autoref"
%for anonymousing the authors (e.g. for double-blind review), add "anonymous"
%for enabling thm-restate support, use "thm-restate"
%for enabling a two-column layout for the author/affilation part (only applicable for > 6 authors), use "authorcolumns"
%for producing a PDF according the PDF/A standard, add "pdfa"


%% extra theorem environments
%\usepackage{theorem}
%\theorembodyfont{\slshape}
%\newtheorem{example}[theorem]{Example}
%\newtheorem{remark}[theorem]{Remark}
\def\endex{\hspace*{\fill} $\triangleleft$\smallskip } %\Diamond
%{\theorembodyfont{\rmfamily} \theoremstyle{break} \newtheorem{algo}{Algorithm}}



%\titlerunning{WVPA \& AMT}
\titlerunning{Symbolic Weighted Language Models and Parsing over Infinite Alphabets}


% author
\author{Florent Jacquemard}{Inria \& CNAM, Paris, France %\and My second affiliation, Country
\and \url{https://jacquema.gitlabpages.inria.fr} }{florent.jacquemard@inria.fr}{https://orcid.org/0000-0003-2269-7550}{}
%funding Inria AEx Codex, ANR Collabscore,  %ANR-20-CE27-0014 , EU H2020 Polifonia}
%TODO mandatory, please use full name; only 1 author per \author macro; first two parameters are mandatory, other parameters can be empty. Please provide at least the name of the affiliation and the country. The full address is optional

\author{Philippe Rigaux}{CNAM, Paris, France}{philippe.rigaux@cnam.fr}{}{}

\author{Lydia Rodriguez de la Nava}{Inria \& CNAM, Paris, France}{lydia.rodriguez-de-la-nava.fr}{}{}


%\author{Florent Jacquemard}
%\institute{INRIA \& CNAM, Paris, France\\
%    \email{florent.jacquemard@inria.fr}}

\authorrunning{F. Jacquemard, P. Rigaux, L. Rodriguez} 
%TODO mandatory. First: Use abbreviated first/middle names. Second (only in severe cases): Use first author plus 'et al.'

\Copyright{Florent Jacquemard et al.} 
%TODO mandatory, please use full first names. LIPIcs license is "CC-BY";  http://creativecommons.org/licenses/by/3.0/

\ccsdesc[500]{Theory of computation~Quantitative automata}
%\ccsdesc[100]{\textcolor{red}{Replace ccsdesc macro with valid one}}
%TODO mandatory: Please choose ACM 2012 classifications from https://dl.acm.org/ccs/ccs_flat.cfm

\keywords{weighted automata, symbolic automata, visibly pushdown automata, parsing}
%TODO mandatory; please add comma-separated list of keywords

%\category{} %optional, e.g. invited paper

\relatedversion{} %optional, e.g. full version hosted on arXiv, HAL, or other respository/website
%\relatedversiondetails[linktext={opt. text shown instead of the URL}, cite=DBLP:books/mk/GrayR93]{Classification (e.g. Full Version, Extended Version, Previous Version}{URL to related version} %linktext and cite are optional

%\supplement{}%optional, e.g. related research data, source code, ... hosted on a repository like zenodo, figshare, GitHub, ...
%\supplementdetails[linktext={opt. text shown instead of the URL}, cite=DBLP:books/mk/GrayR93, subcategory={Description, Subcategory}, swhid={Software Heritage Identifier}]{General Classification (e.g. Software, Dataset, Model, ...)}{URL to related version} %linktext, cite, and subcategory are optional

%\funding{(Optional) general funding statement \dots}%optional, to capture a funding statement, which applies to all authors. Please enter author specific funding statements as fifth argument of the \author macro.

%\acknowledgements{I want to thank \dots}%optional
