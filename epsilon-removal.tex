% gen. to transducers
\medskip
The following proposition shows that the transitions of type $\wei_{00}$
(analogous of $\epsilon$-transitions of automata over finite alphabets)
are not necessary to the expressiveness of $\SWA$, 
and can safely nullified, 
with an appropriate update of the other transitions.
%
\begin{proposition}
When $\Semiring$ is commutative, bounded and complete, 
given a \SWA $A$ 
over $\Sigma$, $\Semiring$ and $\bar\Phi$
there exists an effectively constructible \SWA $A'$ 
over $\Sigma$, $\Semiring$ and $\bar\Phi$,
without $\epsilon$-transitions,
such that for all $s \in \Sigma^*$, $A'(s) = A(s)$.
\end{proposition}
%
\begin{proof} 
Let $A = \< Q, \init, \<\wei_0, \wei_1>, \final >$.
%
We associate to every sequence in $Q^+$ called \emph{$\epsilon$-path}, 
and every $a \in \Sigma$, a weight value 
defined recursively as follows (for $q_1, q_2 \in Q$ and $\sigma \in Q^*$):
\[
\begin{array}{rcl}
\weight_\epsilon(q_1, a) & = & \one\\
\weight_\epsilon(q_1 q_2 \sigma, a) & = & \wei_0(q_1, a, q_2) \otimes \weight_\epsilon(q_2 \sigma, a)\\
\end{array}
\]
%
We shall construct a \SWA $A' = \< Q, \init, \<\wei'_0, \wei'_1>, \final' >$,
where, for all $q, q' \in Q$, and $a \in \Sigma$, 
and 
\[
\begin{array}{rcl}
\wei'_0(q, a, q') & = & \zero\\
\wei'_1(q, a, q') & = & 
\displaystyle\bigoplus_{r \in Q} 
\displaystyle\bigoplus_{\sigma \in Q^*} 
\weight_\epsilon(q\sigma r, a) \otimes  \wei_1(r, a, q')\\
\final'(q, a) & = & 
\final(a, q) \oplus
\displaystyle\bigoplus_{r \in Q} 
\displaystyle\bigoplus_{\sigma \in Q^*} 
\weight_\epsilon(q\sigma r, a) \otimes \final(r, a)\\
\end{array}
\]
Hence $\wei'_0(q, q')$ is the constant function $\zero$ of $\Phi_\Sigma$.
Following Equation~(\ref{eq:SWA-value}),
\[
\begin{array}{lccl}
\weight_{A'}(q, au, q') & = & 
    \displaystyle\bigoplus_{r \in Q, u \neq \epsilon} &
    \wei'_{1}(q, a, r) \otimes \weight_{A'}(r, u, q')
    \oplus \displaystyle\bigoplus_{u = \epsilon} \wei'_{1}(q, a, q') \\
 & = & \displaystyle\bigoplus_{r \in Q, u \neq \epsilon} & 
  \displaystyle\bigoplus_{\sigma \in Q^*} 
  \weight_\epsilon(q\sigma r, a) \otimes  \wei_1(r, a, q') \otimes \weight_{A'}(r, u, q')\\
 & \oplus & \displaystyle\bigoplus_{r \in Q, u = \epsilon} &
   \displaystyle\bigoplus_{\sigma \in Q^*} \weight_\epsilon(q\sigma r, a) \otimes \wei_{1}(r, a, q')  
\end{array}
\]
Moreover, %we can show that
\[
\begin{array}{lccl}
\weight_{A}(q, au, q') & = & 
    \displaystyle\bigoplus_{r \in Q, u \neq \epsilon} &
    \wei'_{1}(q, a, r) \otimes \weight_{A'}(r, u, q')\\
 & = & \displaystyle\bigoplus_{r \in Q, u \neq \epsilon} & 
  \displaystyle\bigoplus_{\sigma \in Q^*} 
  \weight_\epsilon(q\sigma r, a) \otimes  \wei_1(r, a, q')
\end{array}
\]



%By definition of $A(s)$ and $A'(s)$ (Equation~\ref{eq:SWA-value}), based on 
%$\weight_A(q, s, q')$ and $\weight_{A'}(q, s, q')$, 
%and by distributy of $\otimes$ over $\bigoplus$.



In the above definition of $\wei'_1(q, a, q')$ and $\final'(q, a)$, 
the first sum is finite and the second is not. 
%
We will show how to compute effectively the sum 
$\bigoplus_{\sigma \in Q^*} \weight_\epsilon(q\sigma r, a)$
for given $q$, $r$ and $a$, 
using a shortest distance algorithm.


** In the above definition of $\wei'_\Sigma$ we use the operator
of product of function of $\Phi_\Sigma$ by $\Semiring$ 
described in Section~\ref{section:symbols}.
%
By definition of $\weight_{A}$ and distributivity of $\otimes$ on $\oplus$, 
** NO. TBC see \cite{Mohri02ijfcs} **
it holds that $\weight_{A}(q, s, q') = \weight_{A'}(q, s, q')$.
\qed
\end{proof}     

The construction of Proposition~\ref{prop:epsilon} 
can be straigthforwardly generalized to \SWT.

not bounded: 
procedure of removal of $\epsilon$-transitions
... essentially the same as for \WA~\cite{Mohri02ijfcs}.
