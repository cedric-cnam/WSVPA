% !TEX root = main.tex
%
% Proof of best-search algo


The correctness of Algorithm~\ref{algo:Dijkstra}
is stated by the following lemma.

\begin{lemma}\label{lem:bot}
At the termination of Algorithm~\ref{algo:Dijkstra}, 
for all $\< q_1, q_2> \notin \Q$,
$d_\bot(q_1, q_2) =  b_\bot(q_1, q_2)$.
\end{lemma}

The proof is by contradiction,
assuming a counter-example minimal in the length of the witness word.

is ensured by the invariant... expressed in the following lemma.

\begin{lemma}\label{lem:top}
At every step of Algorithm~\ref{algo:Dijkstra}, 
for all $\< q_1, p, q_2> \notin \Q$,
$d_\top(q_1, p, q_2) = b_\top(q_1, p, q_2)$.
\end{lemma}

\noindent
For computing the minimal weight of a computation of $A$, we use the equality~\eqref{eq:min}, 
which, with Lemma~\ref{lem:bot} implies that
at the termination of Algorithm~\ref{algo:Dijkstra}, %it holds that,
%There exist $q_1, q_2 \in Q$
\(
  {\displaystyle \bigoplus_{s \in \Delta^*} A(s)} =
  {\displaystyle\bigoplus_{q, q' \in Q}} \textstyle
  \mathsf{in}(q) \mathop{\otimes} d_\bot(q, q') \mathop{\otimes} \mathsf{out}(q').
\)