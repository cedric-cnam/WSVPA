\documentclass[a4paper,11pt]{article}
%\setcounter{page}{1}

\usepackage[T1]{fontenc}
\usepackage[utf8]{inputenc}
\usepackage[english]{babel}

\usepackage{hyperref}
%\usepackage[bookmarks,bookmarksnumbered,naturalnames,plainpages=false]{hyperref}
%usepackage{url}

% for footnote ref
\usepackage{refcount} 

% symbols
\usepackage{amsmath} 
\usepackage{amssymb} 
\usepackage{amsbsy}
\usepackage{bbold}
\usepackage{latexsym}
%\usepackage{amsfonts}
\usepackage{stmaryrd}
%\usepackage{mathabx}
%\usepackage{MnSymbol}
\usepackage{harmony} % simple music fonts
\usepackage{mathtools} % for arrows
%\usepackage{mathptmx}

%% theorem envs
\usepackage{theorem}
\newtheorem{theorem}{Theorem} %[section]
\newtheorem{definition}[theorem]{Definition}
\newtheorem{lemma}[theorem]{Lemma}
\newtheorem{corollary}[theorem]{Corollary}
\newtheorem{proposition}[theorem]{Proposition}
\newenvironment{proof}{\vspace{-2ex}{\it Proof. }}{\hspace*{\fill} $\Box$\smallskip }
\theorembodyfont{\slshape}
\newtheorem{example}[theorem]{Example}
\newtheorem{remark}[theorem]{Remark}


% extension of enumerate env. (style for displaying counters)
% \usepackage{enumerate} 


%% pictures
% \usepackage{graphicx} 
% \DeclareGraphicsExtensions{.pdf,.png,.jpg}
% \graphicspath{fig/}

%% PGF, Tikz
%% \usepackage{pgfplots}
%% \usepgfplotslibrary{dateplot}
%% \usepackage{pgf,pgfarrows,pgfnodes, pgfautomata}
% \usepackage{tikz}
%% \usetikzlibrary{arrows}
%% \usetikzlibrary{calc}
%% \usetikzlibrary{snakes}
%% \usetikzlibrary{backgrounds}
% \usetikzlibrary{trees}
%% \usetikzlibrary{automata}
%% \usetikzlibrary{positioning}
%% \usetikzlibrary{matrix}
%% \usetikzlibrary{patterns}
%% \usetikzlibrary{shapes}

%% for new macros
\usepackage{xspace}

%% arrows etc
%
%% Extensible arrows from amsmath
%\newcommand{\lrstep}[2]{\xrightarrow{#1}{#2}}    %\mathrel ? 
%\newcommand{\rlstep}[2]{\xleftarrow{#1}{#2}}
%\newcommand{\eqstep}[2]{\xleftrightarrow{#1}{#2}}
%\newcommand{\mapstep}[2]{\mathop{\xmapsto[\scriptstyle #2]{\scriptstyle #1}}}
\makeatletter
\newcommand{\xleftrightarrow}[2][]{\ext@arrow 3359\leftrightarrowfill@{#1}{#2}}
\newcommand{\xdashrightarrow}[2][]{\ext@arrow 0359\rightarrowfill@@{#1}{#2}}
\newcommand{\xdashleftarrow}[2][]{\ext@arrow 3095\leftarrowfill@@{#1}{#2}}
\newcommand{\xdashleftrightarrow}[2][]{\ext@arrow 3359\leftrightarrowfill@@{#1}{#2}}
\def\rightarrowfill@@{\arrowfill@@\relax\relbar\rightarrow}
\def\leftarrowfill@@{\arrowfill@@\leftarrow\relbar\relax}
\def\leftrightarrowfill@@{\arrowfill@@\leftarrow\relbar\rightarrow}
\def\arrowfill@@#1#2#3#4{%
  $\m@th\thickmuskip0mu\medmuskip\thickmuskip\thinmuskip\thickmuskip
   \relax#4#1
   \xleaders\hbox{$#4#2$}\hfill
   #3$%
}
\makeatother


%% Extensible arrows from pgf/tikz
\usetikzlibrary{arrows}
\usetikzlibrary{cd}
\makeatletter
\newbox\xrat@below
\newbox\xrat@above
\newcommand{\yrightarrowhook}[2][]{%
  \setbox\xrat@below=\hbox{\ensuremath{\scriptstyle #1}}%
  \setbox\xrat@above=\hbox{\ensuremath{\scriptstyle #2}}%
  \pgfmathsetlengthmacro{\xrat@len}{max(\wd\xrat@below,\wd\xrat@above)+.7em}%
  \mathrel{\tikz [right hook->,baseline=-.75ex]
                 \draw (0,0) -- node[below=-1.5pt] {\box\xrat@below}
                                node[above=-1.5pt] {\box\xrat@above}
                       (\xrat@len,0) ;}}
\newcommand{\yrightarrow}[2][]{%
  \setbox\xrat@below=\hbox{\ensuremath{\scriptstyle #1}}%
  \setbox\xrat@above=\hbox{\ensuremath{\scriptstyle #2}}%
  \pgfmathsetlengthmacro{\xrat@len}{max(\wd\xrat@below,\wd\xrat@above)+.7em}%
  \mathrel{\tikz [->,baseline=-.75ex]
                 \draw (0,0) -- node[below=-1.5pt] {\box\xrat@below}
                                node[above=-1.5pt] {\box\xrat@above}
                       (\xrat@len,0) ;}}
\makeatother


%% Arrows
\def\Reduction#1#2#3#4{%
\mathrel{\raise1.0ex\hbox{%
\vtop{\ialign{##\crcr%
\raise0.0ex\hbox{$\hfil\scriptstyle{\ #1\ }\hfil$}\crcr%
\noalign{\nointerlineskip}%
\rightarrowfill\crcr%
\noalign{\nointerlineskip}%
\raise0.0ex\hbox{$\hfil\scriptstyle{\ #2\ }\hfil$}\crcr}}}{}^{#3}_{#4}}}
%
\def\Leduction#1#2#3#4{%
\mathrel{\raise1.0ex\hbox{%
\vtop{\ialign{##\crcr%
\raise0.0ex\hbox{$\hfil\scriptstyle{\ #1\ }\hfil$}\crcr%
\noalign{\nointerlineskip}%
\leftarrowfill\crcr%
\noalign{\nointerlineskip}%
$\hfil\scriptstyle{\ #2\ }\hfil$\crcr}}}{}^{#3}_{#4}}}
%
%\def\hookxrightarrow[#1]#2{{\lhook\hspace{-0.20em}\xrightarrow[#1]{#2}}}
\def\hookReduction#1#2#3#4{%
%\lhook\joinrel\hspace{-0.35em}
\mathrel{\raise1.2ex\hbox{%
\vtop{\ialign{##\crcr%
\raise0.0ex\hbox{$\hfil\scriptstyle{\ #1\ }\hfil$}\crcr%
\noalign{\nointerlineskip}%
$\lhook\joinrel$\hspace{-0.35em}
\rightarrowfill\crcr%
\noalign{\nointerlineskip}%
$\hfil\scriptstyle{\ #2\ }\hfil$\crcr}}}{}^{#3}_{#4}}}
%
\def\hoookReduction#1#2#3#4{%
\lhook\joinrel\hspace{-0.50em}
\raise0.85ex\hbox{%
\vtop{\ialign{##\crcr%
\raise0.4ex\hbox{$\hfil\scriptstyle{\ #1\ }\hfil$}\crcr%
\noalign{\nointerlineskip}%
\rightarrowfill\crcr%
\noalign{\nointerlineskip}%
$\hfil\scriptstyle{\ #2\ }\hfil$\crcr}}}{}^{#3}_{#4}}

%\def\reach#1#2{\mathop{#1}[#2]}

%% rewrite steps
%% \frew#1#2#3#4#5#6#7#8
\def\frew#1#2#3#4#5#6#7#8{
\setbox0=\hbox{$#6 #7 #1 #8$}%
\setbox1=\hbox{$#6 #7 #2 #8$}%
\ifdim \wd0>\wd1 \rlap{\rlap{\hbox to \wd0{#5}}%
                            {\hbox to\wd0{\hfil\lower #3\box1\relax\hfil}}}{\raise #4\box0}%
\else \rlap{\rlap{\hbox to \wd1{#5}}{\hbox to\wd1{\hfil\raise #4\box0\relax\hfil}}}{\lower #3\box1}%
\fi
}
%% \fstep
\def\fstep#1#2#3#4#5{\mathchoice{\frew{#1}{#2}{1.10ex}{1.20ex}{#5}{\scriptstyle}{#3}{#4}}%
                                {\frew{#1}{#2}{0.82ex}{1.20ex}{#5}{\scriptstyle}{#3}{#4}}%
                                {\frew{#1}{#2}{0.51ex}{0.82ex}{#5}{\scriptscriptstyle}{#3}{#4}}%
                                {\frew{#1}{#2}{0.51ex}{0.69ex}{#5}{\scriptscriptstyle}{#3}{#4}}}
%% \lrstep, \rlstep, \eqstep
% #1 top label
% #2 bottom_right label
\newcommand{\lrstep}[2]{\mathrel{\fstep{#1}{#2}{\;\>}{\>\>\;}{\rightarrowfill}}}
\newcommand{\lrsteptc}[2]{\mathrel{\fstep{#1\ }{#2\ }{\;\>}{\>\>\;}{\rightarrowfill$^*$}}}
\newcommand{\rlstep}[2]{\mathrel{\fstep{#1}{#2}{\;\>\>}{\;\>}{\leftarrowfill}}}
\newcommand{\eqstep}[2]{\mathrel{\fstep{#1}{#2}{\;\>}{\>\;}{\rlap{\leftarrowfill}{\rightarrowfill}}}}

%% \fstepd   ad hoc.... to avoid hidden overline
\def\fstepd#1#2#3#4#5{\mathchoice{\frew{#1}{#2}{1.10ex}{1.20ex}{#5}{\scriptstyle}{#3}{#4}}%
                                {\frew{#1}{#2}{1.12ex}{1.20ex}{#5}{\scriptstyle}{#3}{#4}}%
                                {\frew{#1}{#2}{0.51ex}{0.82ex}{#5}{\scriptscriptstyle}{#3}{#4}}%
                                {\frew{#1}{#2}{0.51ex}{0.69ex}{#5}{\scriptscriptstyle}{#3}{#4}}}
\newcommand{\lrstepd}[2]{\mathrel{\fstepd{#1}{#2}{\;\>}{\>\>\;}{\rightarrowfill}}}


%% Misc macros

\def\ie{\textit{i.e.}\xspace}
\def\eg{\textit{e.g.}\xspace}
\def\wrt{\textit{wrt}\xspace}
%\def\wlog{\textit{wlog}\xspace}
\def\etc{\textit{etc}\xspace}

\def\<#1>{\langle #1 \rangle}
\newcommand{\pair}[2]{\langle{#1}, {#2}\rangle}
\newcommand{\A}{\mathcal{A}}
\newcommand{\B}{\mathcal{B}}
\newcommand{\D}{\mathbb{D}}
\newcommand{\E}{\mathbb{E}}
\newcommand{\W}{\mathbb{W}}

\newcommand{\Semiring}{\mathbb{S}}
\newcommand{\zero}{\mathbb{0}}
\newcommand{\one}{\mathbb{1}}
\newcommand{\dom}{\ensuremath{\mathit{dom}}}

\def\SWT{\textsf{SWT}\xspace}
\def\SWA{\textsf{SWA}\xspace}
\def\SWVPA{\textsf{SWVPA}\xspace}
\def\weight{\mathsf{weight}}
\def\wei{\mathsf{w}}
\def\init{\mathsf{in}}
\def\final{\mathsf{out}}
\newcommand{\call}[1]{\ensuremath \langle_{#1}}
\newcommand{\return}[1]{\ensuremath {}_{#1}{\rangle}} % $\prescript{}{a}{)}$
\def\Sigmai{{\Sigma_\mathsf{i}}}
\def\Sigmac{{\Sigma_\mathsf{c}}}
\def\Sigmar{{\Sigma_\mathsf{r}}}
\def\Deltai{{\Delta_\mathsf{i}}}
\def\Deltac{{\Delta_\mathsf{c}}}
\def\Deltar{{\Delta_\mathsf{r}}}
\def\Phii{{\Phi_\mathsf{i}}}
\def\Phic{{\Phi_\mathsf{c}}}
\def\Phir{{\Phi_\mathsf{r}}}
\def\Phicr{{\Phi_\mathsf{cr}}}
\def\weii{{\wei_\mathsf{i}}}
\def\weic{{\wei_\mathsf{c}}}
\def\weir{{\wei_\mathsf{r}}}
\def\weie{{\wei_\mathsf{e}}}
\newcommand{\config}[2]{\ensuremath \genfrac{[}{]}{0pt}{}{#1}{#2}} 

%\sloppy

\title{Weighted Visibly Pushdown Automata and Automated Music Transcription}
\author{Florent Jacquemard}
%\institute{INRIA \& CNAM, Paris, France\\
%\email{florent.jacquemard@inria.fr}}

%\titlerunning{WVPA \& AMT}
%\authorrunning{Florent Jacquemard}

\date{\today}
 
\begin{document}
\thispagestyle{empty}
\maketitle

%\begin{abstract}
%Symbolic Weighted (SW) extension of symbolic automata...
%\end{abstract}


\section{SW Automata and Transducers}
\label{section:transducer}

We follow the approach of~\cite{Mohri03EDWA} for the computation of distances
between words with weighted transducers, and propose models of weighted automata 
and weighted transducers over infinite alphabets. 

These models generalize weighted automata and transducers over finite alphabets, see  e.g.~\cite{Mohri03EDWA}, 
by labeling each transition with a weight functions that takes the 
input and output symbols as parameters, instead of a simple weight value.
These functions are similar to the guards of symbolic automata~\cite{Veanes13ciaa}, 
but they can return values in an arbitrary semiring, 
where the latter guards are restricted to the Boolean semiring.


\subsection{Semirings}
\label{section:semiring}
We shall consider semiring domains for weight values.
%
A \emph{semiring} $\< \Semiring, \oplus, \zero, \otimes, \one>$ 
is a structure with a domain~$\Semiring$,
equipped with two associative
binary operators~$\oplus$ and $\otimes$
with respective neutral elements $\zero$ and $\one$ and such that:
%$\< \mathbb{S}, \oplus, \zero>$ is a commutative monoid
%$\< \mathbb{S}, \otimes, \one>$ is a monoid
$\oplus$ is commutative, 
$\otimes$ distributes over~$\oplus$:  $\forall x, y, z \in \mathbb{S}$,
$x \otimes (y \oplus z) = (x \otimes y) \oplus (x \otimes z)$, 
%  and $(x \oplus y) \otimes z = (x \otimes z) \oplus (y \otimes z)$;
and $\zero$ is absorbing for~$\otimes$: 
$\forall x\in \mathbb{S}$, $\zero \otimes x = x \otimes \zero = \zero$.
%Components of a semiring~$\Semiring$ may be subscripted by~$\Semiring$ when needed.
%We simply write $x \in \Semiring$ to mean $x \in \mathbb{S}$.
%
% In the application presented in this paper, intuitively,
% $\oplus$ selects an optimal value amongst two values and 
% $\otimes$ combines two values into a single value.
%and let $\< \Semiring, \oplus, \zero, \otimes, \one>$ be a {semiring}, 

\noindent
A semiring $\Semiring$ is \emph{commutative} if $\otimes$ is commutative.
It is \emph{idempotent} if for each $x \in \dom(\Semiring)$, $x \oplus x = x$.
%
Following the terminology of~\cite{Mohri02semiring},
when $\forall x \in \dom(\Semiring), \one \oplus x = \one$,
the semiring $\Semiring$ is is called \emph{bounded}.
Note that every bounded semiring is idempotent:
by boundedness, 
$\one \oplus \one = \one$, and idempotency follows by multiplying
both sides by $x$ and distributing. 

% \noindent
A semiring $\Semiring$ 
is \emph{monotonic} \wrt a partial ordering~$\leq$ 
iff for all $x, y, z  \in \Semiring$,  $x \leq y$ implies
$x \oplus z \leq y \oplus z$,
$x \otimes z \leq y \otimes z$
and $z \otimes x \leq z \otimes y$,
%
and it is \emph{superior} %\wrt a partial ordering~$\leq$
\wrt~$\leq$ iff for all $x, y \in \Semiring$,  
$x \leq x \otimes y$ and 
$y \leq x \otimes y$~\cite{Huang08advanceddynamic}.
The latter property corresponds to the 
\emph{non-negative weights} condition in shortest-path algorithms~\cite{Dijkstra59anote}.
Intuitively, it means that combining elements always increase their weight. 
% always get worse in term of weight
Note that when $\Semiring$ is superior \wrt~$\leq$, then $\one \leq \zero$
and moreover, for all $x \in \Semiring$, $\one \leq x \leq \zero$.

Every idempotent semiring~$\Semiring$ induces 
a partial ordering~$\leq_\Semiring$ 
called the \emph{natural ordering} of~$\Semiring$
and defined by: 
%implicitly defined by the semiring $\Semiring$ 
for all $x$ and $y$,
$x \leq_\Semiring y \;\mbox{iff}\; x \oplus y = x$.
This ordering is sometimes defined in the opposite direction~\cite{DrosteKuich09semirings};
The above definition follows \cite{Mohri02semiring}, 
and coincides than the usual ordering on the Tropical semiring (\emph{min-plus}).
%
It holds that $\Semiring$ is {monotonic} \wrt~$\leq_\Semiring$.
An idempotent semiring $\Semiring$~is called \emph{total} if
it~$\leq_\Semiring$ is total
\ie when for all $x, y \in \Semiring$, either $x \oplus y = x$ or $x \oplus y = y$.

\medskip
We shall consider below infinite sums with~$\oplus$.
A semiring~$\Semiring$ is called \emph{complete} 
if for every family
$(x_i)_{i \in I}$ %$\{ x_i \mid i \in I \}$
of elements of $\dom(\Semiring)$ over an index set $I \subset \mathbb{N}$,
the infinite sum $\bigoplus_{i \in I} x_i$
is well-defined and in $\dom(\Semiring)$,
and the following properties hold:
\begin{description}
\item[$i.$]
\emph{infinite sums extend finite sums:}
$\displaystyle\bigoplus_{i \in \emptyset} x_i = \zero$,\quad 
      $\forall j\in \mathbb{N}, \displaystyle\bigoplus_{i \in \{ j \}} x_i = x_j$,\quad
      $\forall j, k\in \mathbb{N}, j\neq k, 
      \displaystyle\bigoplus_{i \in \{ j, k \}} x_i = x_j \oplus x_k$,
%
\item[$ii.$]
\emph{associativity and commutativity:}
for all $I \subseteq \mathbb{N}$
and all partition $(I_{j})_{j \in J}$ of $I$, %\subseteq \mathbb{N}$, 
\(
\displaystyle
\bigoplus_{j \in J}\bigoplus_{i \in I_j} x_i = 
\bigoplus_{i \in I} x_i
\),
%
\item[$iii.$] 
\emph{distributivity of product over infinite sum:}\\
for all $I \subseteq \mathbb{N}$,
\(
\displaystyle
\bigoplus_{i \in I} (x \otimes y_i) = x \otimes \bigoplus_{i\in I} y_i\), and
\(
\displaystyle
\bigoplus_{i \in I} (x_i \otimes y) = (\bigoplus_{i \in I} x_i ) \otimes y\).
\end{description}



\subsection{Label Theory}
\label{section:symbols}

Let $\Sigma$ and $\Delta$ be respectively an input and output \emph{alphabets}, 
which are countable (finite or infinite) sets of symbols, 
and let $\Semiring$ be a commutative semiring.

\noindent 
A \emph{label theory} is made of 4 recursively enumerable sets:
%$\Phi_0$, \Phi_\Sigma \uplus \Phi_\Delta \uplus \Phi_{\Sigma, \Delta}$ 
$\Phi_\epsilon \subseteq \Semiring$, % containing constant functions valued in $\Semiring$, 
$\Phi_\Sigma$ and $\Phi_\Delta$, 
containing unary functions in $\Sigma \to \Semiring$, resp. $\Delta \to \Semiring$, 
and $\Phi_{\Sigma, \Delta}$  containing binary functions in $\Sigma \times \Delta \to \Semiring$.
Moreover, we assume that these sets are closed under the operators 
$\oplus$ and $\otimes$ of~$\Semiring$.
More precisely, for all $\phi, \phi' \in \Phi_\Sigma$ all 
$\psi, \psi' \in \Phi_\Delta$, 
and $\eta, \eta' \in \Phi_{\Sigma, \Delta}$, the function
\begin{description}
\item $\phi \otimes \phi' : x \mapsto \phi(x) \otimes \phi'(x)$ belongs to $\Phi_\Sigma$, 
\item $\psi \otimes \psi' : y \mapsto \psi(y) \otimes \psi'(y)$ belongs to $\Phi_\Delta$,
\item $\phi \otimes \eta : x, y \mapsto \phi(x) \otimes \eta(x, y)$ belongs to $\Phi_{\Sigma, \Delta}$,
\item $\eta \otimes \psi : x, y \mapsto \eta(x, y) \otimes \psi(y)$ belongs to $\Phi_{\Sigma, \Delta}$,
\item $\eta \otimes \eta' : x, y \mapsto \eta(x, y) \otimes \eta'(x, y)$ belongs to $\Phi_{\Sigma, \Delta}$.
\end{description}
The same also holds for the binary sum operator $\oplus$.

\noindent
\marginpar{maybe not necessary}
Finally, it is assumed that the codomain of every function of $\Phi_\Sigma$ and $\Phi_\Delta$
is a subset of $\Phi_\epsilon$.
and all partial applications of functions $\Phi_{\Sigma, \Delta}$, 
resp.  $f_a: y \mapsto f(a, y)$ for $a \in \Sigma$ and $y \in \Delta$
and  $f_b: x \mapsto f(x, b)$ for $b \in \Delta$ and $x \in \Sigma$, 
belong resp. to $\Phi_\Sigma$ and $\Phi_\Delta$.

%(SWT)
\begin{definition}
A \emph{symbolic-weighted transducer} $T$ over the input and output alphabet $\Sigma$ and $\Delta$ 
and the semiring $\Semiring$ is a tuple
$T = \< Q, \init, \wei, \final >$,
where $Q$ is a finite set of states, 
$\mathsf{in} : Q \to \Semiring$, 
respectively $\mathsf{out} : Q \to \Semiring,$
are functions defining the weight for entering, 
respectively leaving, a state, 
and $\wei$ is a transition %weight 
function from $Q \times Q$ into %$\Phi$.
$\< \Phi_0, \Phi_\Sigma, \Phi_\Delta, \Phi_{\Sigma, \Delta}>$.
\end{definition}
%
We extend the above transition function into a function from
$Q \times (\Sigma \cup \{ \epsilon \}) \times (\Delta \cup \{ \epsilon \}) \times Q$
into $\Semiring$, also called $\wei$ for simplicity, such that 
for all $q, q' \in Q$, $a \in \Sigma$, $b \in \Delta$, 
and with 
$\< \phi_\epsilon, \phi_\Sigma, \phi_\Delta, \phi_{\Sigma, \Delta}> = \wei(q, q')$, 
\[
\begin{array}{rcl}
\wei(q, \epsilon, \epsilon, q') & = & \phi_\epsilon\\
\wei(q, a, \epsilon, q') & = & \phi_\Sigma(a)\\
\wei(q, \epsilon, b, q') & = & \phi_\Delta(b)\\
\wei(q, a, b, q') & = & \phi_{\Sigma, \Delta}(a, b)
\end{array}      
\]
%More precisely, $Q \times Q$, 
%resp. $Q \times \Sigma \times Q$,
%$Q \times \Delta \times Q$,
%$Q \times \Sigma \times \Delta \times Q$,
%into resp. $\Phi_0$, $\Phi_\Sigma$, $\Phi_\Delta$ $\Phi_{\Sigma, \Delta}$.

These functions $\phi$ act as guards for the transducer's transitions, 
preventing a transition when they return the absorbing $\zero$ of $\Semiring$.

The symbolic-weighted transducer $T$ defines a mapping 
from the pairs of strings of $\Sigma^* \times \Delta^*$ 
into the weights of~$\Semiring$,
based on the following intermediate function $\weight_T$
defined recursively for every $q, q' \in Q$, 
%$a \in \Sigma$, $b \in \Delta$ 
for every strings of $s \in \Sigma^*$, $t \in \Delta^*$:
\[
\begin{array}{rccl}
\weight_T(q, s, t, q') & = & & \wei(q, \epsilon, \epsilon, q')\\
 & & \oplus \displaystyle\bigoplus_{\begin{array}{c}
                                      \scriptstyle q'' \in Q\\[-2pt]
                                      \scriptstyle s = au, a \in \Sigma
                                      \end{array}} &
     \wei(q, a, \epsilon, q'') \otimes \weight_T(q'', u, t, q')\\
 & & \oplus \displaystyle\bigoplus_{\begin{array}{c}
                                    \scriptstyle q'' \in Q\\[-2pt]
                                    \scriptstyle t = bv, b \in \Delta
                                    \end{array}} &
      \wei(q, \epsilon, b, q'') \otimes \weight_T(q'', s, v, q')\\ 
      & & \oplus \displaystyle\bigoplus_{\begin{array}{c}
                                         \scriptstyle q'' \in Q\\[-2pt]
                                         \scriptstyle s = au, a \in \Sigma\\[-2pt]
                                         \scriptstyle t = bv, b \in \Delta
                                         \end{array}} &
      \wei(q, a, b, q'') \otimes \weight_\A(q'', u, v, q')\\ 
\end{array}
\]
Recall that by convention, an empty sum with $\oplus$ is $\zero$. 
%
\noindent
The weight associated by $T$ to  $\< s, t> \in \Sigma^* \times \Delta^*$
is then defined as follows: 
\[
T(s, t)  = 
\displaystyle\bigoplus_{q, q' \in Q} \mathsf{in}(q) 
\mathop{\otimes} \weight_T(q, s, t, q') \mathop{\otimes} \mathsf{out}(q').
\]

A \emph{symbolic weighted automata} (\SWA) $A = \< Q, \init, \weight, \final >$
over $\Sigma$ and $\Semiring$ 
is defined in a similar way by simply omitting the output symbols,
\ie $\wei$ is a function of~$Q \times Q$ into %$\Phi$.
$\< \Phi_0, \Phi_\Sigma >$, 
or equivalently from~$Q \times (\Sigma \cup \{ \epsilon \}) \times Q$ into~$\Semiring$.


\begin{proposition}
Given a \SWT $T$ 
over $\Sigma$, $\Delta$ and $\Semiring$, 
and a word $s \in \Sigma^*$, 
one can construct a \SWA 
$A_{s, T}$ such that for all $t \in \Delta^*$, 
$A_{s, T}(t) = T(s, t)$.
\end{proposition}
The construction time and size of $A_{s, T}$ are $O(| s | . \| T \|)$.



\section{SW Visibly Pushdown Automata}
\label{section:SWVPA}
The following model generalizes Symbolic VPA~\cite{DAntonyAlur14SVPDA}
from Boolean semirings to arbitrary semiring weight domains.

Let $\Sigma$ be a countable alphabet 
%finite (large) or infinite,
that we assume partitioned into :
\begin{itemize}
\item a set $\Sigmai$ of \emph{internal symbols} denoted $a$,
\item a set $\Sigmac$ of \emph{call symbols} denoted $\call{a}$,
\item a set $\Sigmar$ of \emph{return symbols} denoted $\return{a}$.
\end{itemize}
Let $\Semiring$ be a commutative semiring and  
let  $(\Phi_\epsilon, \Phic, \Phir, \Phicr)$ be a label theory over $\Semiring$
%In order to simplify notations, %and following the definition of Section~\ref{section:transducer}, 
%we shall write respectively 
where $\Phic$, $\Phir$ and~$\Phicr$ stand respectively 
for~$\Phi_\Sigmac$, $\Phi_\Sigmar$ and~$\Phi_{\Sigmac, \Sigmar}$.
%
Moreover, we extend this theory with a set $\Phii$ 
of unary functions in $\Sigmai \to \Semiring$,
closed under $\oplus$ and $\otimes$.

\begin{definition}
A \emph{Symbolic Weighted Visibly Pushdown Automata} (\SWVPA)~$A$ 
over  $\Sigma = \Sigmai \uplus \Sigmac \uplus \Sigmar$ and $\Semiring$ is a tuple
$T = \< Q, P, \init, \weii, \weic, \weir, \weie, \final >$,
where $Q$ is a finite set of states, 
$P$ is a finite set of stack symbols, 
$\mathsf{in} : Q \to \Semiring$, 
respectively $\mathsf{out} : Q \to \Semiring,$
are functions defining the weight for entering, 
respectively leaving, a state, 
and 
$\weii : Q \times Q \to \Phii$,  
$\weic : Q \times Q \times P \to \Phic$,  
$\weir : Q \times P \times Q \to \Phicr$,  
$\weie : Q \times Q \to \Phir$,  
are transition functions.
\end{definition}
%
Similarly as in Section~\ref{section:transducer}, 
we extend the above transition functions as follows
for all $q, q' \in Q$, $p \in P$, 
$a \in \Sigmai$, 
$\call{c} \in \Sigmac$, 
$\return{r} \in \Sigmar$, 
overloading the names for simplicity:
\[
\begin{array}{lll}
\weii: Q \times \Sigmai \times Q \to \Semiring & 
\weii(q, a, q') = \phi_\mathsf{i}(a) & 
\mathrm{where~} \phi_\mathsf{i} = \weii(q, q'),\\
%
\weic: Q \times \Sigmac \times Q \times P \to \Semiring & 
\weic(q, \call{c}, q', p) = \phi_\mathsf{c}(\call{c}) & 
\mathrm{where~} \phi_\mathsf{c} = \weic(q, q', p),\\
%
\weir: Q \times \Sigmac \times P \times \Sigmar \times Q \to \Semiring & 
\weir(q, {\call{c}},  p, {\return{r}}, q') = \phi_\mathsf{r}({\call{c}},  {\return{r}}) & 
\mathrm{where~} \phi_\mathsf{r} = \weir(q, p, q'),\\
%
\weie: Q \times \Sigmar \times Q \to \Semiring & 
\weie(q, {\return{r}}, q') = \phi_\mathsf{e}({\return{r}}) &
\mathrm{where~} \phi_\mathsf{e} = \weie(q, q').\\
\end{array}      
\]

\noindent
The intuition is the following for the above transitions.
\begin{description}
\item $\weii$ : read the input internal symbol $a$, change state to $q'$.
\item $\weic$ : read the input call symbol $\call{c}$, push it to the stack along with $p$, change state to $q'$.
\item $\weir$ : when the stack is not empty, 
      read and pop from stack a pair made of $\call{c}$ and $p$, 
      read the input return symbol $\return{r}$, change state to $q'$.
      In this case, the weight function $\phi_\mathsf{r}$ 
      computes the value of matching between the call and return symbols.
      It might be $\zero$ to express that the symbols do not match.
\item $\weie$ : when the stack is empty, 
      read the input symbol $\call{r}$, change state to $q'$.
\end{description}

We give now a formal definition of these transitions of the automaton $A$
in term of a weight value
computed by an intermediate function $\weight_A$.
In the case of a pushdown automaton, a configuration is composed 
of a state $q \in Q$ and a stack content $\theta \in \Theta^*$, where $\Theta = \Sigmac \times P$.
Therefore, $\weight_A$ is a function from 
$Q \times \Theta^* \times \Sigma^* \times Q \times \Theta^*$ into $\Semiring$.
\[
\begin{array}{rcl}
\weight_A\bigl(\config{q}{\theta}, a u, \config{q'}{\theta'}\bigr) & = & 
 {\displaystyle\bigoplus_{q'' \in Q}} \weii(q, a, q'') 
 \otimes \weight_A\bigl(\config{q''}{\theta}, u, \config{q'}{\theta'}\bigr)\\
%
\weight_A\bigl(\config{q}{\theta}, {\call{c}} u, \config{q'}{\theta'}\bigr) & = & 
 {\displaystyle\bigoplus_{\begin{array}{c}
                          \scriptstyle q'' \in Q\\[-2pt]
                          \scriptstyle p \in P
                          \end{array}}}
 \weic\bigl(q, {\call{c}}, q'', p\bigr) 
 \otimes \weight_A\bigl(\config{q''}{{\call{c}}\, p\cdot \theta}, u, \config{q'}{\theta'}\bigr)\\[1mm]
%
\weight_A\bigl(\config{q}{{\call{c}}\, p\cdot \theta}, {\return{r}} u, \config{q'}{\theta'}\bigr) & = & 
 {\displaystyle\bigoplus_{q'' \in Q}} 
 \weir\bigl(q, {\call{c}}, p, {\return{r}}, q''\bigr) 
 \otimes \weight_A\bigl(\config{q''}{\theta}, u, \config{q'}{\theta'}\bigr)\\
%
\weight_A\bigl(\config{q}{\bot}, {\return{r}} u, \config{q'}{\theta'}\bigr) & = & 
 {\displaystyle\bigoplus_{q'' \in Q}} \weie(q, {\return{r}}, q'') 
 \otimes \weight_A\bigl(\config{q''}{\bot}, u, \config{q'}{\theta'}\bigr)\\
\end{array}
\]
where $\bot$ denotes the empty stack and ${\call{c}}\, p\cdot \theta$ 
denotes a stack with the pair made of ${\call{c}}$ and $p$ on its top and $\theta$
as the rest of stack.

\noindent
The weight associated by $A$ to $s \in \Sigma^*$
is then defined as follows,
following empty stack semantics: 
\[
A(s)  = 
{\displaystyle\bigoplus_{q, q' \in Q}} \textstyle
\mathsf{in}(q) \mathop{\otimes} 
\weight_A\bigl(\config{q}{\bot}, s, \config{q'}{\bot}\bigr) 
\mathop{\otimes} \mathsf{out}(q').
\]



\subsection{1-best Computation}

Regarding the infinite sum operator, it must hold that
$\bigoplus_{x \in \Phi_\Sigma} \phi(x)$, 
$\bigoplus_{y \in \Phi_\Delta} \psi(y)$, and  




% The weight of a transition acts as a guard: 
% a transition is activated for a symbol $`a`$ iff its weight $`\phi(a) \neq 0`$ (or $`\psi(a, b) \neq 0`$), 
% the absorbing element.


property of bounded convexity of weight functions:






\section{Application}
Symbolic Automated Music Transcription
and analysis of music performances

\subsection{Time Scales}
Real-Time Unit (RTU) = seconds

\noindent 
Musical-Time Unit (MTU) = number of measures

\noindent 
conversion via tempo value

\subsection{Representation of Music Performances}
A musical performance is represented symbolically as a finite sequence of timestamped events.
similar to piano roll representation~\cite{Muller15fundamentals} chap.1 

\noindent
We consider an infinite alphabet $\Sigma$ of MIDI-like 
events made of:
\begin{itemize}
\item a timestamp in RTU\\ %(real or int = nb samples ?)
      or IOI in RTU
\item a pitch value in 0..127
\item ON | OFF flag
\item a velocity value in 0..127
\end{itemize}


\subsection{Representation of Music Scores}
we consider here the case of monodies for the sake of simplificity.

A music score ris epresented as a structured word
= sequence of quantified events + markups,
see nested words~\cite{AlurMadhusudan09nested}.

We consider an alphabet $\Delta$, every symbol of which is 
composed of a tag, in a finite set $\Xi$, 
and an IOI duration value.
Moreover, $\Delta$ is partitioned into 
$\Delta = \Deltai \uplus \Deltac \uplus \Deltar$, 
like in Section~\ref{section:SWVPA}.

\noindent
The elements of $\Deltai$ represent events:
% (infinite alphabet of internal symbols) made of:
whose tags are one of:
\begin{itemize}
\item continuation (0) : tie or dot
\item note, grace note (pitch) or chord (pitch+)
\item rest
\item ...
\end{itemize}

\noindent
The elements of $\Deltac$ and $\Deltar$ are 
markups for the representation of groups of events
(linearization of rhythm trees \cite{jacquemard:hal-01138642}...)
- parentheses for time divisions : tuplets, bars...
tag %labels: 
contain info such as tuple number, beaming policy...

\noindent
The date or duration of events, in MTU (rational), 
can be computed with the markups and tags (e.g. grace note has duration 0).

\noindent
There are simultaneous events, since grace notes has duration 0. They are ordered.

\noindent
Finite bound on the number of duration ratio. ?


\subsection{Performance/Score Distance Computation}
with a transducer



%%%%%%%%%%%%%%%%%%%%%%%%%%%%%%%%%%%%%%%%%%%%%%%%%%%%%%%%%%%%%%%%%%%%%%%%%%%%%%%%
%% BIBLIO                                                                     %%
%%%%%%%%%%%%%%%%%%%%%%%%%%%%%%%%%%%%%%%%%%%%%%%%%%%%%%%%%%%%%%%%%%%%%%%%%%%%%%%%
%\bibliographystyle{plain}
%\bibliographystyle{plainurl} 
\bibliographystyle{abbrv}
%\bibliographystyle{splncs04}
\bibliography{references}





%%%%%%%%%%%%%%%%%%%%%%%%%%%%%%%%%%%%%%%%%%%%%%%%%%%%%%%%%%%%%%%%%%%%%%%%%%%%%%%%
%% APPENDIX                                                                   %%
%%%%%%%%%%%%%%%%%%%%%%%%%%%%%%%%%%%%%%%%%%%%%%%%%%%%%%%%%%%%%%%%%%%%%%%%%%%%%%%%
\newpage
\appendix 

\section{Edit-Distance}

%\subsection{Distance between words or languages}
...algebraic definition of edit-distance of Mohri, in \cite{Mohri03EDWA}
% Mehryar Mohri 
% Edit-distance of weighted automata: General definitions and algorithms
% International Journal of Foundations of Computer Science 14.06 (2003): 957-982.
distance $d$ over $\Sigma^* \times \Sigma^*$ 
into a semiring  $\Semiring = ( \Semiring, \oplus, \zero, \otimes, \one)$.

%\noindent
Let $\Omega = \Sigma \cup \{ \epsilon \} \times \Sigma \cup \{ \epsilon \} \setminus \{ (\epsilon, \epsilon) \}$,
and let $h$ be the morphism from $\Omega^*$ into $\Sigma^* \times \Sigma^*$  
defined over the concatenation of strings of $\Sigma^*$ (that removes the $\epsilon$'s).
%
\noindent
An \emph{alignment} between 2 strings  $s, t \in \Sigma^*$ is an element $\omega \in \Omega^*$ 
such that $h(\omega) = (s, t)$.
%
\noindent
We assume a base cost function $\Omega$ : $\delta: \Omega \to S$, extended to $\Omega^*$ as follows  
(for $\omega \in \Omega^*$): 
\(
\displaystyle\delta(\omega) = \bigotimes_{0 \leq i < |\omega|} \delta(\omega_i)
\).

\noindent
\begin{definition}
For  $s, t \in \Sigma^*$, the edit-distance between $s$ and $t$ is  
\( 
d(s, t) = \displaystyle\bigoplus_{\omega \in \Omega^*\, h(\omega) = (s, t)} \delta(\omega)
\).
\end{definition}

e.g. Levenstein edit-distance: $S$ is min-plus and $\delta(a, b) = 1$ for all $(a, b) \in \Omega$.


%\paragraph{Distance between a word and a regular language}


\end{document}

