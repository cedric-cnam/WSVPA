\documentclass[a4paper,11pt]{article}
%\setcounter{page}{1}

\usepackage[T1]{fontenc}
\usepackage[utf8]{inputenc}
\usepackage[english]{babel}

\usepackage{hyperref}
%\usepackage[bookmarks,bookmarksnumbered,naturalnames,plainpages=false]{hyperref}
%usepackage{url}

% for footnote ref
\usepackage{refcount} 

% symbols
\usepackage{amsmath} 
\usepackage{amssymb} 
\usepackage{amsbsy}
\usepackage{bbold}
\usepackage{latexsym}
%\usepackage{amsfonts}
\usepackage{stmaryrd}
%\usepackage{mathabx}
%\usepackage{MnSymbol}
\usepackage{harmony} % simple music fonts
\usepackage{mathtools} % for arrows
%\usepackage{mathptmx}

%% theorem envs
\usepackage{theorem}
\newtheorem{theorem}{Theorem} %[section]
\newtheorem{definition}[theorem]{Definition}
\newtheorem{lemma}[theorem]{Lemma}
\newtheorem{corollary}[theorem]{Corollary}
\newtheorem{proposition}[theorem]{Proposition}
\newenvironment{proof}{\vspace{-2ex}{\it Proof. }}{\hspace*{\fill} $\Box$\smallskip }
\theorembodyfont{\slshape}
\newtheorem{example}[theorem]{Example}
\newtheorem{remark}[theorem]{Remark}


% extension of enumerate env. (style for displaying counters)
% \usepackage{enumerate} 


%% pictures
% \usepackage{graphicx} 
% \DeclareGraphicsExtensions{.pdf,.png,.jpg}
% \graphicspath{fig/}

%% PGF, Tikz
%% \usepackage{pgfplots}
%% \usepgfplotslibrary{dateplot}
%% \usepackage{pgf,pgfarrows,pgfnodes, pgfautomata}
% \usepackage{tikz}
%% \usetikzlibrary{arrows}
%% \usetikzlibrary{calc}
%% \usetikzlibrary{snakes}
%% \usetikzlibrary{backgrounds}
% \usetikzlibrary{trees}
%% \usetikzlibrary{automata}
%% \usetikzlibrary{positioning}
%% \usetikzlibrary{matrix}
%% \usetikzlibrary{patterns}
%% \usetikzlibrary{shapes}

%% for new macros
\usepackage{xspace}

%% arrows etc
%
%% Extensible arrows from amsmath
%\newcommand{\lrstep}[2]{\xrightarrow{#1}{#2}}    %\mathrel ? 
%\newcommand{\rlstep}[2]{\xleftarrow{#1}{#2}}
%\newcommand{\eqstep}[2]{\xleftrightarrow{#1}{#2}}
%\newcommand{\mapstep}[2]{\mathop{\xmapsto[\scriptstyle #2]{\scriptstyle #1}}}
\makeatletter
\newcommand{\xleftrightarrow}[2][]{\ext@arrow 3359\leftrightarrowfill@{#1}{#2}}
\newcommand{\xdashrightarrow}[2][]{\ext@arrow 0359\rightarrowfill@@{#1}{#2}}
\newcommand{\xdashleftarrow}[2][]{\ext@arrow 3095\leftarrowfill@@{#1}{#2}}
\newcommand{\xdashleftrightarrow}[2][]{\ext@arrow 3359\leftrightarrowfill@@{#1}{#2}}
\def\rightarrowfill@@{\arrowfill@@\relax\relbar\rightarrow}
\def\leftarrowfill@@{\arrowfill@@\leftarrow\relbar\relax}
\def\leftrightarrowfill@@{\arrowfill@@\leftarrow\relbar\rightarrow}
\def\arrowfill@@#1#2#3#4{%
  $\m@th\thickmuskip0mu\medmuskip\thickmuskip\thinmuskip\thickmuskip
   \relax#4#1
   \xleaders\hbox{$#4#2$}\hfill
   #3$%
}
\makeatother


%% Extensible arrows from pgf/tikz
\usetikzlibrary{arrows}
\usetikzlibrary{cd}
\makeatletter
\newbox\xrat@below
\newbox\xrat@above
\newcommand{\yrightarrowhook}[2][]{%
  \setbox\xrat@below=\hbox{\ensuremath{\scriptstyle #1}}%
  \setbox\xrat@above=\hbox{\ensuremath{\scriptstyle #2}}%
  \pgfmathsetlengthmacro{\xrat@len}{max(\wd\xrat@below,\wd\xrat@above)+.7em}%
  \mathrel{\tikz [right hook->,baseline=-.75ex]
                 \draw (0,0) -- node[below=-1.5pt] {\box\xrat@below}
                                node[above=-1.5pt] {\box\xrat@above}
                       (\xrat@len,0) ;}}
\newcommand{\yrightarrow}[2][]{%
  \setbox\xrat@below=\hbox{\ensuremath{\scriptstyle #1}}%
  \setbox\xrat@above=\hbox{\ensuremath{\scriptstyle #2}}%
  \pgfmathsetlengthmacro{\xrat@len}{max(\wd\xrat@below,\wd\xrat@above)+.7em}%
  \mathrel{\tikz [->,baseline=-.75ex]
                 \draw (0,0) -- node[below=-1.5pt] {\box\xrat@below}
                                node[above=-1.5pt] {\box\xrat@above}
                       (\xrat@len,0) ;}}
\makeatother


%% Arrows
\def\Reduction#1#2#3#4{%
\mathrel{\raise1.0ex\hbox{%
\vtop{\ialign{##\crcr%
\raise0.0ex\hbox{$\hfil\scriptstyle{\ #1\ }\hfil$}\crcr%
\noalign{\nointerlineskip}%
\rightarrowfill\crcr%
\noalign{\nointerlineskip}%
\raise0.0ex\hbox{$\hfil\scriptstyle{\ #2\ }\hfil$}\crcr}}}{}^{#3}_{#4}}}
%
\def\Leduction#1#2#3#4{%
\mathrel{\raise1.0ex\hbox{%
\vtop{\ialign{##\crcr%
\raise0.0ex\hbox{$\hfil\scriptstyle{\ #1\ }\hfil$}\crcr%
\noalign{\nointerlineskip}%
\leftarrowfill\crcr%
\noalign{\nointerlineskip}%
$\hfil\scriptstyle{\ #2\ }\hfil$\crcr}}}{}^{#3}_{#4}}}
%
%\def\hookxrightarrow[#1]#2{{\lhook\hspace{-0.20em}\xrightarrow[#1]{#2}}}
\def\hookReduction#1#2#3#4{%
%\lhook\joinrel\hspace{-0.35em}
\mathrel{\raise1.2ex\hbox{%
\vtop{\ialign{##\crcr%
\raise0.0ex\hbox{$\hfil\scriptstyle{\ #1\ }\hfil$}\crcr%
\noalign{\nointerlineskip}%
$\lhook\joinrel$\hspace{-0.35em}
\rightarrowfill\crcr%
\noalign{\nointerlineskip}%
$\hfil\scriptstyle{\ #2\ }\hfil$\crcr}}}{}^{#3}_{#4}}}
%
\def\hoookReduction#1#2#3#4{%
\lhook\joinrel\hspace{-0.50em}
\raise0.85ex\hbox{%
\vtop{\ialign{##\crcr%
\raise0.4ex\hbox{$\hfil\scriptstyle{\ #1\ }\hfil$}\crcr%
\noalign{\nointerlineskip}%
\rightarrowfill\crcr%
\noalign{\nointerlineskip}%
$\hfil\scriptstyle{\ #2\ }\hfil$\crcr}}}{}^{#3}_{#4}}

%\def\reach#1#2{\mathop{#1}[#2]}

%% rewrite steps
%% \frew#1#2#3#4#5#6#7#8
\def\frew#1#2#3#4#5#6#7#8{
\setbox0=\hbox{$#6 #7 #1 #8$}%
\setbox1=\hbox{$#6 #7 #2 #8$}%
\ifdim \wd0>\wd1 \rlap{\rlap{\hbox to \wd0{#5}}%
                            {\hbox to\wd0{\hfil\lower #3\box1\relax\hfil}}}{\raise #4\box0}%
\else \rlap{\rlap{\hbox to \wd1{#5}}{\hbox to\wd1{\hfil\raise #4\box0\relax\hfil}}}{\lower #3\box1}%
\fi
}
%% \fstep
\def\fstep#1#2#3#4#5{\mathchoice{\frew{#1}{#2}{1.10ex}{1.20ex}{#5}{\scriptstyle}{#3}{#4}}%
                                {\frew{#1}{#2}{0.82ex}{1.20ex}{#5}{\scriptstyle}{#3}{#4}}%
                                {\frew{#1}{#2}{0.51ex}{0.82ex}{#5}{\scriptscriptstyle}{#3}{#4}}%
                                {\frew{#1}{#2}{0.51ex}{0.69ex}{#5}{\scriptscriptstyle}{#3}{#4}}}
%% \lrstep, \rlstep, \eqstep
% #1 top label
% #2 bottom_right label
\newcommand{\lrstep}[2]{\mathrel{\fstep{#1}{#2}{\;\>}{\>\>\;}{\rightarrowfill}}}
\newcommand{\lrsteptc}[2]{\mathrel{\fstep{#1\ }{#2\ }{\;\>}{\>\>\;}{\rightarrowfill$^*$}}}
\newcommand{\rlstep}[2]{\mathrel{\fstep{#1}{#2}{\;\>\>}{\;\>}{\leftarrowfill}}}
\newcommand{\eqstep}[2]{\mathrel{\fstep{#1}{#2}{\;\>}{\>\;}{\rlap{\leftarrowfill}{\rightarrowfill}}}}

%% \fstepd   ad hoc.... to avoid hidden overline
\def\fstepd#1#2#3#4#5{\mathchoice{\frew{#1}{#2}{1.10ex}{1.20ex}{#5}{\scriptstyle}{#3}{#4}}%
                                {\frew{#1}{#2}{1.12ex}{1.20ex}{#5}{\scriptstyle}{#3}{#4}}%
                                {\frew{#1}{#2}{0.51ex}{0.82ex}{#5}{\scriptscriptstyle}{#3}{#4}}%
                                {\frew{#1}{#2}{0.51ex}{0.69ex}{#5}{\scriptscriptstyle}{#3}{#4}}}
\newcommand{\lrstepd}[2]{\mathrel{\fstepd{#1}{#2}{\;\>}{\>\>\;}{\rightarrowfill}}}


%% Misc macros

\def\ie{\textit{i.e.}\xspace}
\def\eg{\textit{e.g.}\xspace}
\def\wrt{\textit{wrt}\xspace}
%\def\wlog{\textit{wlog}\xspace}
\def\etc{\textit{etc}\xspace}

\def\<#1>{\langle #1 \rangle}
\newcommand{\pair}[2]{\langle{#1}, {#2}\rangle}
\newcommand{\A}{\mathcal{A}}
\newcommand{\B}{\mathcal{B}}
\newcommand{\D}{\mathbb{D}}
\newcommand{\E}{\mathbb{E}}
\newcommand{\W}{\mathbb{W}}

\newcommand{\Semiring}{\mathbb{S}}
\newcommand{\zero}{\mathbb{0}}
\newcommand{\one}{\mathbb{1}}
\newcommand{\dom}{\ensuremath{\mathit{dom}}}

\newcommand{\weight}{\mathsf{weight}}
\newcommand{\init}{\mathsf{in}}
\newcommand{\final}{\mathsf{out}}


%\sloppy

\title{Weighted Visibly Pushdown Automata and Automated Music Transcription}
%\author{Florent Jacquemard}
%\institute{INRIA \& CNAM, Paris, France\\
%\email{florent.jacquemard@inria.fr}}

%\titlerunning{WVPA \& AMT}
%\authorrunning{Florent Jacquemard}

\date{\today}
 
\begin{document}
\thispagestyle{empty}
\maketitle
\abstract{\ldots}


Symbolic Weighted (SW) extension of symbolic automata where...



\paragraph{Semirings.}
We shall consider semiring domains for weight values.
%
A \emph{semiring} $\< \Semiring, \oplus, \zero, \otimes, \one>$ 
is a structure with a domain~$\Semiring$,
equipped with two associative
binary operators~$\oplus$ and $\otimes$
with respective neutral elements $\zero$ and $\one$ and such that:
%$\< \mathbb{S}, \oplus, \zero>$ is a commutative monoid
%$\< \mathbb{S}, \otimes, \one>$ is a monoid
$\oplus$ is commutative, 
$\otimes$ distributes over~$\oplus$:  $\forall x, y, z \in \mathbb{S}$,
$x \otimes (y \oplus z) = (x \otimes y) \oplus (x \otimes z)$, 
%  and $(x \oplus y) \otimes z = (x \otimes z) \oplus (y \otimes z)$;
and $\zero$ is absorbing for~$\otimes$: 
$\forall x\in \mathbb{S}$, $\zero \otimes x = x \otimes \zero = \zero$.
%Components of a semiring~$\Semiring$ may be subscripted by~$\Semiring$ when needed.
%We simply write $x \in \Semiring$ to mean $x \in \mathbb{S}$.
%
In the application presented in this paper, intuitively,
$\oplus$ selects an optimal value amongst two values and 
$\otimes$ combines two values into a single value.

%\noindent %properties:
A semiring $\Semiring$ is \emph{commutative} if $\otimes$ is commutative.
%\francesco{How we deal with that since it is non commutative?}
It \emph{bounded}~\cite{Mohri02semiring} if $\forall x \in \dom(\Semiring), \one \oplus x = \one$,
and \emph{idempotent} if for all $x \in \Semiring$, $x \oplus x = x$.
%We restrict in the following of \emph{idempotent} semirings.
Note that every bounded semiring is idempotent:
by boundedness, 
$\one \oplus \one = \one$, and idempotency follows by multiplying
both sides by $x$ and distributing. %


%\noindent
A semiring $\Semiring$ 
is \emph{monotonic} \wrt a partial ordering~$\leq$ 
iff for all $x, y, z  \in \Semiring$,  $x \leq y$ implies
$x \oplus z \leq y \oplus z$,
$x \otimes z \leq y \otimes z$
and $z \otimes x \leq z \otimes y$,
%
and it is \emph{superior} %\wrt a partial ordering~$\leq$
\wrt~$\leq$ iff for all $x, y \in \Semiring$,  
$x \leq x \otimes y$ and 
$y \leq x \otimes y$~\cite{Huang08advanceddynamic}.
The latter property corresponds to the 
\emph{non-negative weights} condition in shortest-path algorithms~\cite{Dijkstra59anote}.
Intuitively, it means that combining elements always increase their weight. 
% always get worse in term of weight
Note that when $\Semiring$ is superior \wrt~$\leq$, then $\one \leq \zero$
and moreover, for all $x \in \Semiring$, $\one \leq x \leq \zero$.

Every idempotent semiring~$\Semiring$ induces 
a partial ordering~$\leq_\Semiring$ 
called the \emph{natural ordering} of~$\Semiring$
and defined by: 
%implicitly defined by the semiring $\Semiring$ 
for all $x$ and $y$,
$x \leq_\Semiring y \;\mbox{iff}\; x \oplus y = x$.
This ordering is sometimes defined in the opposite direction~\cite{DrosteKuich09semirings};
The above definition follows \cite{Mohri02semiring}, 
and coincides than the usual ordering on the Tropical semiring (\emph{min-plus}).
%
It holds that $\Semiring$ is {monotonic} \wrt~$\leq_\Semiring$.
An idempotent Semiring $\Semiring$~is called \emph{total} if
it~$\leq_\Semiring$ is total
\ie when for all $x, y \in \Semiring$, either $x \oplus y = x$ or $x \oplus y = y$.

We shall consider below infinite sums with~$\oplus$.
A semiring~$\Semiring$ is called \emph{complete} 
if for every family
$(x_i)_{i \in I}$ %$\{ x_i \mid i \in I \}$
of elements of $\dom(\Semiring)$ over an index set $I \subset \mathbb{N}$,
the infinite sum $\bigoplus_{i \in I} x_i$
is well-defined and in $\dom(\Semiring)$,
and the following properties hold:
\begin{description}
\item[$i.$]
\emph{infinite sums extend finite sums:}
$\displaystyle\bigoplus_{i \in \emptyset} x_i = \zero$,\quad 
      $\forall j\in \mathbb{N}, \displaystyle\bigoplus_{i \in \{ j \}} x_i = x_j$,\quad
      $\forall j, k\in \mathbb{N}, j\neq k, 
      \displaystyle\bigoplus_{i \in \{ j, k \}} x_i = x_j \oplus x_k$,
%
\item[$ii.$]
\emph{associativity and commutativity:}
for all $I \subseteq \mathbb{N}$
and all partition $(I_{j})_{j \in J}$ of $I$, %\subseteq \mathbb{N}$, 
\(
\displaystyle
\bigoplus_{j \in J}\bigoplus_{i \in I_j} x_i = 
\bigoplus_{i \in I} x_i
\),
%
\item[$iii.$] 
\emph{distributivity of product over infinite sum:}\\
for all $I \subseteq \mathbb{N}$,
\(
\displaystyle
\bigoplus_{i \in I} (x \otimes y_i) = x \otimes \bigoplus_{i\in I} y_i\), and
\(
\displaystyle
\bigoplus_{i \in I} (x_i \otimes y) = (\bigoplus_{i \in I} x_i ) \otimes y\).
\end{description}



\section{SW Visibly Pushdown Automata}
We follow the approach of~\cite{Mohri03EDWA} for the computation of distances...


\subsection{SW Automata and Transducers}
The following definition of weighted transducers over infinite alphabets 
generalizes weighted transducers over finite alphabets, see  e.g.~\cite{Mohri03EDWA}, 
by considering weight functions generalizing the guards of symbolic automata

Let $\Sigma$ and $\Delta$ be respectively an input and output \emph{alphabets}, 
which are finite or infinite sets of symbols, and let $\Semiring$ be a semiring.
%
A \emph{label theory} 
is a 4-uplet of recursively enumerable sets:
%$\Phi_0$, \Phi_\Sigma \uplus \Phi_\Delta \uplus \Phi_{\Sigma, \Delta}$ 
$\Phi_0$ containing constant functions valued in $\Semiring$, 
$\Phi_\Sigma$ and $\Phi_\Delta$, 
containing unary functions in $\Sigma \to \Semiring$, resp. $\Delta \to \Semiring$, 
and $\Phi_{\Sigma, \Delta}$  containing binary functions in $\Sigma \times \Delta \to \Semiring$.
Moreover, we assume that each of these sets is closed under $\oplus$ and $\otimes$, 
and all partial applications of functions $\Phi_{\Sigma, \Delta}$, 
resp. 
$f_a: y \mapsto f(a, y)$ for $a \in \Sigma$ and $y \in \Delta$
and 
$f_b: x \mapsto f(x, b)$ for $b \in \Delta$ and $x \in \Sigma$, 
belong resp. to $\Phi_\Sigma$ and $\Phi_\Delta$.


\begin{definition}
A weighted transducer $T$ over the input and output alphabet $\Sigma$ and $\Delta$ 
and the semiring $\Semiring$ is a tuple
$T = \< Q, \init, \weight, \final >$,
where $Q$ is a finite set of states, 
$\mathsf{in} : Q \to \Semiring$, 
respectively $\mathsf{out} : Q \to \Semiring,$
is a function defining the weight for entering, 
respectively leaving, a state, 
and $\weight$ is a transition %weight 
function of $Q \times Q$ into %$\Phi$.
$\< \Phi_0, \Phi_\Sigma, \Phi_\Delta, \Phi_{\Sigma, \Delta}>$.
\end{definition}
We extend the above transition function into a function from
$Q \times (\Sigma \cup \{ \epsilon \}) \times (\Delta \cup \{ \epsilon \}) \times Q$
into $\Semiring$, also called $\weight$ for simplicity, such that 
for all $q, q' \in Q$, $a \in \Sigma$, $b \in \Delta$, 
and with 
$\< \phi_\epsilon, \phi_\Sigma, \phi_\Delta, \phi_{\Sigma, \Delta}> = \weight(q, q')$, 
\[
\begin{array}{rcl}
\weight(q, \epsilon, \epsilon, q') & = & \phi_\epsilon\\
\weight(q, a, \epsilon, q') & = & \phi_\Sigma(a)\\
\weight(q, \epsilon, b, q') & = & \phi_\Delta(b)\\
\weight(q, a, b, q') & = & \phi_{\Sigma, \Delta}(a, b)
\end{array}      
\]
%More precisely, $Q \times Q$, 
%resp. $Q \times \Sigma \times Q$,
%$Q \times \Delta \times Q$,
%$Q \times \Sigma \times \Delta \times Q$,
%into resp. $\Phi_0$, $\Phi_\Sigma$, $\Phi_\Delta$ $\Phi_{\Sigma, \Delta}$.

These functions $\phi$ act as guards for the transducer's transitions, 
preventing a transition when they return the absorbing $\zero$ of $\Semiring$.



\subsection{Distance between words or languages}
distance $d$:
defined over $\Sigma^* \times \Sigma^*$ 
into a semiring  $\Semiring = ( \Semiring, \oplus, \zero, \otimes, \one)$.




\paragraph{Edit-Distance.}
...algebraic definition of edit-distance of Mohri, in \cite{Mohri03EDWA}
% Mehryar Mohri 
% Edit-distance of weighted automata: General definitions and algorithms
% International Journal of Foundations of Computer Science 14.06 (2003): 957-982.
%
\noindent
Let $\Omega = \Sigma \cup \{ \epsilon \} \times \Sigma \cup \{ \epsilon \} \setminus \{ (\epsilon, \epsilon) \}$,
and let $h$ be the morphism from $\Omega^*$ into $\Sigma^* \times \Sigma^*$  
defined over the concatenation of strings of $\Sigma^*$ (that removes the $\epsilon$'s).
%
\noindent
An \emph{alignment} between 2 strings  $s, t \in \Sigma^*$ is an element $\omega \in \Omega^*$ 
such that $h(\omega) = (s, t)$.
%
\noindent
We assume a base cost function $\Omega$ : $\delta: \Omega \to S$, extended to $\Omega^*$ as follows  
(for $\omega \in \Omega^*$): 
\(
\displaystyle\delta(\omega) = \bigotimes_{0 \leq i < |\omega|} \delta(\omega_i)
\).

\noindent
\begin{definition}
For  $s, t \in \Sigma^*$, the edit-distance between $s$ and $t$ is  
\( 
d(s, t) = \displaystyle\bigoplus_{\omega \in \Omega^*\, h(\omega) = (s, t)} \delta(\omega)
\).
\end{definition}

e.g. Levenstein edit-distance: $S$ is min-plus and $\delta(a, b) = 1$ for all $(a, b) \in \Omega$.


%\paragraph{Distance between a word and a regular language}



\subsection{SW Visibly Pushdown Automata}





\section{Application}
Symbolic Automated Music Transcription


\subsection{Representations}

\paragraph{Performance.}

\paragraph{Score.}


\subsection{Transducer for Distance Computation}



%%%%%%%%%%%%%%%%%%%%%%%%%%%%%%%%%%%%%%%%%%%%%%%%%%%%%%%%%%%%%%%%%%%%%%%%%%%%%%%%
%% BIBLIO                                                                     %%
%%%%%%%%%%%%%%%%%%%%%%%%%%%%%%%%%%%%%%%%%%%%%%%%%%%%%%%%%%%%%%%%%%%%%%%%%%%%%%%%
%\bibliographystyle{plain}
%\bibliographystyle{plainurl} 
\bibliographystyle{abbrv}
%\bibliographystyle{splncs04}
\bibliography{references}

\end{document}

