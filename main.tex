
% SWT without epsilon-transitions
% SWT do not read the other tape in "skip" transitions 10 or 01
%     (hence this version is simpler  than V2)
% SWA without epsilon-transitions
% SW-VPA read top of stack (when not empty) at every transition 
%        (6 kinds of transitions).

% default style article.cls
% !TEX root = main.tex

% default style article.cls
\documentclass[a4paper]{article}  %11pt
\setcounter{page}{1}

%% theorem environments
\usepackage{theorem}
\newtheorem{theorem}{Theorem} %[section]
\newtheorem{definition}[theorem]{Definition}
\newtheorem{lemma}[theorem]{Lemma}
\newtheorem{corollary}[theorem]{Corollary}
\newtheorem{proposition}[theorem]{Proposition}
\newenvironment{proof}{\vspace{-2ex}{\it Proof. }}{\hspace*{\fill} $\Box$\smallskip }
\theorembodyfont{\slshape}
\newtheorem{example}[theorem]{Example}
\newtheorem{remark}[theorem]{Remark}
\def\endex{\hspace*{\fill} $\Diamond$\smallskip }
\def\qed{}
{\theorembodyfont{\rmfamily} \theoremstyle{break} \newtheorem{algo}{Algorithm}}

\author{Florent Jacquemard\\
        INRIA \& CNAM, Paris, France\\
   \url{florent.jacquemard@inria.fr}
} % end author


% Springer llncs style
% https://www.springer.com/gp/computer-science/lncs/conference-proceedings-guidelines
%% !TEX root = main.tex

% https://www.springer.com/gp/computer-science/lncs/conference-proceedings-guidelines
\documentclass[runningheads]{llncs}

%\usepackage[T1]{fontenc}
%\usepackage[utf8]{inputenc}
\usepackage[english]{babel}


%% extra theorem environments
\usepackage{theorem}
%\theorembodyfont{\slshape}
%\newtheorem{example}[theorem]{Example}
%\newtheorem{remark}[theorem]{Remark}
\def\endex{\hspace*{\fill} $\Diamond$\smallskip }
{\theorembodyfont{\rmfamily} \theoremstyle{break} \newtheorem{algo}{Algorithm}}


% author
\author{Florent Jacquemard}
\institute{INRIA \& CNAM, Paris, France\\
    \email{florent.jacquemard@inria.fr}}
    
    

% EPTCS style
% http://www.eptcs.org
%% !TEX root = main.tex

% http://www.eptcs.org
\documentclass[submission,copyright,creativecommons]{eptcs}
\providecommand{\event}{GandALF 2021} % Name of the event you are submitting to
\usepackage{breakurl}                 % Not needed if you use pdflatex only.
\usepackage{underscore}               % Only needed if you use pdflatex.

%\usepackage[T1]{fontenc}
%\usepackage[utf8]{inputenc}
\usepackage[english]{babel}


%% theorem environments
\usepackage{theorem}
\newtheorem{theorem}{Theorem} %[section]
\newtheorem{definition}[theorem]{Definition}
\newtheorem{lemma}[theorem]{Lemma}
\newtheorem{corollary}[theorem]{Corollary}
\newtheorem{proposition}[theorem]{Proposition}
\newenvironment{proof}{\vspace{-2ex}{\it Proof. }}{\hspace*{\fill} $\Box$\smallskip }
\theorembodyfont{\slshape}
\newtheorem{example}[theorem]{Example}
\newtheorem{remark}[theorem]{Remark}
\def\endex{\hspace*{\fill} $\Diamond$\smallskip }
\def\qed{}
{\theorembodyfont{\rmfamily} \theoremstyle{break} \newtheorem{algo}{Algorithm}}

\author{Florent Jacquemard
\institute{INRIA \& CNAM, Paris, France\\
\email{florent.jacquemard@inria.fr}}
} % end author

\def\titlerunning{Symbolic Weighted Language Models and\\ Parsing over Infinite Alphabets}
\def\authorrunning{Florent Jacquemard}


 

\usepackage[T1]{fontenc}
\usepackage[utf8]{inputenc}
\usepackage[english]{babel}

\usepackage{hyperref}
%\usepackage[bookmarks,bookmarksnumbered,naturalnames,plainpages=false]{hyperref}
%usepackage{url}

% for footnote ref
\usepackage{refcount} 

% array and tabular
\usepackage{array}
\newcolumntype{C}[1]{>{\centering\arraybackslash\hspace{0pt}}p{#1}}

% extension of enumerate env. (style for displaying counters)
% \usepackage{enumerate} 

%% pictures
% \usepackage{graphicx} 
% \DeclareGraphicsExtensions{.pdf,.png,.jpg}
% \graphicspath{fig/}

%% PGF, Tikz
\usepackage{tikz-cd}
%% \usepackage{pgfplots}
%% \usepgfplotslibrary{dateplot}
%% \usepackage{pgf,pgfarrows,pgfnodes, pgfautomata}
% \usepackage{tikz}
% \usetikzlibrary{cd}
%% \usetikzlibrary{arrows}
%% \usetikzlibrary{calc}
%% \usetikzlibrary{snakes}
%% \usetikzlibrary{backgrounds}
% \usetikzlibrary{trees}
%% \usetikzlibrary{automata}
%% \usetikzlibrary{positioning}
%% \usetikzlibrary{matrix}
%% \usetikzlibrary{patterns}
%% \usetikzlibrary{shapes}

% symbols
\usepackage{amsmath} 
\allowdisplaybreaks
\usepackage{amssymb} 
\usepackage{amsbsy}
\usepackage{bbold}
\usepackage{latexsym}
%\usepackage{amsfonts}
\usepackage{stmaryrd}
%\usepackage{mathabx}
%\usepackage{MnSymbol}
\usepackage{harmony} % simple music fonts
\usepackage{mathtools} % for arrows
%\usepackage{mathptmx}

%% algorithms
%\usepackage{algorithm} 
%\usepackage{program} 

%% allows for page break within arrays
\usepackage{longtable}

%% arrows etc
%
%% Extensible arrows from amsmath
%\newcommand{\lrstep}[2]{\xrightarrow{#1}{#2}}    %\mathrel ? 
%\newcommand{\rlstep}[2]{\xleftarrow{#1}{#2}}
%\newcommand{\eqstep}[2]{\xleftrightarrow{#1}{#2}}
%\newcommand{\mapstep}[2]{\mathop{\xmapsto[\scriptstyle #2]{\scriptstyle #1}}}
\makeatletter
\newcommand{\xleftrightarrow}[2][]{\ext@arrow 3359\leftrightarrowfill@{#1}{#2}}
\newcommand{\xdashrightarrow}[2][]{\ext@arrow 0359\rightarrowfill@@{#1}{#2}}
\newcommand{\xdashleftarrow}[2][]{\ext@arrow 3095\leftarrowfill@@{#1}{#2}}
\newcommand{\xdashleftrightarrow}[2][]{\ext@arrow 3359\leftrightarrowfill@@{#1}{#2}}
\def\rightarrowfill@@{\arrowfill@@\relax\relbar\rightarrow}
\def\leftarrowfill@@{\arrowfill@@\leftarrow\relbar\relax}
\def\leftrightarrowfill@@{\arrowfill@@\leftarrow\relbar\rightarrow}
\def\arrowfill@@#1#2#3#4{%
  $\m@th\thickmuskip0mu\medmuskip\thickmuskip\thinmuskip\thickmuskip
   \relax#4#1
   \xleaders\hbox{$#4#2$}\hfill
   #3$%
}
\makeatother


%% Extensible arrows from pgf/tikz
\usetikzlibrary{arrows}
\usetikzlibrary{cd}
\makeatletter
\newbox\xrat@below
\newbox\xrat@above
\newcommand{\yrightarrowhook}[2][]{%
  \setbox\xrat@below=\hbox{\ensuremath{\scriptstyle #1}}%
  \setbox\xrat@above=\hbox{\ensuremath{\scriptstyle #2}}%
  \pgfmathsetlengthmacro{\xrat@len}{max(\wd\xrat@below,\wd\xrat@above)+.7em}%
  \mathrel{\tikz [right hook->,baseline=-.75ex]
                 \draw (0,0) -- node[below=-1.5pt] {\box\xrat@below}
                                node[above=-1.5pt] {\box\xrat@above}
                       (\xrat@len,0) ;}}
\newcommand{\yrightarrow}[2][]{%
  \setbox\xrat@below=\hbox{\ensuremath{\scriptstyle #1}}%
  \setbox\xrat@above=\hbox{\ensuremath{\scriptstyle #2}}%
  \pgfmathsetlengthmacro{\xrat@len}{max(\wd\xrat@below,\wd\xrat@above)+.7em}%
  \mathrel{\tikz [->,baseline=-.75ex]
                 \draw (0,0) -- node[below=-1.5pt] {\box\xrat@below}
                                node[above=-1.5pt] {\box\xrat@above}
                       (\xrat@len,0) ;}}
\makeatother


%% Arrows
\def\Reduction#1#2#3#4{%
\mathrel{\raise1.0ex\hbox{%
\vtop{\ialign{##\crcr%
\raise0.0ex\hbox{$\hfil\scriptstyle{\ #1\ }\hfil$}\crcr%
\noalign{\nointerlineskip}%
\rightarrowfill\crcr%
\noalign{\nointerlineskip}%
\raise0.0ex\hbox{$\hfil\scriptstyle{\ #2\ }\hfil$}\crcr}}}{}^{#3}_{#4}}}
%
\def\Leduction#1#2#3#4{%
\mathrel{\raise1.0ex\hbox{%
\vtop{\ialign{##\crcr%
\raise0.0ex\hbox{$\hfil\scriptstyle{\ #1\ }\hfil$}\crcr%
\noalign{\nointerlineskip}%
\leftarrowfill\crcr%
\noalign{\nointerlineskip}%
$\hfil\scriptstyle{\ #2\ }\hfil$\crcr}}}{}^{#3}_{#4}}}
%
%\def\hookxrightarrow[#1]#2{{\lhook\hspace{-0.20em}\xrightarrow[#1]{#2}}}
\def\hookReduction#1#2#3#4{%
%\lhook\joinrel\hspace{-0.35em}
\mathrel{\raise1.2ex\hbox{%
\vtop{\ialign{##\crcr%
\raise0.0ex\hbox{$\hfil\scriptstyle{\ #1\ }\hfil$}\crcr%
\noalign{\nointerlineskip}%
$\lhook\joinrel$\hspace{-0.35em}
\rightarrowfill\crcr%
\noalign{\nointerlineskip}%
$\hfil\scriptstyle{\ #2\ }\hfil$\crcr}}}{}^{#3}_{#4}}}
%
\def\hoookReduction#1#2#3#4{%
\lhook\joinrel\hspace{-0.50em}
\raise0.85ex\hbox{%
\vtop{\ialign{##\crcr%
\raise0.4ex\hbox{$\hfil\scriptstyle{\ #1\ }\hfil$}\crcr%
\noalign{\nointerlineskip}%
\rightarrowfill\crcr%
\noalign{\nointerlineskip}%
$\hfil\scriptstyle{\ #2\ }\hfil$\crcr}}}{}^{#3}_{#4}}

%\def\reach#1#2{\mathop{#1}[#2]}

%% rewrite steps
%% \frew#1#2#3#4#5#6#7#8
\def\frew#1#2#3#4#5#6#7#8{
\setbox0=\hbox{$#6 #7 #1 #8$}%
\setbox1=\hbox{$#6 #7 #2 #8$}%
\ifdim \wd0>\wd1 \rlap{\rlap{\hbox to \wd0{#5}}%
                            {\hbox to\wd0{\hfil\lower #3\box1\relax\hfil}}}{\raise #4\box0}%
\else \rlap{\rlap{\hbox to \wd1{#5}}{\hbox to\wd1{\hfil\raise #4\box0\relax\hfil}}}{\lower #3\box1}%
\fi
}
%% \fstep
\def\fstep#1#2#3#4#5{\mathchoice{\frew{#1}{#2}{1.10ex}{1.20ex}{#5}{\scriptstyle}{#3}{#4}}%
                                {\frew{#1}{#2}{0.82ex}{1.20ex}{#5}{\scriptstyle}{#3}{#4}}%
                                {\frew{#1}{#2}{0.51ex}{0.82ex}{#5}{\scriptscriptstyle}{#3}{#4}}%
                                {\frew{#1}{#2}{0.51ex}{0.69ex}{#5}{\scriptscriptstyle}{#3}{#4}}}
%% \lrstep, \rlstep, \eqstep
% #1 top label
% #2 bottom_right label
\newcommand{\lrstep}[2]{\mathrel{\fstep{#1}{#2}{\;\>}{\>\>\;}{\rightarrowfill}}}
\newcommand{\lrsteptc}[2]{\mathrel{\fstep{#1\ }{#2\ }{\;\>}{\>\>\;}{\rightarrowfill$^*$}}}
\newcommand{\rlstep}[2]{\mathrel{\fstep{#1}{#2}{\;\>\>}{\;\>}{\leftarrowfill}}}
\newcommand{\eqstep}[2]{\mathrel{\fstep{#1}{#2}{\;\>}{\>\;}{\rlap{\leftarrowfill}{\rightarrowfill}}}}

%% \fstepd   ad hoc.... to avoid hidden overline
\def\fstepd#1#2#3#4#5{\mathchoice{\frew{#1}{#2}{1.10ex}{1.20ex}{#5}{\scriptstyle}{#3}{#4}}%
                                {\frew{#1}{#2}{1.12ex}{1.20ex}{#5}{\scriptstyle}{#3}{#4}}%
                                {\frew{#1}{#2}{0.51ex}{0.82ex}{#5}{\scriptscriptstyle}{#3}{#4}}%
                                {\frew{#1}{#2}{0.51ex}{0.69ex}{#5}{\scriptscriptstyle}{#3}{#4}}}
\newcommand{\lrstepd}[2]{\mathrel{\fstepd{#1}{#2}{\;\>}{\>\>\;}{\rightarrowfill}}}


%% music symbols
% see http://tug.ctan.org/info/latex4musicians/latex4musicians.pdf
\usepackage{musicography}

%% for new macros
\usepackage{xspace}

%% Misc macros
\def\ie{\textit{i.e.}\xspace}
\def\eg{\textit{e.g.}\xspace}
\def\wrt{\textit{wrt}\xspace}
%\def\wlog{\textit{wlog}\xspace}
\def\etc{\textit{etc}\xspace}

\def\<#1>{\langle #1 \rangle}
\newcommand{\pair}[2]{\langle{#1}, {#2}\rangle}

%\newcommand{\config}[2]{\ensuremath \genfrac{[}{]}{0pt}{}{#1}{#2}} 
\newcommand{\config}[2]{\ensuremath{#1}[{#2}]} 
\newcommand{\configup}[2]{\ensuremath{#1}\left[\begin{array}{c} #2 \end{array}\right]}
\def\stacksep{\cdot}
\def\stackup{\\} 

\newcommand{\opluseq}{\ensuremath\mathrel{\oplus}=}

% \bigominus
\DeclareFontFamily{U}{mathx}{\hyphenchar\font45}
\DeclareFontShape{U}{mathx}{m}{n}{
<-6> mathx5 <6-7> mathx6 <7-8> matha7
<8-9> mathx8 <9-10> mathx9
<10-12> mathx10 <12-> mathx12
}{}
\DeclareSymbolFont{mathx}{U}{mathx}{m}{n}
\DeclareMathSymbol{\bigominus}{\mathop}{mathx}{"C1}

%\newcommand{\A}{\mathcal{A}}
%\newcommand{\B}{\mathcal{B}}
\newcommand{\D}{\mathbb{D}}
\newcommand{\E}{\mathbb{E}}
%\newcommand{\P}{\mathcal{P}}
\newcommand{\Q}{\mathcal{Q}}
\newcommand{\R}{\mathcal{R}}
\newcommand{\T}{\mathcal{T}}
\newcommand{\W}{\mathbb{W}}
\newcommand{\Semiring}{\mathbb{S}}
\newcommand{\zero}{\mathbb{0}}
\newcommand{\one}{\mathbb{1}}
\newcommand{\dom}{\ensuremath{\mathit{dom}}}

\def\SA{\textsf{sA}\xspace}
\def\WA{\textsf{wA}\xspace}
\def\SWT{\textsf{swT}\xspace}
\def\SWA{\textsf{swA}\xspace}
\def\SWTA{\textsf{swTA}\xspace}
\def\SWVPA{\textsf{sw-VPA}\xspace}
\def\weight{\mathsf{weight}}
\def\wei{\mathsf{w}}
\def\mei{\mathsf{m}}
\def\init{\mathsf{in}}
\def\final{\mathsf{out}}
\newcommand{\call}[1]{\ensuremath #1} %{\ensuremath \langle_{#1}}
\newcommand{\return}[1]{\ensuremath #1} %{\ensuremath {}_{#1}{\rangle}} % $\prescript{}{a}{)}$
\def\Omegai{{\Omega_\mathsf{i}}}
\def\Omegac{{\Omega_\mathsf{c}}}
\def\Omegar{{\Omega_\mathsf{r}}}
\def\Sigmai{{\Sigma_\mathsf{i}}}
\def\Sigmac{{\Sigma_\mathsf{c}}}
\def\Sigmar{{\Sigma_\mathsf{r}}}
\def\Deltai{{\Delta_\mathsf{i}}}
\def\Deltac{{\Delta_\mathsf{c}}}
\def\Deltar{{\Delta_\mathsf{r}}}
\def\Phii{{\Phi_\mathsf{i}}}
\def\Phic{{\Phi_\mathsf{c}}}
\def\Phir{{\Phi_\mathsf{r}}}
\def\Phix{{\Phi_\mathsf{x}}}
\def\Phici{{\Phi_\mathsf{ci}}}
\def\Phicc{{\Phi_\mathsf{cc}}}
\def\Phicr{{\Phi_\mathsf{cr}}}
\def\Phicx{{\Phi_\mathsf{cx}}}
\def\weii{{\wei_\mathsf{i}}}
\def\weic{{\wei_\mathsf{c}}}
\def\weir{{\wei_\mathsf{r}}}
\newcommand{\weie[1]}{\wei_{#1}^\mathsf{e}}
\def\weiei{\weie[\mathsf{i}]}
\def\weiec{\weie[\mathsf{c}]}
\def\weier{\weie[\mathsf{r}]}
\def\weiex{\weie[\mathsf{x}]}
\newcommand{\ioi}[1]{\mathsf{ioi}({#1})}
\newcommand{\rank}{\mathsf{rk}}
\newcommand{\lin}{\mathsf{lin}}


%\sloppy

% Parsing over infinite alphabet as optimal alignment computation 
% as edit-distance between string and language
% 
%\title{Symbolic Weighted Parsing and Automated Music Transcription}
%\title{Symbolic Weighted Language Models and Automated Music Transcription}
\title{Symbolic Weighted Language Models and\\ Parsing over Infinite Alphabets}
%\title{Weighted Visibly Pushdown Automata and Automated Music Transcription}
%\titlerunning{WVPA \& AMT}
%\titlerunning{Symbolic Weighted Parsing} % and Automated Music Transcription
%\authorrunning{Florent Jacquemard}

\date{\today}
 
\begin{document}
\thispagestyle{empty}
\maketitle

\begin{abstract}
% !TEX root = main.tex
%
% Symbolic Weighted (SW) extension of symbolic automata...
%
We propose a framework for weighted parsing over infinite alphabets.
%
It is based on language models called Symbolic Weighted Automata (\SWA)
at the joint between Symbolic Automata (\SA) and Weighted Automata (\WA),
as well as Transducers (\SWT) and Visibly Pushdown (\SWVPA) variants.
%
Like \SA, \SWA deal with large or infinite input alphabets,
and like \WA, they output a weight value in a semiring domain.
The transitions of \SWA are labeled by functions from an infinite alphabet into the weight domain.
This generalizes \SA, whose transitions are guarded by Boolean predicates
overs symbols in an infinite alphabet,
and also \WA, whose transitions are labeled by constant weight values,
and which deal only with finite alphabets.
%
We present some properties of \SWA, \SWT, and \SWVPA models
that we use to define and solve a variant of parsing
over infinite alphabets.
%
We illustrate the model with a motivating application to
automated music transcription.

\end{abstract}



%%%%%%%%%%%%%%%%%%%%%%%%%%%%%%%%%%%%%%%%%%%%%%%%%%%%%%%%%%%%%%%%%%%%%%%%%%%%%%%%%%%%%%%%%%%%%%%%%%%%
%% intro
%%%%%%%%%%%%%%%%%%%%%%%%%%%%%%%%%%%%%%%%%%%%%%%%%%%%%%%%%%%%%%%%%%%%%%%%%%%%%%%%%%%%%%%%%%%%%%%%%%%%


\section{Introduction} \label{sec:intro}
% !TEX root = main.tex
%
% Introduction
%
Various extensions of language models have been proposed for handling infinite alphabets.
%words carrying data values in an infinite domain (e.g. integers) e.g. data processing
Some automata 
%\reviews{1) and grammar with storage?}
with memory extensions
allow restricted storage and comparison of input symbols
%and correspond with logics,
(see~\cite{Segoufin06csl} for a survey).
\florent{register: skip refs and details, add Mikolaj recent}
They use pebbles for marking positions~\cite{NevenSchwentickVianu04FSMinfinite},
registers~\cite{KaminskiFrancez94},
or %computations in 2 steps
separate computation on subsequences
with the same attribute values~\cite{Bojanczyk11FO2}. % data words automata.
%
%for the and verification of infinite state systems
%(model checkers: alphabet = long bit-vectors)
%...For the representation of  in model checking, verification and
%
Moreover, automata at the core of model checkers
compute on input symbols represented by large bitvectors~\cite{Vardi07ciaa} %\cite{BaierKatoen08MC}
%(sets of assignments of Boolean variables) %propositional variables)
and, in practice,  %implementation
each transition accepts a set of such symbols (instead of an individual symbol),
represented by Boolean Formulas or Binary Decision Diagrams.
%
Following this idea, % and generalizing,
in symbolic automata (\SA)~\cite{Veanes12symbolic,dAntoniVeanes17CAV,dAntoni21CACM},
transitions are guarded by predicates over large or infinite domains.
With appropriate closure conditions on the sets of such predicates, % (alphabet theories),
all the good properties enjoyed by automata over finite alphabets are preserved.
The ability to compare input symbols of \SA is quite restricted compared to the 
former automata with memory. 
It can however be extended with the addition of a stack~\cite{dAntonyAlur14SVPDA}.

Other extensions of language models  %(automata and grammars)
assign one weight value to every input~\cite{Droste09handbook}.
%help in dealing with non-determinism
This is useful for the quantitative modelling of
\eg probabilistic or stochastic recursive programs, %( for pushdown models),
quantitative database queries, or semi-structured data, % (in infinite domains), 
as well as for the verification of quantitative properties about 
\florent{quantitative reasoning/verification on programs}
quality measures, resource-consumption, distance metrics, probabilistic guarantees, \etc
%
In the context of parsing, when grammars return weight values, 
it is possible to rank derivations (and \emph{abstract syntax trees})
in order to select a best one (or $n$ bests),
%\florent{= "weighted parsing"}
\eg in case of ambiguity~\cite{Goodman99SemiringParsing,Nederhof03weightedParsing,MorbitzVogler19weighted-parsing}.
%This is roughly the principle of \emph{weighted parsing} approaches
%\reviews{1) jump from automata to grammars is uneasy and not really necessary}
% for which there may exist several derivations yielding one input word.
%
In \emph{weighted language models}
like \eg probabilistic context-free grammars % (CFG),
and weighted automata (\WA)~\cite{Droste09handbook},
a weight is associated to each transition rule, % production rule
and the weights of the rules involved in a computation are combined with an
associative product operator~$\otimes$. % to yield the weight of a whole computation.
%\reviews{1) AST for grammar, run for WA}
A second operator~$\oplus$
is moreover used to resolve the ambiguity raised by the existence
of several (in general exponentially many) computations on a given input.
Typically, $\oplus$ selects the best of two weight values.
%Intuitively,~$\oplus$ selects, or ranking, the syntax trees.
The weight domain, equipped with these two operators is typically 
a \emph{semiring} %$\Semiring$
where $\oplus$ can be extended to infinite sums,
%\cite{Eilenberg74automata}
such as the Viterbi semiring and the tropical min-plus algebra% % see Figure~\ref{fig:semirings}.
%of domain $\mathbb{R}^+ \cup \{ +\infty\}$,
%where $\oplus$ is min and $\otimes$ is plus .
%by ranking
%making the weight domain a semiring.
%Some efficient specialized parsing algorithms~\cite{Huang05kbest} have been proposed in this context
%% models represented as hypergraphs \cite{Huang05kbest}
%in order to compute the $n$ best syntax trees of a given input word without having to enumerate them all.
%Generally based on dynamic programming, these algorithms rely on
%additional algebraic properties of~$\Semiring$.
%-- see \eg~\cite{Huang05kbest} for some NLP applications.
%The extraction of $n$ best list is useful
%the problem: quantitative parsing or symbolic parsing
%of parsing of words over infinite input alphabet.

In this paper, we present some results for 
\emph{Symbolic Weighted} finite states language models %(\swM) 
generalizing the Boolean guards of~\SA %in transitions
to functions into an arbitrary semiring, 
and also the~\WA, by handling infinite alphabets (Figure~\ref{fig:hierarchy}).
%
The models considered are also 
%\SWA and \SWVPA are both 
particular cases of the very general class of 
Weighted Symbolic Automata with Data Storage~\cite{Herrmann16dlt,Herrmann20phd}, 
using appropriate storage types.
%with special data storage types...
%\florent{Both WA and SA generalized by \cite{Herrmann16dlt,Herrmann20phd}...}
%\reviews{cite \cite{Herrmann16dlt,Herrmann20phd}}

\begin{figure}
\centering
\begin{tikzpicture}
\node (SWADS) at (3.0,5.0) {%
  \(
  \begin{array}{c}
  \mathsf{SWADS}\,\mbox{\cite{Herrmann16dlt,Herrmann20phd}}
  \end{array}
  \)
};
%
\node (SWA) at (0,3.4) {%
  \(
  \begin{array}{c}
  \mathsf{SWA}: \Sigma_{\mathsf{inf}}^* \to \mathbb{S}\\
  q \xrightarrow{\phi} q', \phi : \Sigma_\mathsf{inf} \to \mathbb{S}
  \end{array}
  \)
};
%
\node (WA) at (-1.8,1.7) {%
  \(
  \begin{array}{c}
  \mathsf{WA}\,\mbox{\cite{Droste09handbook}}: \Sigma_{\mathsf{fin}}^* \to \mathbb{S}\\
  q \xrightarrow{a, w} q', a \in \Sigma_\mathsf{fin}, w \in \mathbb{S}
  \end{array}
  \)
};
%
\node (SA) at (1.8,1.7) {%8
  \(
  \begin{array}{c}
  \mathsf{SA}\,\mbox{\cite{Veanes12symbolic,dAntoni21CACM}}: \Sigma_{\mathsf{inf}}^* \to \mathbb{B}\\
  q \xrightarrow{\phi} q', \phi : \Sigma_\mathsf{inf} \to \mathbb{B}
  \end{array}
  \)
};
%
\node (NFA) at (0,0) {%
  \(
  \begin{array}{c}
  \mathsf{FA} : \Sigma_{\mathsf{fin}}^* \to \mathbb{B}\\
  q \xrightarrow{a} q', a \in \Sigma_\mathsf{fin}
  \end{array}
  \)
};
%
\node (VPA) at (5.6,0) {%
  \(
  \mathsf{VPA}\,\mbox{\cite{AlurMadhusudan09nested}}
  \)
};
%
\node (WVPA) at (4.6,1.7) {%
  \(
  \mathsf{WVPA}\,\mbox{\cite{Mathissen08weighted,Caralp12VPAmult}}
  \)
};
%
\node (SVPA) at (6.6,1.7) {%
  \(
  \mathsf{SVPA}\,\mbox{\cite{dAntonyAlur14SVPDA}}
  \)
};
%
\node (SWVPA) at (5.6,3.4) {%
  \(
  \mathsf{SWVPA}
  \)
};
%
\draw[->] (NFA)--(WA);
\draw[->] (NFA)--(SA);
\draw[->] (WA)--(SWA);
\draw[->] (SA)--(SWA);
\draw[->] (SWA)--(SWADS);
\draw[->] (VPA)--(WVPA);
\draw[->] (VPA)--(SVPA);
\draw[->] (WVPA)--(SWVPA);
\draw[->] (SVPA)--(SWVPA);
\draw[->] (SWVPA)--(SWADS);
%\begin{array}{c} \mathsf{NFA} : \Sigma^* \to \mathbb{B} \end{array}
\end{tikzpicture}
\caption{Classes of Symbolic/Weighted Automata.
Here, $\Sigma_\mathsf{fin}$ and $\Sigma_\mathsf{inf}$ denote finite/countable alphabets,
$\mathbb{B}$  the Boolean algebra,
$\mathbb{S}$ a commutative semiring.
$q \xrightarrow{\dots} q'$ is a transition between states $q$ and $q'$.}
\label{fig:hierarchy}
\end{figure}
%
%This approach is close to the case of
%Symbolic Automata (SA)~\cite{dAntoniVeanes17CAV,dAntoni21CACM},
%except that the domain for weight values is not restricted to be Boolean,
%like for the guards in the rules of SA,
%but can be an arbitrary commutative semiring (assuming some restrictions).

\noindent
In short, a transition rule $q \xrightarrow{\phi} q'$ of a SW automaton, 
from state $q$ to $q'$,
%\florent{c'est un résumé de la Fig. \ref{fig:hierarchy}}
%input symbols are variables, %(parameters)
is labeled by a function~$\phi$ associating to every input symbol~$a$, a weight value~$\phi(a)$
in a semiring~$\Semiring$.
%
The models studied are: 
symbolic-weighted automata (\SWA),
defining series over infinite alphabets, 
transducers (\SWT), 
defining distances between finite words over infinite alphabets, % following~\cite{Mohri03ijfcs}.
and pushdown automata with a visibility restriction~\cite{AlurMadhusudan09nested} (\SWVPA),
operating sequentially on %\emph{nested words}~\cite{AlurMadhusudan09nested},
words structured with markup symbols (parentheses) and describing linearizations of trees.
%A \SWVPA $A$ associates a weight value $A(t)$ % \in \Semiring$
%to a given nested word $t$, which is itsef the linearization of a weighted AST. %representing a parse tree.
%
The main contributions of this paper are:
%\reviews{1) not $i$! see response.txt}
%($i$)~the introduction of automata, \SWA, transducers, \SWT (Section~\ref{section:SWA}),
%and visibly pushdown automata \SWVPA (Section~\ref{sec:SWVPA}),
%generalizing the corresponding classes of symbolic and weighted models,
%for the weighted computation on (nested) words over infinite alphabets;
($i$)~the construction à la Bar-Hillel Perles Shamir of a \SWA
     computing a \SWT-defined distance between a \SWA input language and a word (Section~\ref{section:SWA}), 
($ii$)~closure results for \SWVPA, 
($iii$)~a polynomial best-search algorithm for \SWVPA (Section~\ref{sec:SWVPA}). %(Section~\ref{sec:best})
% a framework for parsing over infinite alphabets,
Moreover, we present in Section~\ref{sec:parsing} an 
($iv$) application to the problem of weighted parsing over infinite input alphabets, 
called \emph{SW-parsing}. 
%
The goal of this problem is, given an input word~$s$, 
to find~$t$ minimizing the distance, in the sense of~\cite{Mohri03ijfcs}, 
$T(s, t) \otimes A(t)$, where $T$ is a \SWT and $A$ a \SWVPA.
The notion of transducer-based distances allows to consider 
different infinite alphabets for the input $s$ and output $t$.
Moreover, the use of \SWVPA permits to consider output $t$ in the form of a (nested) word, instead of a tree.
%technical convenience
%the SW models are integrated in a uniform framework.
\emph{SW-parsing} is solved with the Bar-Hillel construction~\cite{NederhofSatta03ParsingIntersection} 
%\cite{GruneJacobs08parsing}
of a \SWVPA $B$ such that, for all~$t$, $B(t) = T(s, t) \otimes A(t)$
and the application of a best-search procedure to this automaton $B$.
% reduction to the shortest distance in graphs~\cite{Mohri02semiring,Huang05kbest}.

%Like weighted-parsing methods~\cite{Goodman99SemiringParsing,Nederhof03weightedParsing,MorbitzVogler19weighted-parsing},
%\reviews{1) WARNING: \cite{MorbitzVogler19weighted-parsing} incomparable to our def. weighted parsing}

%Parsing is the problem of structuring a linear representation
%(a finite word) according to a language model. % (a formal grammar). % natural language, programming language,

Context-free parsing approaches~\cite{GruneJacobs08parsing}
generally assume a finite and reasonably small input alphabet. %models and algorithms
%Such a restriction makes perfect sense in the context of
%NLP tasks such as constituency parsing, 
%compilers or interpreters for programming languages.
Considering large or infinite alphabets can however be of
practical interest when dealing with large characters encodings such as UTF-16~\cite{dAntoni21CACM}.
\florent{applis à la Fossacs chez Veanes et al?}
% processing strings in \eg for vulnerability detection in Web-applications~\cite{dAntoni21CACM},
%We believe that these results could also be useful 
It is also true in the context of automata-based quantitative verification,  %of systems processing %analysis of 
\eg when dealing with data streams, 
serialization of structured documents~\cite{Segoufin06csl,NevenSchwentickVianu04FSMinfinite},
or timed execution traces~\cite{Bouyer03algebraic}.

% algebraic definition of a class of data languages
% (notion of monoid recognizability, based on registers, comparable to Bojancszik et al. data words)

The latter case is related to a motivation  of  the present work: \florent{idem}
automated music transcription. Representations capturing music performances
are essentially linear~\cite{Selfridge-Field97beyondMIDI};
%audio files or the widely used MIDI format %\cite{MIDIfile}.
they ignore the hierarchical structures that frame the
conception of music, at least in the Western area. 
These structures, on the other hand, are present, either explicitly  or implicitly,
in Common Western Music Notation~\cite{Gould11Notation}:
%in music notation~\cite{Gould11Notation}:
Music scores are partitioned in measures,
measures in beats, and beats can be further recursively divided.
It follows that written music events do not occur at arbitrary timestamps,
but respect a discrete partitioning of the timeline incurred by
these recursive divisions.
The \emph{transcription problem} takes
as input a linear performance (audio or MIDI) and aims at re-constructing
structured notation, 
by mapping input events to this hierarchical rhythmic space.
It can therefore be stated as a parsing problem
over an infinite alphabet of timed events~\cite{foscarin:hal-01988990}.

%In expressiveness, they are equivalent to weighted CFG.
%and can be used in a general approach for parsing over infinite input alphabets.
%
%Let $A$ be a \SWVPA, associating $A(w) \in \Semiring$
%to a given a nested word $w$ (representing a parse tree),
%and let a \SWT compute a distance $d$, in $\Semiring$,
%between 2 strings over respectively an infinite input alphabet and the
%same (infinite) alphabet of $A$.
%Then, the problem of Symbolic Weighted Parsing is,
%given an input string $s$, to find a nested word $w$ minimizing
%(according to the ranking defined by $\oplus$)
%the distance $d(s, w) \otimes A(w)$ between $s$ and $A$,
%as defined in~\cite{Mohri03ijfcs}.

% First one general edit-distance is defined by a weighted word
% transducer~\cite{Mohri}
% %Symbolic automata transducers are extended models~\cite{VeanesdAntoniJACM}
% %dealing with infinite set of input symbols...
% value in a semiring...


\florent{expressiveness: VPA have restricted equality test.
        comparable to pebble automata? $\to$ conclusion}
        
        
\begin{example}%[Running example]
\label{ex:running}
We illustrate our approach with a very simplified %toy
running example of \emph{music transcription}:
a given input \emph{timeline} of musical events
from an infinite alphabet $\Sigma$,
is parsed into a structured music score.
Input events of $\Sigma$
are pairs $\mu \tstamp{\tau}$, where $\mu$ is a
MIDI key number~\cite{Selfridge-Field97beyondMIDI}, %\cite{MIDIfile}
and $\tau \in \mathbb{Q}$ is a timestamp in seconds.
Such inputs typically correspond
to the recording of a live performance, 
\eg~$I = 69\tstamp{0.07},
	     71\tstamp{0.72},
    	 73\tstamp{0.91},
	     74\tstamp{1.05},
	     76\tstamp{1.36},
	     77\tstamp{1.71}$. % over interval $[0,2[$,

The output of parsing is a sequence of
timed symbols $\nu\tstamp{\tau'}$ in an alphabet $\Delta$,
%events (or \emph{notes})
%in Common Western Music Notation (CWMN)~\cite{Gould11Notation}
where $\nu$  represents
an \emph{event} (or \emph{note}),
specified by its \emph{pitch name}
(\eg, $\mbox{\footnotesize A4}$, $\mbox{\footnotesize G5}$, \etc.),
an event \emph{continuation} (symbol `$-$`, see Example~\ref{ex:SWT}),
or a \emph{markup} (opening or closing parenthesis). %(\emph{parenthesis}),
The temporal information $\tau'$ 
is either a time interval, for the opening parentheses
(representing the duration between the parenthesis 
 and the matching closing one), 
or a timestamp, for the other symbols.
The time points in $\tau'$ belong to a rhythmic ``grid'' obtained from recursive divisions:
whole notes ($\musWhole$) split in halves ($\musHalf$), halves
in quarters ($\musQuarter$), eights ($\musEighth$), \etc.
%
For instance, the output score
\includegraphics[scale=0.35,trim=0 5mm 0 0]{pictures/ex1.pdf},
  %\includegraphics[scale=0.20]{pictures/score5.png},
corresponds to a hierarchical structure
that can be linearized as the sequence
$O =$
$\ccall{\mathsf{m}}\tstamp{[0,1]}$,
$\ccall{2}\tstamp{[0,1]}$,
$\mbox{\footnotesize A4}\tstamp{0}$,
$\ccall{2}\tstamp{[\frac{1}{2},1]}$,
$-\tstamp{\frac{1}{2}}$,
$\ccall{2}\tstamp{[\frac{3}{4},1]}$,
$\mbox{\footnotesize B4}\tstamp{\frac{3}{4}}$,
$\mbox{\footnotesize C$\sharp$5}\tstamp{\frac{7}{8}}$,
$\creturn{2}\tstamp{1}$,
$\creturn{2}\tstamp{1}$,
$\creturn{2}\tstamp{1}$,
$\creturn{\mathsf{m}}\tstamp{1}$,
$\ccall{\mathsf{m}}\tstamp{[1,2]}$,
$\ccall{3}\tstamp{[1,2]}$,
$\mbox{\footnotesize D5}\tstamp{1}$,
$\mbox{\footnotesize E5}\tstamp{\frac{4}{3}}$,
$\mbox{\footnotesize F5}\tstamp{\frac{5}{3}}$,
$\creturn{3}\tstamp{2}$,
$\creturn{\mathsf{m}}\tstamp{2}$.
%\end{enumerate}
The opening markups $\ccall{\mathsf{m}}$ %and $\creturn{\mathsf{m}}$
delimit \emph{measures},
which are time intervals of duration~1 in this example,
while the subsequences of $O$ between markups~$\ccall{d}$ and~$\creturn{d}$,
for some natural number~$d$,
represent a division of the time interval attached to $\ccall{d}$,
of duration $\ell$,
into $d$ sub-intervals of equal duration $\frac{\ell}{d}$.
%
We will show that $O$ is a solution for the
parsing of $I$. Note that several other parsings are possible
like \eg \includegraphics[scale=0.35,trim=0 5mm 0 0]{pictures/ex2.pdf}.
SW-parsing associates a weight value
to each solution, and our approach
aims at selecting the best one with respect to this weight.
\endex
\end{example}



%%%%%%%%%%%%%%%%%%%%%%%%%%%%%%%%%%%%%%%%%%%%%%%%%%%%%%%%%%%%%%%%%%%%%%%%%%%%%%%%%%%%%%%%%%%%%%%%%%%%
%% prelin
%%%%%%%%%%%%%%%%%%%%%%%%%%%%%%%%%%%%%%%%%%%%%%%%%%%%%%%%%%%%%%%%%%%%%%%%%%%%%%%%%%%%%%%%%%%%%%%%%%%%

\section{Preliminary Notions}
\label{section:prelim}\label{sec:prelim}

%notations: for set $S$ : sets of sequences $S^*$ and $S^+$...
%interval $[i..j]$ of natural numbers



\subsection{Semirings} 
\label{section:semiring}\label{sec:semiring}
% !TEX root = main.tex
%
% Semiring basics
%%
%We shall consider semirings for the weight values of our language models.
%\florent{The results are established for a general class of semirings. They can be instantiated for concrete cases}
\noindent
\textbf{Semirings}. A \emph{semiring} $\< \Semiring, \oplus, \zero, \otimes, \one>$
is a structure with a domain~$\Semiring$,
equipped with two associative
binary operators~$\oplus$ and $\otimes$,
with respective neutral elements $\zero$ and $\one$, such that:
%$\< \mathbb{S}, \oplus, \zero>$ is a commutative monoid
%$\< \mathbb{S}, \otimes, \one>$ is a monoid
\begin{itemize}
\item $\oplus$ is commutative:
 $\< \Semiring, \oplus, \zero>$ is a commutative monoid
   and $\< \Semiring, \otimes, \one>$ a monoid,
\item $\otimes$ distributes over~$\oplus$: \\ $\forall x, y, z \in \mathbb{S}$,
$x \otimes (y \oplus z) = (x \otimes y) \oplus (x \otimes z)$,
and $(x \oplus y) \otimes z = (x \otimes z) \oplus (y \otimes z)$,
\item $\zero$ is absorbing for~$\otimes$:
$\forall x\in \mathbb{S}$, $\zero \otimes x = x \otimes \zero = \zero$.
\end{itemize}
%Components of a semiring~$\Semiring$ may be subscripted by~$\Semiring$ when needed.
%We simply write $x \in \Semiring$ to mean $x \in \mathbb{S}$.
%
In the models presented in this paper,
$\oplus$ selects an optimal value from two given values,
in order to handle non-determinism,
and $\otimes$ combines two values into a single one.
%and let $\< \Semiring, \oplus, \zero, \otimes, \one>$ be a {semiring},
A semiring~$\Semiring$ is \emph{commutative} if $\otimes$ is commutative.
It is \emph{idempotent} if for every $x \in \Semiring$, $x \oplus x = x$.
%
Every idempotent semiring~$\Semiring$ induces
a partial ordering~$\leq_\oplus$
which is called the \emph{natural ordering} of~$\Semiring$~\cite{Mohri02semiring},
and is defined by:
%implicitly defined by the semiring $\Semiring$
for every $x, y\in \Semiring$,
$x \leq_\oplus y \;\mbox{iff}\; x \oplus y = x$.
%(see~\cite{Mohri02semiring} for the proof that it is an ordering).
%
The natural ordering is sometimes defined in the opposite direction~\cite{DrosteKuich09semirings};
We follow here the direction  %follows \cite{Mohri02semiring}, and
that coincides with the usual ordering on the Tropical semiring \emph{min-plus}
(Figure~\ref{fig:semirings}).
%
\noindent
An idempotent semiring $\Semiring$~is called \emph{total} if
~$\leq_\oplus$ is total,
\ie when for every $x, y \in \Semiring$, either $x \oplus y = x$ or $x \oplus y = y$.
%\florent{is total necessary?}

\begin{lemma}[Monotony, \cite{Mohri02semiring}] \label{lem:monotonic}
Let $\< \Semiring, \oplus, \zero, \otimes, \one>$ be an idempotent semiring.
For every $x, y, z  \in \Semiring$,
if $x \leq_\oplus y$ then
$x \oplus z \leq_\oplus y \oplus z$,
$x \otimes z \leq_\oplus y \otimes z$
and $z \otimes x \leq_\oplus z \otimes y$.
\end{lemma}
%When  holds,

We thus say the
$\Semiring$ is \emph{monotonic} \wrt $\leq_\oplus$.
%A semiring $\Semiring$
%is \emph{monotonic} \wrt a partial ordering~$\leq$
%iff for all $x, y, z  \in \Semiring$,  $x \leq y$ implies
%$x \oplus z \leq y \oplus z$,
%$x \otimes z \leq y \otimes z$,
%and $z \otimes x \leq z \otimes y$.
%
Another important semiring property in the context of optimization
is {superiority}~\cite{Huang08advanceddynamic},
which corresponds to the
\emph{non-negative weights} condition in shortest-path algorithms~\cite{Dijkstra59anote}.
Intuitively, it means that combining elements with~$\otimes$ always increases their weight.
Formally, it is defined as the property ($i$) below. %of the following lemma.

\begin{lemma}[Superiority, Boundedness]
\label{lem:superior}\label{lem:bounded}
Let $\< \Semiring, \oplus, \zero, \otimes, \one>$ be an idempotent semiring.
The two following statements are equivalent:
\begin{itemize}
\item [$i.$] for every $x, y \in \Semiring$,
$x \leq_\oplus x \otimes y$ and
$y \leq_\oplus x \otimes y$
\item[$ii.$] for every $x \in \Semiring$,  $\one \oplus x = \one$.
\end{itemize}
\end{lemma}
%
\begin{proof} %(Lemma~\ref{lem:bounded})
$(ii) \Rightarrow (i)$ :
$x \oplus (x \otimes y) = x \otimes (\one \oplus y) = x$,
by distributivity of~$\otimes$ over~$\oplus$.
Hence $x \leq_\oplus x \otimes y$.
Similarly, $y \oplus (x \otimes y) = (\one \oplus x) \otimes y = y$,
hence $y \leq_\oplus x \otimes y$.
%
$(i) \Rightarrow (ii)$ :
%$(i)$ implies that $\one \leq_\oplus x$ for all $x \in \Semiring$,
by the second inequality of ($i$), with $y = \one$,
$\one \leq_\oplus x \otimes \one = x$, \ie,
by definition of $\leq_\oplus$, $\one \oplus x = \one$.
%\qed
\end{proof}

In~\cite{Huang08advanceddynamic}, when  property~$(i)$ holds,
$\Semiring$ is called \emph{superior} \wrt~$\leq_\oplus$.
It implies (proof of Lemma~\ref{lem:bounded})
that
$\one \leq_\oplus x$ for every $x \in \Semiring$.
Similarly, by the first inequality of ($i$) with $y = \zero$,
$x \leq_\oplus x \otimes \zero = \zero$.
%
Hence, in a superior semiring,
for every $x \in \Semiring$, $\one \leq_\oplus x \leq_\oplus \zero$.
%
From an optimization point of view,
it means that $\one$ is the best value, and $\zero$ the worst.
%** superior implies $\Semiring$ bounded~\cite{Mohri02semiring} see **
%
In~\cite{Mohri02semiring},
$\Semiring$ with the property ($ii$) of Lemma~\ref{lem:bounded}
is called \emph{bounded} -- we shall use this term in the rest of the paper.
% \emph{negative boundedness}
It implies that, when looking for a best path in a graph whose edges
are weighted by values of $\Semiring$, the loops can be safely avoided,
because, for every $x \in \Semiring$ and $n \geq 1$,
 $x \oplus x^n = x \otimes (\one \oplus x^{n-1}) = x$.


\begin{lemma}[\cite{Mohri02semiring}, Lemma~3]\label{lem:idempotent}
Every bounded semiring is idempotent.
\end{lemma}
\begin{proof}
By boundedness, $\one \oplus \one = \one$,
and idempotency follows by multiplying
both sides by $x$ and distributing.
%\qed
\end{proof}

%\medskip
\noindent
We need to extend ~$\oplus to infinitely many operands$.
A semiring~$\Semiring$ is called \emph{complete}~\cite{Droste09handbook}
%(\cite{Droste09handbook} chapter 1) %\cite{Kuich97semirings}
if it has an %infinite sum
operation $\bigoplus_{i \in I} x_i$
for every family
$(x_i)_{i \in I}$ %$\{ x_i \mid i \in I \}$
of elements of $\dom(\Semiring)$ over an index set $I \subseteq \mathbb{N}$, such that:
\begin{description}
\item[$i.$]
\emph{infinite sums extend finite sums:}\\
$\displaystyle\bigoplus_{i \in \emptyset} x_i = \zero$,\quad
      $\forall j\in \mathbb{N}, \displaystyle\bigoplus_{i \in \{ j \}} x_i = x_j$,
      $\forall j, k\in \mathbb{N}, j\neq k,
      \displaystyle\bigoplus_{i \in \{ j, k \}} x_i = x_j \oplus x_k$,
%
\item[$ii.$]
\emph{associativity and commutativity:}\\
for evert $I \subseteq \mathbb{N}$
and all partitions $(I_{j})_{j \in J}$ of $I$, %\subseteq \mathbb{N}$,
\(
\displaystyle
\bigoplus_{j \in J}\bigoplus_{i \in I_j} x_i =
\bigoplus_{i \in I} x_i
\),
%
\item[$iii.$]
\emph{distributivity of products over infinite sums:}\\
for every $I \subseteq \mathbb{N}$,
\(
\displaystyle
\bigoplus_{i \in I} (x \otimes y_i) = x \otimes \bigoplus_{i\in I} y_i\), and
\(
\displaystyle
\bigoplus_{i \in I} (x_i \otimes y) = (\bigoplus_{i \in I} x_i ) \otimes y\).
\end{description}



%\begin{example}
%Figure~\ref{fig:semirings} presents examples of semirings interesting in practice
%and enjoying the above properties.
%\end{example}


\begin{figure}[t]
\begin{center}
\begin{tabular}{|c|c|C{2em}|C{2em}|C{2em}|C{2em}|}
\hline
        & domain & $\oplus$ & $\otimes$ & $\zero$  & $\one$\\ %& nat. ordering\\
\hline\hline
Boolean & $\{\bot, \top\}$ & $\vee$ & $\wedge$ & $\bot$ & $\top$\\ %& $\top \leq_\oplus \bot$  \\
\hline
Counting & $\mathbb{N}$ & $+$ & $\times$ & 0 & 1 \\
\hline
Viterbi & $[0, 1] \subset \mathbb{R}$ & $\mathit{max}$ & $\times$ & 0 & 1\\ % & $x \leq_\oplus y \iff x \ge y$  \\
\hline
Tropical min-plus & $\mathbb{R}_+ \cup \{ \infty\}$ & $\mathit{min}$ & $+$ & $\infty$ & $0$\\ % & $x \leq_\oplus y \iff x \leq y$   \\
\hline
%MaxPlus & $\mathbb{Q} \cup \{ -\infty\}$ & $\mathsf{max}$ & $+$ & $-\infty$ & $0$ \\
%\hline
%Word lang. & $2^{\Sigma^*}$ & $\cup$ & $\cdot$ & $\emptyset$ & $\{ \varepsilon \}$ \\
%\hline
\end{tabular}
\end{center}
\vskip-1em
\caption{Some commutative, bounded, total and complete semirings.}
\label{fig:semirings}
\end{figure}
%\florent{results of this paper: for semirings commutative, bounded, total and complete}
%%% semiring properties used in paper
% - commutative
% - bounded (implies idempotent and superior)
% - total
% - complete



\subsection{Label Theory}
\label{section:symbols}
% !TEX root = main.tex
%
% Label Theory
%
We shall now define the functions labeling the transitions of SW automata and transducers,
generalizing the Boolean algebras of~\cite{dAntoniVeanes17CAV} 
from Boolean to other semiring domains.
%
We consider \emph{alphabets}, which are countable sets of symbols 
denoted $\Sigma$, $\Delta$,...
%Let $\< \Semiring, \oplus, \zero, \otimes, \one>$ be a commutative, complete semiring.
%
\noindent 
\philippe{OK, donc c'est là que les fonctions d'étiquettes prennent en argument l'input de la règle. Je ne sais
pas dans quelle mesure il faut donner un peu d'explications pour faciliter la compréhension du formalisme.}
Given a semiring $\< \Semiring, \oplus, \zero, \otimes, \one>$, 
a \emph{label theory} over $\Semiring$
is a set $\bar\Phi$ of recursively enumerable sets denoted
%$\Phi_\epsilon \subseteq \Semiring$, % containing constant functions valued in $\Semiring$, 
$\Phi_\Sigma$, %and $\Phi_\Delta$, 
containing unary functions of type $\Sigma \to \Semiring$, %resp. $\Delta \to \Semiring$, 
or $\Phi_{\Sigma, \Delta}$, containing binary functions $\Sigma \times \Delta \to \Semiring$, 
and such that:

\noindent -- 
for all $\Phi_{\Sigma, \Delta} \in \bar\Phi$, we have
$\Phi_{\Sigma} \in \bar\Phi$ and $\Phi_{\Delta} \in \bar\Phi$

\noindent -- 
every $\Phi_{\Sigma}\in \bar\Phi$ contains all the constant functions from $\Sigma$ into $\Semiring$, 
 
\noindent -- 
for all $\alpha \in \Semiring$ and $\phi \in \Phi_\Sigma$,
      $\alpha \otimes \phi : x \mapsto \alpha \otimes \phi(x)$, 
      and $\phi \otimes \alpha : x \mapsto \phi(x) \otimes \alpha$\\
\phantom{--} belong to $\Phi_\Sigma$, and similarly for $\oplus$ 
      and for $\Phi_{\Sigma, \Delta}$

\noindent -- 
for all $\phi, \phi' \in \Phi_\Sigma$,
$\phi \otimes \phi': x \mapsto \phi(x) \otimes \phi'(x)$ belongs to $\Phi_\Sigma$

\noindent -- 
for all $\eta, \eta' \in \Phi_{\Sigma, \Delta}$
$\eta \otimes \eta': x, y \mapsto \eta(x, y) \otimes \eta'(x, y)$ belongs to $\Phi_{\Sigma, \Delta}$

\noindent -- 
for all $\phi \in \Phi_\Sigma$ and $\eta \in \Phi_{\Sigma, \Delta}$,
$\phi \otimes_1 \eta: x, y \mapsto \phi(x) \otimes \eta(x, y)$ and\\
\phantom{--} $\eta \otimes_1 \phi: x, y \mapsto \eta(x, y) \otimes \phi(x)$
belong to $\Phi_{\Sigma, \Delta}$

\noindent -- 
for all $\psi \in \Phi_\Delta$ and $\eta \in \Phi_{\Sigma, \Delta}$,
$\psi \otimes_2 \eta: x, y \mapsto \psi(y) \otimes \eta(x, y)$ and\\
\phantom{--} $\eta \otimes_2 \psi: x, y \mapsto \eta(x, y) \otimes \psi(y)$ 
belong to $\Phi_{\Sigma, \Delta}$

\noindent -- 
similar closures hold for $\oplus$.

%\noindent -- 
%the partial applications $\eta \in \Phi_{\Sigma, \Delta}$
%and $\eta_a: y \mapsto \eta(a, y)$ for $a \in \Sigma$ %and $y \in \Delta$
%and\\ 
%\phantom{--} $\eta_b: x \mapsto \eta(x, b)$ for $b \in \Delta$ %and $x \in \Sigma$, 
%belong respectively to~$\Phi_\Delta$ and~$\Phi_\Sigma$.
\florent{partial application is needed?}

%Moreover, these sets are required to be closed under the 
%operators~$\oplus$ and~$\otimes$ of~$\Semiring$:
%for all $\phi, \phi' \in \Phi_\Sigma$,
%$\psi, \psi' \in \Phi_\Delta$, 
%and $\eta, \eta' \in \Phi_{\Sigma, \Delta}$, %the function
%%
%\begin{center}
%\begin{tabular}{cclll}
%$\phi \otimes \phi'$ & : & $x \mapsto \phi(x) \otimes \phi'(x)$ & belongs to $\Phi_\Sigma$,\\
%$\psi \otimes \psi'$ & : & $y \mapsto \psi(y) \otimes \psi'(y)$ & belongs to $\Phi_\Delta$,\\
%$\phi \otimes \eta$\;  & : & $x, y \mapsto \phi(x) \otimes \eta(x, y)$ & belongs to $\Phi_{\Sigma, \Delta}$,\\
%$\eta \otimes \psi$  & : & $x, y \mapsto \eta(x, y) \otimes \psi(y)$ & belongs to $\Phi_{\Sigma, \Delta}$,\\
%$\eta \otimes \eta'$ & : & $x, y \mapsto \eta(x, y) \otimes \eta'(x, y)$ & belongs to $\Phi_{\Sigma, \Delta}$, &
%\multicolumn{1}{r}{and similarly for $\oplus$.}\\ %the same also holds for the binary sum operator $\oplus$.
%\end{tabular}
%\end{center}
%
%Finally, it is also required 
%% that the codomain of every function of $\Phi_\Sigma$ and $\Phi_\Delta$ 
%% is a subset of $\Phi_\epsilon$, and
%that the partial applications of a function $\eta \in \Phi_{\Sigma, \Delta}$, 
%resp.  $\eta_a: y \mapsto f(a, y)$ for $a \in \Sigma$ and $y \in \Delta$
%and  $\eta_b: x \mapsto f(x, b)$ for $b \in \Delta$ and $x \in \Sigma$, 
%belong resp. to~$\Phi_\Sigma$ and~$\Phi_\Delta$.
%
\noindent 
Intuitively, the operators $\bigoplus_\Sigma$ 
return global minimum, \wrt $\leq_\oplus$, of functions of~$\Phi_\Sigma$. 
%
When the semiring $\Semiring$ is complete, we 
consider the following operators on the functions of~$\bar\Phi$. % a label theory.
%(we use overloading to simplify notations):
\[
\begin{array}{ll}
\bigoplus_\Sigma : \Phi_\Sigma \to \Semiring,\ 
  \phi \mapsto \displaystyle\bigoplus_{a \in \Sigma} \phi(a)\\
\bigoplus^1_\Sigma : 
  \Phi_{\Sigma,\Delta} \to \Phi_\Delta,\ 
  \eta \mapsto \bigl( y \mapsto \displaystyle\bigoplus_{a \in \Sigma} \eta(a, y) \bigr) &
\bigoplus^2_\Delta : 
  \Phi_{\Sigma,\Delta} \to \Phi_\Sigma,\ 
  \eta \mapsto \bigl( x \mapsto \displaystyle\bigoplus_{b \in \Delta} \eta(x, b) \bigr)\\
\end{array}
\]
%
\medskip\noindent
In what follows, we might omit the sub- and superscripts in 
$\otimes_1$, $\bigoplus^1_\Sigma$..., 
%$\otimes_2$, $\oplus_1$, $\oplus_2$
when there is no ambiguity.
We shall keep them only for the special case $\Sigma = \Delta$,
\ie $\eta \in \Phi_{\Sigma, \Sigma}$, % for~$\otimes_1$ above,
%and $\eta \in \Phi_{\Delta, \Delta}$ for~$\otimes_2$. 
%Similarly as for the above product and sum of functions, 
%the superscripts in $\bigoplus^1_\Sigma$ and $\bigoplus^2_\Sigma$
%shall be reserved to the ambiguous case of $\Phi_{\Sigma,\Sigma}$,
in order to be able to distinguish between the first and the second argument.
%
\begin{definition}\label{def:label-th-complete}
A label theory~$\bar\Phi$ is \emph{complete} when 
the underlying semiring~$\Semiring$ is complete, and
for all $\Phi_{\Sigma, \Delta} \in \bar\Phi$ 
and all $\eta \in \Phi_{\Sigma, \Delta}$,
$\bigoplus^1_\Sigma \eta \in \Phi_{\Delta}$ and 
$\bigoplus^2_\Delta \eta \in \Phi_{\Sigma}$.
\end{definition}
%
\florent{notion of diagram of functions akin BDD for transitions in practice}

\noindent 
The following facts are immediate.
\florent{mv appendix?}
%
\begin{lemma}
For $\bar\Phi$ complete %label theory for all 
$\alpha \in \Semiring$,
$\phi, \phi' \in \Phi_{\Sigma}$,
$\psi \in \Phi_{\Delta}$, and
$\eta \in \Phi_{\Sigma, \Delta}$:
%
\begin{enumerate}
\item[$i.$]
\( \bigoplus_{\Sigma}\bigoplus^2_{\Delta} \eta = \bigoplus_{\Delta}\bigoplus^1_{\Sigma} \eta \)
\item[$ii.$] 
\( \alpha \otimes \bigoplus_{\Sigma} \phi = \bigoplus_{\Sigma} (\alpha \otimes \phi) \) and
\( \bigl( \bigoplus_{\Sigma} \phi \bigr) \otimes\alpha = \bigoplus_{\Sigma} (\phi \otimes \alpha) \),
and similarly for~$\oplus$
\item[$iii.$]
\( \bigl(\bigoplus_{\Sigma} \phi\bigr) \oplus \bigl(\bigoplus_{\Sigma} \phi'\bigr) 
   = \bigoplus_{\Sigma} (\phi \oplus \phi') \) and
\( \bigl(\bigoplus_{\Sigma} \phi\bigr) \otimes \bigl(\bigoplus_{\Sigma} \phi'\bigr) 
   = \bigoplus_{\Sigma} (\phi \otimes \phi') \)
\item[$iv.$] 
\( \bigl(\bigoplus^2_{\Delta} \eta\bigr) \oplus \bigl(\bigoplus^2_{\Delta} \eta' \bigr) = 
 \bigoplus^2_{\Delta} (\eta \oplus \eta') \), and
\( \bigl(\bigoplus^2_{\Delta} \eta\bigr) \otimes \bigl(\bigoplus^2_{\Delta} \eta' \bigr) = 
 \bigoplus^2_{\Delta} (\eta \otimes \eta') \)
\item[$v.$] 
%\( \phi \oplus \bigl(\bigoplus_{\Delta} \eta\bigr) = \bigoplus_{\Delta} (\phi \oplus \eta) \),
\( \phi \otimes \bigl(\bigoplus^2_{\Delta} \eta\bigr) = \bigoplus_{\Delta} (\phi \otimes_1 \eta) \), and
\( \bigl(\bigoplus^2_{\Delta} \eta\bigr) \otimes \phi = \bigoplus_{\Delta} (\eta \otimes_1 \phi) \),
and similarly for~$\oplus$
\item[$vi.$] 
%\( \psi \oplus \bigl(\bigoplus_{\Sigma} \eta\bigr) = \bigoplus_{\Sigma} (\psi \oplus \eta) \),
\( \psi \otimes \bigl(\bigoplus^1_{\Sigma} \eta\bigr) = \bigoplus_{\Sigma} (\psi \otimes_2 \eta) \), and
\( \bigl(\bigoplus^1_{\Sigma} \eta\bigr) \otimes \psi = \bigoplus_{\Sigma} (\eta \otimes_2 \psi) \),
and similarly for~$\oplus$
\end{enumerate}
\end{lemma}

%we call \emph{summary} of a function
%$\phi \in \Phi_\Sigma$,
%resp. $\eta \in \Phi_{\Sigma, \Delta}$,
%the value $\bigoplus_{a \in \Sigma} \phi(a)$, 
%resp. $\bigoplus_{a \in \Sigma} \bigoplus_{b \in \Delta} \eta(a, b)$.
%By definition of infinite sums in complete semirings, 
%a summary of $\phi \oplus \phi'$, $\alpha \otimes \phi$ and $\phi \otimes \alpha$
%can be computed from $\alpha \in \Semiring$ and summaries of $\phi$ and $\phi'$ in $\Phi_{\Sigma}$, 
%using the operators of $\Semiring$, 
%and the same holds for $\Phi_{\Delta}$ and $\Phi_{\Sigma, \Delta}$. 


\philippe{Je trouve qu'il y a beaucoup de notions à retenir (complete, effective) et ça devient
difficile pour un lecteur non spécialiste. Est-ce que tout est nécessaire (je ne sais plus qui m'avait dit: 
un concept en plus, un point en moins.}

\noindent
A label theory is called \emph{effective} when 
for all $\phi \in \Phi_\Sigma$ and $\eta \in \Phi_{\Sigma, \Delta}$, 
$\bigoplus_{\Sigma} \phi$, 
$\bigoplus_{\Delta}\bigoplus_{\Sigma} \eta$, and
$\bigoplus_{\Sigma}\bigoplus_{\Delta} \eta$
can be effectively computed from $\phi$ and $\eta$.
\florent{$\exists$ oracle returning ...  in worst time complexity $T$.}


%there is an oracle returning, in constant time,
%$\bigoplus_{\Sigma} \phi$, 
%$\bigoplus_{\Sigma} \eta$, and
%$\bigoplus_{\Delta} \eta$
%and one symbol where the function reaches this minimum,
%denoted $\bigominus_\Sigma \phi$

%\begin{definition}\label{def:label-th-convex}
%Let $\Omega$ be an alphabet 
%A function $\phi \in \Phi_\Sigma$ in a label theory over a complete semiring $\Semiring$
%is called $k$-\emph{convex}, for a natural number $k$, iff 
%$\mathrm{card}\{ a \in \Sigma \mid \phi(a) = \bigoplus_{\Sigma} \phi \} \leq k$.
%\end{definition}
%A label theory is $k$-convex if all its functions are $k$-convex.

%s.t. for all $\phi \in \Phii$, 
%$\psi \in \Phir$, 
%and $\eta \in \Phicr$, 
%$\displaystyle\bigoplus_{a \in \Omegai} \phi(a)$ 
%$\displaystyle\bigoplus_{r \in \Omegar} \phi(r)$ and 
%$\displaystyle\bigoplus_{{\call{c}} \in \Omegac} 
%\displaystyle\bigoplus_{{\return{r}} \in \Omegar} \eta({\call{c}}, {\return{r}})$
%are computable...
% 
% total ? 
% monotonic and superior writ natural ordering
%Regarding the infinite sum operator, note that
%$\bigoplus_{x \in \Phi_\Omega} \phi(x)$, 
%$\bigoplus_{y \in \Phi_\Delta} \psi(y)$, and  
%... exist and in $\Semiring$.



%Concretely, in one of the language models defined below, 
%we consider a finite number of base functions $\phi, \eta$ of the underlying label theory,
%labelling transitions, and combine them with the above operators for construction of 
%other models.
%The combinations might be represented by dags (diagrams) whose leaves are labeled
%by base functions and inner nodes by operators.
% we can compute the value of a diagram in time...








%%%%%%%%%%%%%%%%%%%%%%%%%%%%%%%%%%%%%%%%%%%%%%%%%%%%%%%%%%%%%%%%%%%%%%%%%%%%%%%%%%%%%%%%%%%%%%%%%%%%
%% SWT & SWA
%%%%%%%%%%%%%%%%%%%%%%%%%%%%%%%%%%%%%%%%%%%%%%%%%%%%%%%%%%%%%%%%%%%%%%%%%%%%%%%%%%%%%%%%%%%%%%%%%%%%

\section{SW Automata and Transducers}
\label{section:transducer}\label{sec:transducer}
\label{section:SWA}\label{sec:SWA}
\label{section:SWT}\label{sec:SWT}

We follow the approach of~\cite{Mohri03EDWA} for the computation of distances,
between words and languages, using weighted transducers, 
and extend it to infinite alphabets.
% with models of symbolic weighted automata and transducers. 
%
The models introduced in this section generalize 
weighted automata and transducers~\cite{Droste09handbook} 
%over finite alphabets, see  e.g.~\cite{Mohri03EDWA}, 
by labelling each transition with a weight function (instead of a simple weight value), 
that takes the input and output symbols as parameters. 
These functions are similar to the guards of symbolic automata~\cite{dAntoniVeanes17CAV,dAntoni21CACM},
but they can return values in an generic semiring, 
whereas the latter guards are restricted to the Boolean semiring.


%\subsection{Definitions} \label{sec:SWTdef}\label{sec:SWAdef}
%(SWT)
\noindent 
Let $\Semiring$ be a commutative semiring, 
$\Sigma$ and $\Delta$ be alphabets called respectively \emph{input} and \emph{output}, %{alphabets}, 
and $\bar\Phi$ be a label theory over $\Semiring$
containing $\Phi_\Sigma$, $\Phi_\Delta$, $\Phi_{\Sigma, \Delta}$.

\begin{definition}
\label{def:transducer} \label{def:SWT}
A \emph{symbolic-weighted transducer} (\SWT)
over $\Sigma$, $\Delta$, $\Semiring$ and $\bar\Phi$
%the input and output alphabets~$\Sigma$ and $\Delta$ with label theory $\bar\Phi$, and the semiring $\Semiring$ 
is a tuple
$T = \< Q, \init, \bar{\wei}, \final >$,
where $Q$ is a finite set of states, 
$\mathsf{in} : Q \to \Semiring$   %\Phi_{\Sigma, \Delta}
respectively $\mathsf{out} : Q \to \Semiring$  %\Phi_{\Sigma, \Delta}
are functions defining the weight for entering, 
respectively leaving, computation in a state,  
and $\bar{\wei}$ is a triplet of transition functions 
$\wei_{10}: Q \times Q \to \Phi_{\Sigma}$,
$\wei_{01}: Q \times Q \to \Phi_{\Delta}$, and 
$\wei_{11}: Q \times Q \to \Phi_{\Sigma, \Delta}$.
\end{definition}
%
\noindent
For convenience, we shall sometimes present transition functions 
as functions of  
$Q \times (\Sigma \cup \{ \varepsilon \}) \times (\Delta \cup \{ \varepsilon \}) \times Q \to \Semiring$,
overloading the function names,
such that, for all $q, q' \in Q$, $a \in \Sigma$,  $b \in \Delta$, 
\[
\begin{array}{rcll}
\wei_{10}(q, a, \varepsilon, q') & = & \phi(a) & %\wei_\Sigma(q, q')(a)
\quad\mathrm{where~} \phi = \wei_{10}(q, q') \in \Phi_\Sigma,\\
\wei_{01}(q, \varepsilon, b, q') & = & \psi(b) & 
\quad\mathrm{where~} \psi = \wei_{01}(q, q') \in \Phi_\Delta,\\
\wei_{11}(q, a, b, q') & = & \eta(a, b) & 
\quad\mathrm{where~} \eta = \wei_{11}(q, q') \in \Phi_{\Sigma, \Delta}.\\
\end{array}      
\]
%
\noindent 
% The symbolic-weighted transducer
The \SWT $T$ computes on pairs of words $\< s, t> \in \Sigma^* \times \Delta^*$,
$s$ and $t$, being respectively called input and output word.
% \noindent
More precisely, $T$ defines a mapping 
from $\Sigma^* \times \Delta^*$ into~$\Semiring$,
based on an intermediate function $\weight_T$
defined recursively for every states $q, q' \in Q$, 
and every strings $\< s, t> \in \Sigma^* \times \Delta^* \setminus \{ \< \varepsilon, \varepsilon >\}$ by:
%$a \in \Sigma$, $u \in \Sigma^*$, $b\in \Delta$, $v\in \Delta^*$, by:
%
\begin{align}
\weight_T(q, \varepsilon, \varepsilon, q)  & = \one \label{eq:SWT-weight}\\ %\final(q)\\
\weight_T(q, \varepsilon, \varepsilon, q') & = \zero \quad \mathrm{if~} q \neq q'\nonumber\\
%\weight_T(q, au, \varepsilon, q') & = \displaystyle\bigoplus_{q'' \in Q} 
%    \wei_{10}(q, a, \varepsilon, q'') \otimes \weight_T(q'', u, \varepsilon, q')\nonumber\\
%\weight_T(q, \varepsilon, bv, q') & = \displaystyle\bigoplus_{q'' \in Q} 
%    \wei_{01}(q, \varepsilon, b, q'') \otimes \weight_T(q'', \varepsilon, v, q')\nonumber\\
\weight_T(q, s, t, q') & = \displaystyle\bigoplus_{\begin{array}{c}
                                                   \scriptstyle q'' \in Q\\
                                                   \scriptstyle s = au,\, a \in \Sigma
                                                   \end{array}} 
    \wei_{10}(q, a, \varepsilon, q'') \otimes \weight_T(q'', u, t, q')\nonumber\\
                    & \oplus \displaystyle\bigoplus_{\begin{array}{c}
                                                     \scriptstyle q'' \in Q\\ 
                                                     \scriptstyle t = bv,\, b \in \Delta\\
                                                     \end{array}} 
    \wei_{01}(q, \varepsilon, b, q'') \otimes \weight_T(q'', s, v, q')\nonumber\\
                    & \oplus \displaystyle\bigoplus_{\begin{array}{c}
                                                     \scriptstyle q'' \in Q\\
                                                     \scriptstyle s = au,\, t = bv\\
                                                     \end{array}} 
    \wei_{11}(q, a, b, q'') \otimes \weight_T(q'', u, v, q')\nonumber
\end{align}
%
We recall that, by convention (Section~\ref{sec:semiring}), 
an empty sum with $\bigoplus$ is equal to~$\zero$. 
%
Intuitively, using a transition $\wei_{ij}(q, a, b, q')$ means for $T$:
when reading respectively $a$ and $b$ at the current positions in the input and output words, 
increment the current position in the input word if and only if $i = 1$, 
and in the output word iff $j = 1$ (otherwise, do not change it),
and change state from $q$ to $q'$.
When $a = \varepsilon$ (resp. $b = \varepsilon$), the current symbol 
in the input (resp. output) is not read.
%
%In contrast with the models of weighted transducers over finite alphabets~\cite{Mohri03ijfcs},
%the input and output symbols at current positions are always read by transitions, 
%even when they do not change the reading position the head's position.
%This is an important feature in the case of an infinite alphabet in 
%order to compare input and output symbols.
%which cannot be stored in the finite memory of the transducer.
%
Since $\zero$ is absorbing for~$\otimes$ in~$\Semiring$,
one term $\wei_{ij}(q, a, b, q'')$ equal to $\zero$ in the above expression 
will be ignored in the sum, meaning that there is no possible transition
from state $q$ into state $q'$ while reading $a$ and $b$.
This is analogous to the case of a transition's guard not satisfied by $\<a, b>$ for 
symbolic transducers.

%whereas considering the current symbol may be useful to compute a transition weight.
%(even when it does not change the head's position, like with $\varepsilon$-transitions).
%
%The cases $\weight_T(q, au, \varepsilon, q')$ and $\weight_T(q, \varepsilon, bv, q')$
%are missing in the definition of $\weight_T$.
%It means that $T$ must avoid configurations where it reached the end of 
%the output word and not of the input one, or vice-versa.
%This can be done by using $\wei_{10}$ and $\wei_{01}$
%before reaching the end of word, and using a special state for this purpose.

The expression \eqref{eq:SWT-weight} 
can be seen as a stateful definition of 
an edit-distance between a word $s \in \Sigma^*$ and a word $t \in \Delta^*$,
see also~\cite{Mohri03ijfcs}.
Intuitively, 
$\wei_{10}(q, a, \varepsilon, r)$ is the cost of 
the deletion of the symbol~$a \in \Sigma$ in~$s$, 
$\wei{01}(q, \varepsilon, b, r)$ is the cost 
of the insertion of~$b \in \Delta$ in $t$, 
and $\wei_{11}(q, a, b, r)$ is the cost 
of the substitution of  $a \in \Sigma$ by~$b \in \Delta$.
%
The cost of a sequence of such operations transforming $s$ into $t$, 
is the product, with $\otimes$, of the individual costs of the operations involved;
And the distance between $s$ and $t$ is the sum, with $\oplus$,
of all possible products.

\medskip\noindent
%Let $\< s, t> \in \Sigma^* \times \Delta^*$, with $s = s_1\ldots s_n$, and $t = t_1\ldots t_m$. 
Formally, the weight associated by $T$ to $\< s, t> \in \Sigma^* \times \Delta^*$ is: 
%defined as follows:
\begin{equation}
T(s, t)  = 
\displaystyle\bigoplus_{q, q' \in Q} \init(q) 
\mathop{\otimes} \weight_T(q, s, t, q') \mathop{\otimes} \final(q')
\label{eq:SWT-value}
\end{equation}

%\begin{example}
%We can define with a \SWT the computation of a similarity measure 
%between timed sequences
%similar to dynamic time warping (DTW).
%
%Let $\Semiring$  be the tropical (\emph{min-plus}) semiring of Figure~\ref{fig:semirings} and 
%let $\Sigma = \Delta = \mathbb{R}_+$ be sets of timestamps.
%We consider a \SWT with one state $q$ and transitions
%$\wei_{11}(q, d, d', q) = 
% \wei_{01}(q, d, d', q) = 
% \wei_{10}(q, d, d', q) = |d' - d|$,
%for all $d, d' \in \mathbb{R}_+$.
%% needs reading input/output symbols even by epsilon-transitions.
%The recursive definition of $\weight_T$ correspond to the dynamic programming equations of DTW
%for the computation of an optimal match between words, 
%the matching cost for two symbols being the 
%the time distance between them.
%\endex
%\end{example}

\begin{example}
In Common Western Music Notation~\cite{Gould11Notation}, 
several symbols may be used to represent a unique sounding event.
For instance, several notes can be combined with a tie, 
like in $\musQuarter\!\!\!\mathrel{\raisebox{-1.5mm}{$\smile$}}\!\!\!\musEighth$,  
and one note can be augmented by half its duration with a dot like in~\musQuarterDotted{}

\noindent 
We propose a small transducer model that compares an input sequence of sounding 
events (music "performance") 
to an output sequence of written events (music "score").
% simple pointwise distance between two sequences of timestamped events **
%
Let us consider the tropical (\emph{min-plus}) semiring~$\Semiring$ 
of Figure~\ref{fig:semirings} and 
let $\Sigma = \mathbb{R}_+$ be an input alphabet of event dates
and $\Delta = \{ \mathsf{e}, \mathsf{-} \} \times \mathbb{R}_+$ 
be an output alphabet of symbols with timestamps. 
A symbol $\< \mathsf{e}, d > \in \Delta$ represents an event starting at date $d$, 
and $\< \mathsf{-}, d >$ is a continuation of the previous event.
%More precisely, we let 
%$\Sigma = \{ a, - \} \times \R_+$ and 
%This example of $\Delta$ is motivated by the case of music notation, 
%where several notated events (notes) can be tied together, 
%with a \emph{tie} or a \emph{dot}
%meaning that they will be played as a unique sounding event.
%The timestamp of $a \in \Sigma$, denoted by $\mathsf{t}(a)$, is expressed as a rational number.

We consider a \SWT with two states $q_0$ and $q_1$ whose purpose 
is to compare a recorded performance $s \in \Sigma^*$
with notated music sheet $t \in \Delta^*$.
One timestamp $d_i \in \Sigma$ may corresponds 
to one notated event $\<\mathsf{e}, d'_i> \in \Sigma$, in which case 
the weight value computed by the \SWT is the time distance between both
(see transitions $\wei_{11}$ below).
%
If $\<\mathsf{e}, d'_i>$ is followed by continuations 
$\<\mathsf{-}, d'_{i+1}>$..., they are just skip with no cost (transitions $\wei_{01}$ or weight $\one$).
%transitions
\[
\begin{array}{rclcrcl}
\wei_{11}(q_0, d, \< \mathsf{e}, d'>, q_0) & = & |d' - d| & \quad &
\wei_{11}(q_1, d, \< \mathsf{e}, d'>, q_0) & = & |d' - d|\\
\wei_{01}(q_0, \varepsilon, \< \mathsf{-}, d'>, q_0) & = & \one & &
\wei_{01}(q_1, \varepsilon, \< \mathsf{-}, d'>, q_0) & = & \one\\
\wei_{10}(q_0, d, \varepsilon, q_1) & = & \alpha & & %\multicolumn{3}{l}{\mathrm{for~all~} b \in \Delta}
\end{array}
\]
%
Moreover, it may happen that the performers plays an extra note accidentally, but only once in a row. 
This is modelled by the transition $\wei_{10}$ with an arbitrary weight value $\alpha \in \Semiring$, 
switching from state $q_0$ (normal) to $q_1$ (error).
The transitions in the second column below switch back to the normal state $q_0$.
% the metric computed by the \SWT is the smallest sum of point wise distances 
% between dates of input and output events.
At last, we let $q_0$ be the only initial and final state, with
$\init(q_0) = \final(q_0) = \one$, and 
$\init(q_1) = \final(q_1) = \zero$. 
%$\init(q_0, d, b) = \final(q_0, d, b) = \one$, and 
%$\init(q_1, d, b) = \final(q_1, d, b) = \zero$, 
%for all $d \in \Sigma$ and $b \in \Delta$.
\endex
\end{example}

\noindent
The \emph{Symbolic Weighted Automata} %$A = \< Q, \init, \weight, \final >$
%over $\Sigma$,  $\Semiring$ and $\bar\Phi$
are defined similarly as the transducers of Definition~\ref{def:SWT}, 
by simply omitting the output symbols.
%
In this case, the label theory $\bar\Phi$ can be reduced to a singleton $\< \Phi_\Sigma>$.
%over $\Sigma$ is reduced to
%a set $\Phi_\Sigma$ closed under~$\oplus$ and~$\otimes$.
%
\begin{definition} \label{def:SWA}
A \emph{symbolic-weighted automaton} (\SWA)
over $\Sigma$, $\Semiring$ and $\bar\Phi$
%the input alphabet~$\Sigma$ and the commutative semiring $\Semiring$ 
is a tuple
$A = \< Q, \init, {\wei_1}, \final >$,
where $Q$ is a finite set of states, 
$\mathsf{in} : Q \to \Semiring$, %\Phi_\Sigma$, 
respectively $\mathsf{out} : Q \to \Semiring$  %\Phi_\Sigma,$
are functions defining the weight for entering, 
respectively leaving, computation in a state, 
and ${\wei_1}$ is a transition functions 
from $Q \times Q$ into~$\Phi_{\Sigma}$.
\end{definition}
%      
\noindent
As above in the case of \SWT, 
when $\wei_1(q, q') = \phi \in \Phi_\Sigma$, 
%respectively $\mathsf{in}(q) = \phi$, $\mathsf{out}(q') = \phi$,
we may write $\wei_1(q, a, q')$ for~$\phi(a)$. 
%respectively $\mathsf{in}(q, a) = \phi(a)$, $\mathsf{out}(q', a) = \phi(a)$.
%$\wei_1: Q \times \Sigma \times Q \to \Semiring$, 
The computation of $A$ on words $s \in \Sigma^*$
is defined with an intermediate function $\weight_A$, 
defined as follows for $q, q' \in Q$, $a \in \Sigma$, $u \in \Sigma^*$,
%
\begin{align}
\weight_A(q, \varepsilon, q) & = \one\\ %\final(q)\\
\weight_A(q, \varepsilon, q') & = \zero \quad \mathrm{if~} q \neq q'\nonumber\\
\weight_A(q, au, q') & =  \displaystyle\bigoplus_{q'' \in Q} 
    \wei_{1}(q, a, q'') \otimes \weight_A(q'', u, q')\nonumber
\label{eq:SWA-weight}
\end{align}
%
\noindent
and the weight value associated by $A$ to 
$s \in \Sigma^*$ is defined as follows: %$s = s_1\ldots s_n \in \Sigma^+$ 
\begin{equation}
A(s)  = 
\displaystyle\bigoplus_{q, q' \in Q} \init(q) 
\mathop{\otimes} \weight_A(q, s, q') \mathop{\otimes} \final(q') 
\label{eq:SWA-value}
\end{equation}


%
%When $\wei_\varepsilon(q, q') = \zero$ for all $q, q' \in Q$, 
%the automaton~$A$ is called \emph{without $\varepsilon$-transitions}.
      
%The \emph{summary} of a $\SWT$, resp. a $\SWA$, is ***
      
      
      
%\subsection{Properties}
\noindent
The following property will be useful to the approach on 
symbolic weighted parsing presented in Section~\ref{sec:parsing}.

\begin{proposition} \label{prop:epsilon}
Given a \SWT $T$ over $\Sigma$, $\Delta$, 
$\Semiring$ commutative, bounded and complete,
and $\bar\Phi$ effective,
and a $\SWTA$ $A$ over $\Sigma$, $\Semiring$ and $\bar\Phi$,
there exists an effectively constructible \SWA 
$B_{T, A}$ over $\Delta$, $\Semiring$ and $\bar\Phi$,
such that for all $t \in \Delta^*$, 
$B_{T, A}(t) = \displaystyle\bigoplus_{s\in \Sigma^*} A(s) \otimes T(s, t)$.
\end{proposition}
%
\begin{proof}
Let $T = \< Q, \init_T, \bar{\wei}, \final_T >$,
where $\bar{\wei}$ contains $\wei_{10}$, $\wei_{01}$, and $\wei_{11}$,
from $Q \times Q$ into respectively 
$\Phi_{\Sigma}$, $\Phi_{\Delta}$, and $\Phi_{\Sigma, \Delta}$,
and let $A = \< P, \init_A, \wei_1, \final_A >$
with $\wei_1: Q \times Q \to \Phi_{\Sigma}$.
%
\noindent
The state set of $B_{T, A}$ will be $Q' = P \times Q$.
The entering, leaving and transition functions of $B_{T, A}$ will 
simulate synchronized computations of $A$ and $T$, 
while reading an output word of $\Delta^*$.
%
Its state entering functions is defined 
for all $p \in P$, $q \in Q$ 
by $\init'(p, q) = \init_A(p) \otimes \init_T(q)$.
The transition function $\wei'_1$ will roughly perform 
a synchronized product of transitions defined by $\wei_1$, 
$\wei_{01}$ ($T$ reading in output word and not in input word)
and $\wei_{11}$ ($T$ reading in output word and input word).
%
Moreover, $\wei'_1$ also needs to simulate transitions 
defined by $\wei_{10}$: $T$ reading in intput word and not in output word.
Since $B_{T, A}$  will read only in the output word, such a transition corresponds
to an $\varepsilon$-transition of $\SWA$, 
but $\SWA$ have been defined without $\varepsilon$-transitions.
Therefore, in order to take care of this case, we perform an on-the-fly
suppression of $\varepsilon$-transition in the $\SWA$ in construction, 
following the algorithm of~\cite{LombardySakarovitch12ciaa}. 
%
% Initialize state leaving functions i
%and $\final'(p, q) = \final_A(p) \otimes \final_T(q)$. 

%Every transition of $B_{T, A}$ will
%simulate a sequence of transitions of $T$ performing the following steps:
%advance in the input word while staying immobile in the output word, 
%and then make one step in the output word (and advance in the input word or not).

\noindent
Initially, for all $p_1, p_2 \in P$, and $q_1, q_2 \in Q$, let
\[
\wei'_1\bigl( \< p_1, q_1>, \< p_2, q_2>\bigr) = 
\wei_1(p_1, p_2) \otimes
\bigl[
\wei_{01}(q_1, q_2) 
\oplus
\displaystyle\bigoplus_{\Sigma}
\wei_{11}(q_1, q_2) 
\bigr].
\]

\noindent
Iterate the following for all $p_1\in P$ and $q_1, q_2 \in Q$:
for all $p_2\in P$ and $q_3 \in Q$,
\[
\wei'_1\bigl( \< p_1, q_1>, \< p_2, q_3>\bigr) \opluseq 
\displaystyle\bigoplus_{\Sigma} \wei_{10}(q_1, q_2) 
\otimes 
\wei'_1\bigl( \< p_1, q_2>, \< p_2, q_3>\bigr)
\]
and
\[
\final'(p_1, q_1) \opluseq 
\displaystyle\bigoplus_{\Sigma} \wei_{10}(q_1, q_2) 
\otimes \final'(p_1, q_2)
\] 
\marginpar{\tiny proof correctness}
\qed
\end{proof}

%The complexity of construction
\noindent
The construction time and size for $B_{T, A}$ are $O(\| T \|^3 . \| A \|^2)$,
where the sizes $\| T \|$ and $\| A \|$ are their number of states.
%of~$T$ is its number of states $|Q|$.



\begin{corollary} \label{cor:epsilon}
Given a \SWT $T$ over $\Sigma$, $\Delta$, 
$\Semiring$ commutative, bounded and complete,
and $\bar\Phi$ effective,
and $s \in \Sigma^+$, 
there exists an effectively constructible \SWA 
$B_{T, s}$ over $\Delta$, $\Semiring$ and $\bar\Phi$,
such that for all $t \in \Delta^*$, $B_{T, s}(t) = T(s, t)$.
\end{corollary}


%\noindent
%The construction time and size for $B_{T, s}$ are $O(\| T \|^3 . | s |^2)$,
%where the size $\| T \|$ of~$T$ is its number of states $|Q|$.

 





%%%%%%%%%%%%%%%%%%%%%%%%%%%%%%%%%%%%%%%%%%%%%%%%%%%%%%%%%%%%%%%%%%%%%%%%%%%%%%%%%%%%%%%%%%%%%%%%%%%%
%% SW-VPA
%%%%%%%%%%%%%%%%%%%%%%%%%%%%%%%%%%%%%%%%%%%%%%%%%%%%%%%%%%%%%%%%%%%%%%%%%%%%%%%%%%%%%%%%%%%%%%%%%%%%


\section{SW Visibly Pushdown Automata}
\label{section:SWVPA}\label{sec:SWVPA}
The model presented in this section generalizes Symbolic VPA~\cite{dAntonyAlur14SVPDA}
from Boolean semirings to arbitrary semiring weight domains.
It will compute on nested words over infinite alphabets, 
associating to every such word a weight value. 
Nest words are able to describe structures of labeled trees, 
and in the context of parsing, they will be useful to 
represent parse trees.



\subsection{Definition}\label{se:SWVPA-def}
Let $\Omega$ be a countable alphabet 
%finite (large) or infinite,
that we assume partitioned into three 
subsets~$\Omegai$, $\Omegac$, $\Omegac$,
whose elements are respectively called 
\emph{internal}, \emph{call} and \emph{return} symbols.
% names are  coined by application to functional program verification
% \begin{itemize}
% \item a set $\Omegai$ of \emph{internal symbols} denoted $a$,
% \item a set $\Omegac$ of \emph{call symbols} denoted $\call{a}$,
% \item a set $\Omegar$ of \emph{return symbols} denoted $\return{a}$.
% \end{itemize}
Let~$\< \Semiring, \oplus, \zero, \otimes, \one>$ be a commutative and complete semiring and let  
$\bar\Phi = \< \Phii, \Phic, \Phir, \Phici, \Phicc, \Phicr>$ 
be a label theory over $\Semiring$
%In order to simplify notations, %and following the definition of Section~\ref{section:transducer}, 
%we shall write respectively 
where $\Phii$, $\Phic$, $\Phir$ and~$\Phicx$ (with $\mathsf{x} \in \{ \mathsf{i}, \mathsf{c}, \mathsf{r}\}$) 
stand respectively 
for~$\Phi_\Omegai$, $\Phi_\Omegac$, $\Phi_\Omegar$ and~$\Phi_{\Omegac, \Omega_\mathsf{x}}$.
%
%Moreover, we extend this theory with a set $\Phii$ 
%of unary functions in $\Omegai \to \Semiring$,
%closed under $\oplus$ and $\otimes$.

\begin{definition}
A \emph{Symbolic Weighted Visibly Pushdown Automata} (\SWVPA) 
over  $\Omega = \Omegai \uplus \Omegac \uplus \Omegar$, $\Semiring$ and $\bar\Phi$ 
is a tuple $A = \< Q, P, \init, \bar\wei, \final >$,
where $Q$ is a finite set of states, 
$P$ is a finite set of stack symbols, 
$\mathsf{in} : Q \to \Semiring$, 
respectively $\mathsf{out} : Q \to \Semiring,$
are functions defining the weight for entering, 
respectively leaving, a state, 
and $\bar\wei$ is a sextuplet composed of the transition functions:
$\weii : Q \times P \times Q \to \Phici$,  
$\weiei : Q \times Q \to \Phii$,  
$\weic : Q \times P \times Q \times P \to \Phicc$,  
$\weiec : Q \times P \times Q \to \Phic$,  
$\weir : Q \times P \times Q \to \Phicr$,  
$\weier : Q \times Q \to \Phir$.
%and 
%$\weiex : Q \times Q \to \Phix$ 
%with $\mathsf{x} \in \{ \mathsf{i}, \mathsf{c}, \mathsf{r}\}$.
\end{definition}
%
Similarly as in Section~\ref{section:transducer}, 
we extend the above transition functions as follows
for all $q, q' \in Q$, $p \in P$, 
$a \in \Omegai$, 
$\call{c} \in \Omegac$, 
$\return{r} \in \Omegar$, 
overloading their names: % for simplicity:
\[
\begin{array}{lll}
\weii: Q \times \Omegac \times P \times \Omegai \times Q \to \Semiring & 
\weii(q, c, p, a, q') = \eta_\mathsf{ci}(c, a) & 
\mathrm{where~} \eta_\mathsf{ci} = \weii(q, p, q'),\\
%
\weiei: Q \times \Omegai \times Q \to \Semiring & 
\weiei(q, a, q') = \phi_\mathsf{i}(a) &
\mathrm{where~} \phi_\mathsf{i} = \weiei(q, q').\\[2pt]
%
\weic: Q \times \Omegac \times P \times  \Omegac \times P \times Q \to \Semiring & 
\weic(q, c, p, \call{c'}, p', q') = \eta_\mathsf{cc}(c, \call{c'}) & 
\mathrm{where~} \eta_\mathsf{cc} = \weic(q, p, p', q'),\\
%
\weiec: Q \times \Omegac \times P \times Q \to \Semiring & 
\weiec(q, {\call{c}}, p, q') = \phi_\mathsf{c}({\call{c}}) &
\mathrm{where~} \phi_\mathsf{c} = \weiec(q, p, q').\\[2pt]
%
\weir: Q \times \Omegac \times P \times \Omegar \times Q \to \Semiring & 
\weir(q, {\call{c}},  p, {\return{r}}, q') = \eta_\mathsf{cr}({\call{c}},  {\return{r}}) & 
\mathrm{where~} \eta_\mathsf{cr} = \weir(q, p, q'),\\
%
\weier: Q \times \Omegar \times Q \to \Semiring & 
\weier(q, {\return{r}}, q') = \phi_\mathsf{r}({\return{r}}) &
\mathrm{where~} \phi_\mathsf{r} = \weier(q, q').\\
\end{array}      
\]

\noindent
The intuition is the following for the above transitions.

\noindent
$\weii$ and $\weiei$ both read an input internal symbol $a$ and change state from $q$ to $q'$, 
without changing the stack. 
Moreover, $\weii$ reads a pair made of 
${\call{c}} \in \Omegac$ and $p \in P$ at the top of the stack 
($c$ is compared to $a$ by the weight function $\eta_\mathsf{ci} \in \Phici$)
and $\weiei$ applies if and only if the stack is empty.

\noindent
$\weic$ and $\weiec$ read the input call symbol $\call{c}'$, 
push it to the stack along with $p'$, and change state from $q$ to to $q'$.
Moreover, $\weic$ reads ${\call{c}}$ and $p$ at the top of the stack 
($c$ is compared to $c'$),
and $\weiec$ applies iff the stack is empty.

\noindent
$\weir$ and $\weier$ read the input return symbol $\return{r}$, and change state from $q$ to to $q'$.
Moreover, $\weir$ reads and pop from stack a pair made of $\call{c}$ and $p$, 
($\call{c}$ is compared to $\return{r}$),
and $\weier$ applies iff the stack is empty.
%In this case, the weight function $\phi_\mathsf{r}$ 
%computes a value of matching between the call and return symbols $c$ and $r$.
%This value might be set to $\zero$ in order to express that the symbols do not match.

Formally, the transitions of the automaton~$A$ are defined
in term of %a weight value computed by 
an intermediate function $\weight_A$, like in Section~\ref{sec:SWT}.
In the case of a pushdown automaton, a configuration, denoted by $q[\gamma]$ 
is composed of a state $q \in Q$ 
and a stack content $\gamma \in \Gamma^*$, 
where $\Gamma = \Omegac \times P$.
Hence, $\weight_A$ is a function from 
$[Q \times \Gamma^*] \times \Omega^* \times [Q \times \Gamma^*]$ into~$\Semiring$
(the empty stack is denoted by $\bot$). 
%
\begin{align}
%\begin{array}[t]{rcl}
\weight_A\bigl(\config{q}{\bot}, \varepsilon, \config{q}{\bot}) & = \one\label{eq:SWVPA-weight}\\
\weight_A\bigl(\config{q}{\bot}, \varepsilon, \config{q'}{\gamma}) & = \zero 
\mathrm{~if~} q \neq q'\nonumber\\
\weight_A\bigl(\configup{q}{\<{\call{c}}, p>\stackup\gamma}, a\, u, 
               \config{q'}{\gamma'}\bigr) & =  
 {\displaystyle\bigoplus_{q'' \in Q}} \weii(q, c, p, a, q'') 
  \otimes \weight_A\bigl(\configup{q''}{\<{\call{c}}, p> \stackup \gamma}, u, 
                         \config{q'}{\gamma'}\bigr)\nonumber\\
 %
\weight_A\bigl(\config{q}{\bot}, a\, u, 
               \config{q'}{\gamma'}\bigr) & =  
  {\displaystyle\bigoplus_{q'' \in Q}} \weiei(q, a, q'') 
   \otimes \weight_A\bigl(\config{q''}{\bot}, u, \config{q'}{\gamma'}\bigr)\nonumber\\
%
\weight_A\bigl(\configup{q}{\<{\call{c}}, p>\stackup\gamma}, {\call{c}'}\, u, 
               \config{q'}{\gamma'}\bigr) & =  
 {\displaystyle\bigoplus_{\begin{array}{c}
                          \scriptstyle q'' \in Q\\[-2pt]
                          \scriptstyle p' \in P
                          \end{array}}}
 \weic\bigl(q, {\call{c}}, p, {\call{c}'}, p', q''\bigr) 
 \otimes \weight_A\bigl(\configup{q''}{\<{\call{c}'}, p'>\stackup \<{\call{c}}, p>\stackup \gamma}, u, 
                        \config{q'}{\gamma'}\bigr)\nonumber\\[1mm]
%
\weight_A\bigl(\config{q}{\bot}, {\call{c}}\, u, 
               \config{q'}{\gamma'}\bigr) & =  
 {\displaystyle\bigoplus_{\begin{array}{c}
                          \scriptstyle q'' \in Q\\[-2pt]
                          \scriptstyle p \in P
                          \end{array}}}
  \weiec(q, {\call{c}}, p, q'') 
  \otimes \weight_A\bigl(\config{q''}{{\call{c}}\, p}, u, 
                         \config{q'}{\gamma'}\bigr)\nonumber\\
%
\weight_A\bigl(\configup{q}{\<{\call{c}}, p>\stackup \gamma}, {\return{r}}\, u, 
               \config{q'}{\gamma'}\bigr) & =  
 {\displaystyle\bigoplus_{q'' \in Q}} 
  \weir\bigl(q, {\call{c}}, p, {\return{r}}, q''\bigr) 
  \otimes \weight_A\bigl(\config{q''}{\gamma}, u, 
                         \config{q'}{\gamma'}\bigr)\nonumber\\
%
\weight_A\bigl(\config{q}{\bot}, {\return{r}}\, u, 
               \config{q'}{\gamma'}\bigr) & =  
 {\displaystyle\bigoplus_{q'' \in Q}} \weier(q, {\return{r}}, q'') 
  \otimes \weight_A\bigl(\config{q''}{\bot}, u, 
                         \config{q'}{\gamma'}\bigr)\nonumber
%\end{array}
\end{align}
%
% and ${\call{c}}\, p\stacksep \gamma$ 
%denotes a stack where the pair made of ${\call{c}} \in \Omegac$ and $p \in P$ is the top symbol 
%and $\gamma$ is the rest of stack.

\noindent
The weight associated by $A$ to $s \in \Omega^*$
is defined according to empty stack semantics: 
%
\begin{equation}
A(s)  = 
{\displaystyle\bigoplus_{q, q' \in Q}} \textstyle
\mathsf{in}(q) \mathop{\otimes} 
\weight_A\bigl(\config{q}{\bot}, s, \config{q'}{\bot}\bigr) 
\mathop{\otimes} \mathsf{out}(q').
\label{eq:weightA}
\end{equation}

\begin{example}
structured words with timed symbols...
intro language of music notation? (markup = time division, leaves = events etc)
\end{example}

\noindent 
Every $\SWA$ $A = \< Q, \init, {\wei_1}, \final >$,
over $\Sigma$, $\Semiring$ and $\bar\Phi$
is a particular case of $\SWVPA$ 
$\< Q, \emptyset, \init, \bar\wei, \final >$ 
over $\Omega$, $\Semiring$ and $\bar\Phi$
with $\Omegai = \Sigma$ and $\Omegac = \Omegar = \emptyset$,
and computing with an always empty stack:
$\weiei = \wei_1$ and all the other functions 
of~$\bar\wei$ are the constant~$\zero$.



\subsection{Properties}
Like VPA and symbolic VPA, 
the class of \SWVPA is closed under the binary operators of the underlying semiring.
%
\begin{proposition}\label{prop:SWVPA-product}
Let $A_1$ and $A_2$ be two \SWVPA
over the same $\Omega$, $\Semiring$ and $\bar\Phi$.
There exists two effectively constructible $\SWVPA$ 
$A_1 \oplus A_2$ and $A_1 \otimes A_2$,  
such that for all $s \in \Omega^*$, 
$(A_1 \oplus A_2)(s) = A_1(s) \oplus A_2(s)$ and 
$(A_1 \otimes A_2)(s) = A_1(s) \otimes A_2(s)$.
\end{proposition}
%
\begin{proof}
The construction is essentially the same 
as in the case of the Boolean semiring~\cite{dAntonyAlur14SVPDA}.
\end{proof}


\subsection{Best Search} 
\label{sec:best}\label{sec:search}
%**hypotheses**
Let us assume that the semiring~$\Semiring$ is
commutative, bounded, and complete,  
\marginpar{\tiny total?}
and assume an effective label theory.
%s.t. for all $\phi \in \Phii$, 
%$\psi \in \Phir$, 
%and $\eta \in \Phicr$, 
%$\displaystyle\bigoplus_{a \in \Omegai} \phi(a)$ 
%$\displaystyle\bigoplus_{r \in \Omegar} \phi(r)$ and 
%$\displaystyle\bigoplus_{{\call{c}} \in \Omegac} 
%\displaystyle\bigoplus_{{\return{r}} \in \Omegar} \eta({\call{c}}, {\return{r}})$
%are computable...
% 
% total ? 
% monotonic and superior writ natural ordering
%Regarding the infinite sum operator, note that
%$\bigoplus_{x \in \Phi_\Omega} \phi(x)$, 
%$\bigoplus_{y \in \Phi_\Delta} \psi(y)$, and  
%... exist and in $\Semiring$.
%
We propose a Dijkstra algorithm computing, for a $\SWVPA$ $A$
over~$\Omega$, $\Semiring$ and~$\bar\Phi$, 
the minimal weight %\wrt~$\leq_\oplus$, 
for a word in~$\Omega^*$.

\noindent
More precisely, 
let $b_\bot : Q \times Q \to \Semiring$ be the function:
%
\begin{equation}\label{eq:bbot}
  b_\bot(q, q') = \bigoplus_{s\in \Omega^*} 
  \textstyle
  \weight_A\bigl(\config{q}{\bot}, s, \config{q'}{\bot}\bigr)
\end{equation}
%
Since $\Semiring$ is complete, the infinite sum in \eqref{eq:bbot} is well defined,
and, roviding that $\Semiring$ is total, it is the minimum in $\Omega^*$,
\wrt~$\leq_\oplus$, % this ordering.
of the fonction 
$s \mapsto \weight_A(\config{q}{\sigma}, s, \config{q'}{\sigma})$.
%
The term $\config{q}{\bot}, s, \config{q'}{\bot}$ 
of this sum is the central expression in 
the definition \eqref{eq:weightA} of $A(s_0)$, for the minimum $s_0$
of the function $\weight_A$.

%
\noindent
Let $\top$ be a fresh stack symbol which does not belong to $\Gamma$,
and let $b_\top : Q \times P \times Q \to \Phic$ be such that,
for every two states $q, q' \in Q$ 
and stack symbol $p \in P$: %and $\sigma \in \{ \bot, \top \}$, 
\begin{equation}\label{eq:btop}
  b_\top(q, p, q') : c \mapsto \bigoplus_{s\in \Omega^*} 
  \textstyle
  \weight_A\bigl(\configup{q}{\< c, p> \stackup \top }, s, \configup{q'}{\<c, p> \stackup \top}\bigr). 
\end{equation}
%
Intuitively, the function defined in \eqref{eq:btop}
associateds to $c \in \Omegac$ 
the minimum weight of a computation of $A$
starting in state $q$ with a stack 
$c p \cdot \gamma \in \Gamma^+$ 
and ending in state $q'$ with the same stack,
such that the computation does pop 
the pair made of $c$ and $p$ at the top of this stack,
but may read these symbols.
Moreover, $A$ may push another pair $\< c', p'>$ %call symbols 
on the top of $c p \cdot \gamma$,
following the the third case of 
in the definition \eqref{eq:SWVPA-weight} of $\weight_A$,
and may pop $\< c', p'>$ later, following the fifth case of \eqref{eq:SWVPA-weight} (return symbol). 
%However, it cannot apply one of the two last cases (return symbol and empty stack)
%when the current stack is $\gamma$.
%pop symbols in $\gamma$.
% Note that having a stack reduced to such a symbol makes impossible the application of the 
% two last cases in the definition of $\weight_A$ (return symbol and empty stack). 
% However, it is possible to apply the two first cases 
% (internal symbol or call symbol, with a push on the top of $\top$).

Algorithm~\ref{algo:Dijkstra}
constructs iteratively markings 
$d_\bot : Q \times Q \to \Semiring$ and 
$d_\top : Q \times P \times Q \to \Phic$
%of the triplets $\<q, \sigma, q'>$ 
%of states of $A$ by weight values in $\Semiring$, 
that converges eventually to $b_\top$ and $b_\bot$.
%It uses for that purpose a priority queue $P$ containing triplets of 
%$Q \times \{ \bot, \top \} \times Q$.


\begin{algo}[best search for \SWVPA]\label{algo:Dijkstra}
\textbf{initially} 
let $\Q = (Q \times Q) \cup (Q \times P \times Q)$, %contains all 
and let 
$d_\bot(q_1, q_2) = d_\top(q_1, p, q_2) = \one$ if $q_1 = q_2$ and
$d_\bot(q_1, q_2) = d_\top(q_1, p, q_2) = \zero$ otherwise.

\smallskip\noindent
\textbf{while} $\Q$ is not empty

\noindent\quad
\textbf{extract} $\< q_1, q_2>$ or $\< q_1, p, q_2>$ from $\Q$ 

\noindent\quad
such that $d_\bot(q_1, q_2)$, resp. $\bigoplus_{c\in \Omegac} d_\top(q_1, p, q_2)(c)$,
is minimal in $\Semiring$ wrt $\leq_\oplus$.

\noindent\quad 
%\textbf{for all} $q_0, q_3 \in Q$ 
\textbf{update} $d_\bot$ with $\< q_1, q_2>$ or $d_\top$ with $\< q_1, p, q_2>$
(Figure~\ref{fig:best-update}).
\end{algo}



\begin{figure}
For all $q_0, q_3 \in Q$, %$p \in P$,
\[
\begin{array}{lcl}
%\multicolumn{3}{l}{\mathrm{For~all~} q_0, q_3 \in Q, p \in P,}\\
d_\top(q_1, p, q_3) & \opluseq &
  d_\top(q_1, p, q_2) \otimes 
  \displaystyle\bigoplus_{\Omegai} \weii(q_2, p, q_3)\\
%\multicolumn{3}{l}{\quad
%\mathrm{where~} \weii(q_2, p, q_3)_a \in \Phic
%\mathrm{~is~the~partial~application~}
%x_c \mapsto \weii(q_2, x_c, p, a, q_3)}\\
%
d_\bot(q_1, p, q_3) & \opluseq &
  d_\bot(q_1, q_2) \otimes 
  \displaystyle\bigoplus_{\Omegai} \weiei(q_2, q_3)\\
%     
d_\top(q_0, p, q_3) & \opluseq &
  {\displaystyle\bigoplus_{\Omegac}}^2 
  \bigl[ \bigl( \weic(q_0, p, p', q_1) \otimes_2 
  d_\top(q_1, p', q_2) \bigr) \otimes_2
  {\displaystyle\bigoplus_{\Omegar}} \weir(q_2, p', q_3) \bigr]\\
%\multicolumn{3}{l}{\quad
%\mathrm{where~} \weic(q_0, p, p', q_1)_{c'} \in \Phic
%\mathrm{~is~the~partial~application~}
%x_c \mapsto \weic(q_0, p, x_c, c', p', q_1)}\\[2pt]
%
d_\bot(q_0, q_3) & \opluseq &
  {\displaystyle\bigoplus_{\Omegac}}
  \bigl(
  \weiec(q_0, p, q_1) \otimes 
   d_\top(q_1, p, q_2) \otimes
  \displaystyle\bigoplus_{\Omegar} \weir(q_2, p, q_3)\bigr)\\
%
d_\bot(q_1, q_3) & \opluseq &
  d_\bot(q_1, q_2) \otimes 
  \displaystyle\bigoplus_{\Omegar} \weier(q_2, q_3)\\
%
d_\top(q_1, p, q_3) & \opluseq & 
  d_\top(q_1, p, q_2) \otimes d_\top(q_2, p, q_3), 
  \mathrm{if~} \< q_2, \top, q_3> \notin P \\
%
d_\bot(q_1, q_3) & \opluseq & 
  d_\bot(q_1, q_2) \otimes d_\bot(q_2, q_3), \mathrm{if~} \< q_2, \bot, q_3> \notin P \\
\end{array}
\]
%$\weii(q_2, p, q_3)_a \in \Phic$, for $a \in \Omegai$,
%is the partial application
%$x_c \mapsto \weii(q_2, x_c, p, a, q_3)$.
%
%$\weic(q_0, p, p', q_1)_{c'} \in \Phic$, for $c' \in \Omegac$,
%is the partial application
%$x_c \mapsto \weic(q_0, p, x_c, c', p', q_1)$.
\caption{Update $d_\bot$ with $\<q_1, q_2>$ or $d_\top$ with $\< q_1, p, q_2>$.} 
\label{fig:best-update}
\end{figure}


\noindent
The infinite sums in the updates of $d$ in Algorithm~\ref{algo:Dijkstra}, 
Figure~\ref{fig:best-update}
are well defined
since~$\Semiring$ is complete.
** effectively computable by hypothese that the label theory is effective**
The algorithm performs $2.|Q|^2$ iterations until $P$ is empty, 
and each iteration has a time complexity $O(|Q|^2 . |P|)$.
This gives a time complexity $O(|Q|^4 . |P|)$. 
It can be reduced by implementing $P$ as a priority queue, 
prioritized by the value returned by $d$
***complete***. %$|Q|^3.\log(|Q|^2)$

The correctness of Algorithm~\ref{algo:Dijkstra} 
is ensured by the invariant expressed in the following lemma.
\begin{lemma}\label{lem:bot}
For all $\< q_1, q_2> \notin \Q$,
$d_\bot(q_1, q_2) =  b_\bot(q_1, q_2)$/
\end{lemma}
The proof is by contradiction, 
assuming a counter-example minimal in the length of the witness word.

\begin{lemma}\label{lem:top}
For all $\< q_1, p, q_2> \notin \Q$, 
$d_\top(q_1, p, q_2) =  b_\top(q_1, p, q_2)$,
\end{lemma}

\noindent
For computing the minimal weight of a computation of $A$, we use the fact that,
at the termination of Algorithm~\ref{algo:Dijkstra}, %it holds that,
%There exist $q_1, q_2 \in Q$
\[
  {\displaystyle \bigoplus_{s \in \Omega^*} A(s)} = 
  {\displaystyle\bigoplus_{q, q' \in Q}} \textstyle
  \mathsf{in}(q) \mathop{\otimes} d_\bot(q, q') \mathop{\otimes} \mathsf{out}(q').
\]

%\medskip
\noindent
In order to obtain effectively a witness 
(word of $\Omega^*$ with a computation of $A$ of minimal weight), 
we require the additional property of convexity of weight functions.

\begin{proposition}\label{th:best-search}
For a \SWVPA $A$ 
over $\Omega$, 
$\Semiring$ commutative, bounded, total and complete, %semiring 
and $\bar\Phi$ effective and $k$-convex, % label theory,
one can construct in PTIME a word $s \in \Omega^*$ 
such that $A(s)$ is minimal \wrt the natural ordering for $\Semiring$. 
\end{proposition} 



\subsection{Nested-Words and Parse-Trees}
\label{sec:trees}
The hierarchical structure of nested-words, defined with the \emph{call} and \emph{return} markup symbols  
suggest a correspondence with trees. 
The lifting of this correspondence to languages, of tree automata and VPA,
has been discussed in~\cite{AlurMadhusudan09nested}, 
and~\cite{Caralp12VPAmult} for the weighted case.
In this section, we describe a correspondence between the symbolic-weighted extensions
of tree automata and VPA.

Let $\Omega$ be a countable ranked alphabet, such that 
every symbol $a \in \Omega$ has a rank 
$\rank(a) \in [0..M]$ where $M$ is a fixed natural number.
We denote by $\Omega_k$ the subset of all symbols $a$ of $\Omega$
with $\rank(a) = k$, where $0 \leq k \leq M$, 
and $\Omega_{>0} = \Omega \setminus \Omega_0$.
%
\noindent 
The free $\Omega$-algebra of finite, ordered, 
$\Omega$-labeled trees is denoted by $\T(\Omega)$.
It is the smallest set such that  $\Omega_0 \subset \T(\Omega)$
and for all $1 \leq k \leq M$, all $a \in \Omega_k$, 
and all $t_1, \ldots, t_k \in \T(\Omega)$, $a(t_1, \ldots, t_k) \in \T(\Omega)$.
%
% tree = single node (leave) labeled with a symbol of $a \in \Omegai$
% (such a tree is simply denoted by $a$)
% or the composition, denoted by $b(t_1,\ldots, t_n$) of a node labeled with $b$
% and $n$ subtrees $t_1$,\ldots, $t_n$.
%
Let us assume a commutative semiring~$\Semiring$ 
and a label theory~$\bar{\Phi}$ over~$\Semiring$ 
containing one set~$\Phi_{\Omega_k}$ for each $k \in [0..M]$.
%
\renewcommand{\call}[1]{\ensuremath \langle_{#1}}
\renewcommand{\return}[1]{\ensuremath {}_{#1}{\rangle}} % $\prescript{}{a}{)}$

% A \emph{regular tree grammar} over $\Omega$ 
% is a triplet $G = \< N, q_\mathsf{i}, R>$ where
% $N$ is a finite set of non-terminal symbols denoted $q$..., 
% $q_\mathsf{i} \in N$ is the starting non-terminal, 
% $R$ is a finite set of production rules of the form
% $q_0 \to a(q_1\ldots q_k)$ where 
% $q_0, q_1, \ldots, q_k \in N$
% $a \in \Omega_k$.
% A tree $t \in \T(\Omega)$ is in the language of $G$ 
% if it can be generated from $q_\mathsf{i}$ by 
% non terminal replacement following the rules of $R$.
%
\begin{definition}  \label{def:SWTA}
A \emph{symbolic-weighted tree automaton} (\SWTA)
over $\Omega$, $\Semiring$, and~$\bar\Phi$
is a triplet $A = \< Q, \init, \bar{\wei} >$ where
$Q$ is a finite set of states, 
$\mathsf{in} : Q \to \Phi_\Omega$ is the starting weight function, 
and $\bar{\wei}$ is a tuplet of transition functions containing, 
for each $k \in [0..M]$, 
%$\wei_\varepsilon$ from $Q \times Q$ into $\Semiring$, and, 
the functions $\wei_{k}: Q \times Q^{k} \to \Phi_{\Omega_{>0},\Omega_k}$
and $\weie[k]: Q \times Q^{k} \to \Phi_{\Omega_k}$.
\end{definition}
%
%Like in Section~\ref{sec:SWAdef}, 
We define %from $\bar{\wei}$ 
a transition 
function~$\wei: Q \times (\Omega_{> 0} \cup \{ \varepsilon \}) \times \Omega \times \bigcup_{k=0}^{M} Q^k 
  \to \Semiring$
by: %also called $\wei$ for simplicity, 
%such that, for all $q, q' \in Q$, $a \in \Sigma$, and $b \in \Delta$, 
\[
\begin{array}{rcll}
%\wei(q_0, \varepsilon, q_1) & = &  \wei_\varepsilon(q_0, q_1),\\ %\phi_\varepsilon\\
\wei(q_0, a, b, q_1 \ldots q_k) & = & \eta(a, b) &
\quad\mathrm{where~} \eta = \wei_{k}(q_0, q_1\ldots q_k)\\
\wei(q_0, \varepsilon, b, q_1 \ldots q_k) & = & \phi(b) &
\quad\mathrm{where~} \phi = \weie[k](q_0, q_1\ldots q_k).
\end{array}      
\]
%
\noindent
where $q_1\ldots q_k$ is $\varepsilon$ if $k = 0$.
The first case deals with a strict subtree, with a parent node labeled by $a$, 
and the second case is for a root tree.

\noindent
Every \SWTA %of Definition~\ref{def:SWTA} 
defines a mapping 
from trees of $\T(\Omega)$ into~$\Semiring$, %the weight values in~$\Semiring$,
based on the following intermediate function
$\weight_A: Q \times (\Omega \cup \{ \varepsilon \}) \times \T(\Omega) \to \Semiring$ 
\begin{equation}
%\begin{array}{rccl}
\weight_A(q_0, a, t) =  % & = &
%   \displaystyle\bigoplus_{q_1 \in Q} &
%   \wei(q, \varepsilon, q_1) \otimes \weight_A(q_1, t)\\
 \displaystyle\bigoplus_{q_1 \ldots q_k \in Q^k} % &
              \wei(q_0, a, b, q_1 \ldots q_k ) 
   \otimes \displaystyle\bigotimes_{i=1}^{k}
           \weight_A(q_i, b, t_i)
%\end{array}
\end{equation}
where $q_0 \in Q$, $a \in \Omega_{>0} \cup \{ \varepsilon \}$ and 
$t = b(t_1,\ldots, t_k) \in \T(\Omega)$,
$0 \leq k \leq M$.

\medskip\noindent
Finally, the weight associated by $A$ to  $t \in \T(\Omega)$ is 
\begin{equation}
A(t)  = 
\displaystyle\bigoplus_{q \in Q} \mathsf{in}(q) \mathop{\otimes} \weight_A(q, \varepsilon, t)
\label{eq:weightTA}
\end{equation}

\noindent
Intuitively, $\wei(q_0, a, b, q_1 \ldots q_k)$ can be seen as
the weight of a production rule $q_0 \to b(q_1, \ldots, q_k)$ 
of a regular tree grammar~\cite{tata}, 
that replaces the non-terminal symbol $q_0$ by $b(q_1, \ldots, q_k)$, 
provided that the parent of $q_0$ is labeled by $a$
(or $q_0$ is the root node if $a = \varepsilon$).
%in a step of tree building.
%
%Such a grammar computes the weights of the derivation trees 
%of the Context-Free grammar obtained by forgetting the labeling symbols of $\Omega_{>0}$.
The above production rule can also be seen as 
a rule of a weighted CF grammar, of the form
$[a, b]\, q_0 := q_1 \ldots q_k$ if $k > 0$,
and $[a]\, q_0 := b$ if $k = 0$. 
In the first case, $b$ is a label of the rule, 
and in the second case, it is a terminal symbol.
And in both cases, $a$ is a constraint on the label of rule applied 
on the parent node in the derivation tree.
This features of observing the parent's label  
are useful in the case of infinite alphabet, 
where it is not possible to memorize a label with the states.
%
%One can observe that 
\noindent The weight of a labeled derivation tree $t$
of the weighted CF grammar associated to~$A$ as above, 
is $\weight_A(q, t)$, 
when $q$ is the start non-terminal.
%
We shall now establish a correspondence between such derivation tree $t$
and some word describing a linearization of $t$, 
in a way that $\weight_A(q, t)$ can be computed by a $\SWVPA$.

Let $\hat\Omega$ be the countable (unranked) alphabet obtained
from $\Omega$ by: 
$\hat\Omega = \Omegai \uplus \Omegac \uplus \Omegar$, with
$\Omegai = \Omega_0$, 
$\Omegac = \{ \; \call{a} \mid a \in \Omega_{>0} \}$,
$\Omegar = \{ \; \return{a} \mid a \in \Omega_{>0} \}$.

\noindent
We associate to $\hat\Omega$
a label theory $\hat{\Phi}$ 
like in Section~\ref{se:SWVPA-def}, 
%
\noindent
and we define a linearization of trees of $\T(\Omega)$ into 
words of $\hat{\Omega}^*$ as follows:
\begin{description}
\item $\lin(a) = a$ for all $a \in \Omega_0$, 
\item $\lin\bigl( b(t_1, \ldots, t_k)\bigr) = 
       \call{b} \; \lin(t_1) \ldots \lin(t_k) \; \return{b}$ 
       when $b \in \Omega_k$ for $1 \leq k \leq M$.
\end{description}

\begin{proposition}\label{lem:SWTA}
For all \SWTA $A$ over~$\Omega$, $\Semiring$ commutative, and $\bar\Phi$,
there exists an effectively constructible \SWVPA $A'$ over 
$\hat\Omega$, $\Semiring$ and $\hat\Phi$ 
such that for all $t \in \T(\Omega)$, $A'\bigl(\lin(t)\bigr) = A(t)$.
\end{proposition} 
% 
\begin{proof}
Let $A = \< Q, \init, \bar{\wei} >$ where $\bar{\wei}$ is presented as above by a function
    %
We build 
$A' = \< Q', P', \init', \bar{\wei}', \final' >$,
%computing over $\hat\Omega = \Omegai \uplus \Omegac \uplus \Omegar$,
where $Q' = \bigcup_{k=0}^{M} Q^k$ is the set of sequences of state symbols of $A$, 
of length at most $M$, including the empty sequence denoted by~$\varepsilon$, 
and where $P' = Q'$ and $\bar\wei$ is defined by:

\[
\begin{array}{lcll}
\weii(q_0\, \bar{u}, {\call{c}}, \bar{p}, a, \bar{u}) & = & \wei(q_0, c, a, \varepsilon) & 
\mathrm{for~all~} c \in \Omega_{>0}, a \in \Omega_0\\ %\bar{p}\in P', 
%
\weiei(q_0\,\bar{u}, a, \bar{u}) & = & \wei(q_0, \varepsilon, a, \varepsilon) & 
\mathrm{for~all~} a \in \Omega_0\\
%
\weic(q_0\,\bar{u}, {\call{c}}, \bar{p}, \call{d}, \bar{u}, \bar{q}) & = & \wei(q_0, c, d, \bar{q}) &  
\mathrm{for~all~} c, d \in \Omega_{>0}\\ % \bar{p}\in P'
%
\weiec(q_0\,\bar{u}, \call{c}, \bar{u}, \bar{q}) & = & \wei(q_0, \varepsilon, c, \bar{q}) & 
\mathrm{for~all~} c \in \Omega_{>0}\\
%
%\weir: Q \times \Omegac \times P \times \Sigmar \times Q \to \Semiring & 
\weir(\varepsilon, {\call{c}}, \bar{p}, {\return{c}}, \bar{p}) & = & \one & 
\mathrm{for~all~}  c \in \Omega_{>0}\\ % \bar{p}\in P'
%
%\weie: Q \times \Omegar \times Q \to \Semiring & 
\weier(\bar{u}, {\return{c}}, \bar{q}) & = & \zero &
\mathrm{for~all~}  c \in \Omega_{>0}
\end{array}      
\]
\noindent
All cases not matched by one of the above equations have a weight $\zero$, 
for instance  %This is in particular the case of
$\weir(\bar{u}, {\call{c}}, \bar{p}, {\return{d}}, \bar{q}) = \zero$
if $c \neq d$
or $\bar{u} \neq \varepsilon$
or $\bar{q} \neq \bar{p}$.
%
%\noindent
%It is sufficient to consider in $Q'$ only the prefixes of 
%sequences in transition with a non-null weight.
\qed
\end{proof}


\section{Symbolic Weighted Parsing}
\label{sec:parsing}
Let us now apply the models and results of the previous sections %in order to define 
to the problem of parsing over infinite alphabet. %appropriate
%
%\subsection{Definition}
%
Let~$\Sigma$ be a countable input alphabet 
and~$\Omega$ be a countable output ranked alphabet, 
with maximal rank value~$M$, 
and let $\hat\Omega = \Omegai \uplus \Omegac \uplus \Omegar$
be the alphabet with nesting symbols associated to $\Omega$ like in Section~\ref{sec:trees},
for the linearization of trees of $\T(\Omega)$.
%(remember that $\Omegai = \Omega_0$).
%
Let $\< \Semiring, \oplus, \zero, \otimes, \one>$ be a 
commutative, bounded, and complete \marginpar{\tiny total?} semiring  
and let $\bar\Phi$ be an effective label theory over $\Semiring$,
containing $\Phi_\Sigma$, $\Phi_{\Sigma, \Omegai}$, as well as
$\Phii$, $\Phic$, $\Phir$, $\Phicr$
(following the notations of Section~\ref{se:SWVPA-def}).
%It is moreover assumed computable and $k$-convex for some fixed~$k$.
%
\noindent
We assume given the following input:
\begin{description}
\item[--] a \SWT $T$ over $\Sigma$, $\Omegai$, $\Semiring$, and $\bar\Phi$, 
defining a measure %between words 
$T: \Sigma^* \times \Omegai^* \to \Semiring$,

\item[--] a \SWTA $A$ over $\Omega$, $\Semiring$, and $\bar\Phi$,
defining a measure $A: \T(\Omega) \to \Semiring$,
%a \SWVPA $A$ over $\Omega$, and $\Semiring$, defining a series of nested words
%      $A : \Omega^* \to \Semiring$,
\item[--] an input word $s \in \Sigma^*$.
\end{description}
%
As explained in Section~\ref{sec:trees}, 
$A = \< Q, \init, \bar{\wei} >$ can be seen as a weighted regular tree grammar, 
or a labeled weighted CF grammar whose derivation trees are in $\T(\Omega)$, 
with weight defined by $A$.
The purpose of the transducer $T$ is to measure a distance between input and output words, 
\ie between the word $s$ and the sequence leaves of a derivation tree, in $\Omegai^*$.
%that generates (weighted) trees by replacement of a state symbol~$q_0$ (non-terminal), 
%by a tree $a(q_1,\ldots, q_k)$, where $k = \rank(a)$.
%A replacement rule $q_0 \to a(q_1,\ldots, q_k)$, 
%of weight $\wei(q_0, a, q_1 \ldots q_k) \in \Semiring$ 
%according to Definition~\ref{def:SWTA}, 
%corresponds to the production rule $q_0 := a(q_1,\ldots, q_k)$ of a weighted CF grammar,
%with set non-terminal symbols $Q$ and set of terminal symbols $\Omega_0$.
%
%This actually is a slight generalization of CFG since 
%each such production rule is labelled by  a symbol of $\Omega_{>0}$, 
%hence parse trees %derivation trees
%are trees of $\T(\Omega)$.
%
%Another (more original) generalization is that the set of terminal symbols 
%$\Omega_0$ may be infinite.

\noindent 
We extend the measure defined by $T$ to 
$d : \Sigma^* \times \T(\Omega) \to \Semiring$ as follows.
Given a word $w \in {\hat\Omega}^*$, the projection of $w$ onto $\Omegai$,
denoted $w|_\Omegai$,
is the word of $\Omegai^*$ obtained from $w$ by removing all symbols 
in $\hat\Omega \setminus \Omegai$.
Using this notation
and the tree linearization operator defined in Section~\ref{sec:trees}, 
$d$ is defined by:
\begin{equation}
d(s, t) = T\bigl(s, \lin(t)|_\Omegai \bigr)  \mathrm{~for~} s \in \Sigma^*, t\in \T(\Omega)
\end{equation}

\noindent 
\emph{Symbolic weighted parsing} is the problem, 
given the above input, 
to find a tree $t \in \T(\Omega)$ %nested word $t \in \Omega$ 
minimizing \( d(s, t) \otimes A(t)\)
\wrt $\leq_\oplus$, 
\ie such that: %Hence, it is the problem of finding 
%
\begin{equation}\label{eq:distance-lang}
d(s, t) \otimes A(t) = \displaystyle\bigoplus_{t' \in \T(\Omega)} d(s, t') \otimes A(t') 
\end{equation}
%
The measure expressed in~\eqref{eq:distance-lang} is called
called the edit-distance between~$s$ and~$A$ in~\cite{Mohri03EDWA}.
%
%The input language can also be expressed as a \SWTA, or, 
%as a particular case, as a weighted context-free grammar, 
%converted in turn into a \SWVPA following Lemma~\ref{lem:SWTA}.
%
The problem of searching, in a WTA language, 
the best parse tree matching a given input,  
sometimes referred as~\emph{weighted parsing},
corresponds to SW parsing in the case of finite alphabets
and when the transducer $T$ characterizes identity
see \eg~\cite{Goodman99SemiringParsing} 
and~\cite{MorbitzVogler19weighted-parsing} for a more general 
weighted parsing framework.
%
%Indeed, it corresponds to the case where $T$
%accepts only the pairs $\<s, t>$ such that 
%$s$ is the projection of $t$ on $\Omegai$. 
%This can be done with a single state $q$ and 
%with transition rules of the form:
%\begin{description}
%\item[] $\wei(q, \varepsilon, a, q) = \one$ for all $a \in \Omegac \cup \Omegar$,
%\item[] $\wei(q, a, a, q) = \one$ for all $a \in \Omegai$,
%\item[] $\wei(q, a, b, q) = \zero$ for all $a, b \in \Omegai$, $a \neq b$.
%\end{description}


%\subsection{Computation}
%
\begin{proposition}
The problem of Symbolic Weighted  parsing 
can be solved in PTIME in the size of the input \SWT $T$, \SWTA $A$ %\SWVPA (or ) 
and input word $s$, 
and the computation time of the functions of the label theory.
\end{proposition}
\begin{proof} (sketch)
We follow a \emph{Bar-Hillel} construction, also called parsing by intersection.

\noindent
We first extend the \SWT $T$ over $\Sigma$, $\Omegai$, $\Semiring$, and $\bar\Phi$, 
into a \SWT $T'$ over $\Sigma$ and $\hat\Omega$ (and the same semiring and label theory),
such that for all $s \in \Sigma^*$, and $u \in {\hat\Omega}^*$, 
$T'(s, u) = T(s, u|_{\Omegai})$. The transducer $T'$ simply skips every symbol 
$b \in {\hat\Omega} \setminus \Omegai$, 
by the addition to the transition of $T$,
of new transitions of the form $\wei_{01}(q, \varepsilon, b, q')$.

\noindent
Then, given an input word $s \in \Sigma^*$, 
we compute the \SWA $A_{T', s}$, using Proposition~\ref{prop:epsilon}.
This automaton is such that for all $t \in \T(\Omega)$, 
\[ 
   A_{T', s}\bigl(\lin(t)\bigr) 
 = T'\bigl(s, \lin(t)\bigr) 
 = T'\bigl(s, \lin(t)|_{\Omegai}\bigr) 
 = d(s, t).
\]

\noindent
Next, we convert the input \SWTA $A$ over $\Omega$
into a \SWVPA $A'$ over $\hat\Omega$, using Lemma~\ref{lem:SWTA}, 
and we compute the \SWVPA $A_{T', s} \otimes A'$, 
using Proposition~\ref{prop:SWVPA-product}.

\noindent
It remains to compute a best nested-word $w \in {\hat\Omega}^*$ 
using the best-search procedure of Proposition~\ref{th:best-search},
and convert it into a best tree in $\T(\Omega)$ in order to solve SW parsing
for~$T$, $A$ and~$s$.
\qed  
\end{proof}

\paragraph{Application to Automated Music Transcription.}
%



\subsection{Application to Automated Music Transcription}
\label{sec:transcription}
Symbolic Automated Music Transcription
and analysis of music performances

\subsubsection{Time Scales}
Real-Time Unit (RTU) = seconds

\noindent 
Musical-Time Unit (MTU) = number of measures

\noindent 
conversion via tempo value

\subsubsection{Representation of Music Performances}
We consider symbolic representations of musical performances, as finite sequences of events.
It corresponds to the concrete case of a MIDI file~\cite{SMF} 
recorded  from an electronic keyboard, 
or the output of a transcription from audio files~\cite{Benetos18AMTsurvey}.
%
For the sake of simplicity, 
we shall only consider here the case of monophonic performances, 
where at most one note is sounding at a time. 
The approach however extends to the polyphonic case.

A music performance is a finite sequence of events in a set~$\Sigma$.
Every event $e \in \Sigma$ has attributes such from a finite domain, 
like a number of key for a note 
or a flag indicating that it is a rest 
(\textsf{ON} and \textsf{OFF} messages in~\cite{SMF})
and a velocity value (0..127 in~\cite{SMF})).
%This representation is similar to the piano roll ~\cite{Muller15fundamentals} chap.1. 
Moreover, it contains a RTU value $\ioi{e}$ (real number) 
representing the time distance to the next event, 
or to the end of performance for the last event,
also called \emph{inter-onset interval}.


\subsubsection{Representation of Music Scores}
Music score are represented as structured words
made of timed %quantified 
events and parenthesized markups,
akin of nested words~\cite{AlurMadhusudan09nested}.

We consider an alphabet $\Delta$, every symbol of which is 
composed of a tag, in a finite set $\Xi$, 
and an MTU (rational) IOI duration value.
%The alphabet $\Delta$ 
It is partitioned into 
$\Delta = \Deltai \uplus \Deltac \uplus \Deltar$, 
like in Section~\ref{section:SWVPA}.
%
\noindent
The symbols of $\Deltai$ represent events:
% (infinite alphabet of internal symbols) made of:
with tags indicating a new note or grace-note (with null IOI), 
a rest or the continuation of the previous note (tie or dot).
%
The elements of $\Deltac \uplus \Deltar$ are matched
markups for describing the structure of the score, 
\ie the hierarchical grouping of events, and also, 
importantly the division of time in measures, tuplets...
%- parentheses for time divisions : tuplets, bars...
(linearization of rhythm trees \cite{jacquemard:hal-01138642}...).
They contain additional info such as tuple number, beaming policy...

\noindent
The duration values of letters of $\Delta$, in MTU (rational), 
can be computed with the markups and tags (\eg grace note has duration 0).

%\noindent
%There are simultaneous events, since grace notes has duration 0. They are ordered.
%
%\noindent
%Finite bound on the number of duration ratio. ?

\begin{example}
...      
\end{example}

\subsubsection{Performance/Score Distance Computation}
\label{app:distance}
We define a distance between performance and score representations
by a $\SWT$ $T = \< Q, \init, \wei, \final >$, over a semiring $\Semiring$.
** detail the elements of $\Semiring$ ....**
%are quadruplets of the form
%$\< t, s, \delta_t, \delta_s>$

Every state of $Q$ contains a 
tempo value in a finite domain (e.g. 30..300 bpm).
This value can be fixed 
or recomputed by the $T$ %transducer 
after reading each event, 
according to a perceptive/cognitive model of tempo 
such as~\cite{LargeJones99tempo}
(also used in the context of score following~\cite{Cont10TPAMI}).
% we wont detail here.


\subsection{Transcription by SW Parsing} %Best-first Search}
We assume a score language defined by a \SWVPA over the semiring 
$\Semiring$ of Section~\ref{app:distance}.


...



\section*{Conclusion}
% summary 
We have introduced weighted language models (SW transducers and visibly pushdown automata)
computing over infinite alphabets, 
and applied them to the problem of parsing 
with infinitely many possible input symbols (typically timed events).
%handled with suitable language formalisms, 
%
This approach extends conventional parsing and weighted parsing
by computing a derivation tree modulo 
a generic distance between words,  
defined by a SW transducer given in input.
This enables to consider finer word relationships than strict equality, 
%as in the conventional parsing approach, 
opening possibilities of quantitative analysis via this method.

% discussion
\noindent
Ongoing and future work include 

\noindent
-- The study of other theoretical properties of SW models, 
such as the extension of the best search algorithm from $1$-best to $n$-best~\cite{Huang05kbest}, 
and to $k$-\emph{closed} semirings~\cite{Mohri02semiring}
(instead of \emph{bounded}, which corresponds to $0$-\emph{closed}).

\noindent
-- ...there is room to improve the complexity bounds for the algorithms
... modular approach with oracles ...

\noindent
-- present here an offline algorithm for best search, 
semi-online implementation for AMT (bar-by-bar approach)
with an on-the-fly automata construction.




%%%%%%%%%%%%%%%%%%%%%%%%%%%%%%%%%%%%%%%%%%%%%%%%%%%%%%%%%%%%%%%%%%%%%%%%%%%%%%%%
%% BIBLIO                                                                     %%
%%%%%%%%%%%%%%%%%%%%%%%%%%%%%%%%%%%%%%%%%%%%%%%%%%%%%%%%%%%%%%%%%%%%%%%%%%%%%%%%
%\bibliographystyle{plain}
%\bibliographystyle{plainurl} 
\bibliographystyle{abbrv}
%\bibliographystyle{splncs04}
%\bibliographystyle{eptcs}
%\bibliography{generic}

\bibliography{references}




%%%%%%%%%%%%%%%%%%%%%%%%%%%%%%%%%%%%%%%%%%%%%%%%%%%%%%%%%%%%%%%%%%%%%%%%%%%%%%%%
%% APPENDIX                                                                   %%
%%%%%%%%%%%%%%%%%%%%%%%%%%%%%%%%%%%%%%%%%%%%%%%%%%%%%%%%%%%%%%%%%%%%%%%%%%%%%%%%
\newpage
\appendix 




\end{document}



\section{Edit-Distance}

%\subsection{Distance between words or languages}
...algebraic definition of edit-distance of Mohri, in \cite{Mohri03EDWA}
% Mehryar Mohri 
% Edit-distance of weighted automata: General definitions and algorithms
% International Journal of Foundations of Computer Science 14.06 (2003): 957-982.
distance $d$ over $\Sigma^* \times \Sigma^*$ 
into a semiring  $\Semiring = ( \Semiring, \oplus, \zero, \otimes, \one)$.

%\noindent
Let $\Omega = \Sigma \cup \{ \varepsilon \} \times \Sigma \cup \{ \varepsilon \} \setminus \{ (\varepsilon, \varepsilon) \}$,
and let $h$ be the morphism from $\Omega^*$ into $\Sigma^* \times \Sigma^*$  
defined over the concatenation of strings of $\Sigma^*$ (that removes the $\varepsilon$'s).
%
\noindent
An \emph{alignment} between 2 strings  $s, t \in \Sigma^*$ is an element $\omega \in \Omega^*$ 
such that $h(\omega) = (s, t)$.
%
\noindent
We assume a base cost function $\Omega$ : $\delta: \Omega \to S$, extended to $\Omega^*$ as follows  
(for $\omega \in \Omega^*$): 
\(
\displaystyle\delta(\omega) = \bigotimes_{0 \leq i < |\omega|} \delta(\omega_i)
\).

\noindent
\begin{definition}
For  $s, t \in \Sigma^*$, the edit-distance between $s$ and $t$ is  
\( 
d(s, t) = \displaystyle\bigoplus_{\omega \in \Omega^*\, h(\omega) = (s, t)} \delta(\omega)
\).
\end{definition}

e.g. Levenstein edit-distance: $S$ is min-plus and $\delta(a, b) = 1$ for all $(a, b) \in \Omega$.


%\paragraph{Distance between a word and a regular language}
