% !TEX root = main.tex
%
% Proof of Proposition closure swA by swT
%
Let us show that 
$B_{A, T}(t) = \displaystyle\bigoplus_{s\in \Sigma^*} A(s) \otimes T(s, t)$
for all $t\in \Delta^+$.

\noindent
We call \emph{run} from $\< p_0, q_0>$ to $\< p_n, q_n>$ 
a finite sequence of the form:
\florent{revise run with seq. of transitions}
$\rho = \< p_0, q_0>$, $a_1, \< p_1, q_1>$,\ldots, $a_n, \< p_n, q_n>$
where $\< p_i, q_i> \in Q'$ for all $0 \leq i \leq n$
and $a_j \in \Sigma$ for all $1 \leq j \leq n$.
The state $\< p_0, q_0>$ is called source of the run, denoted $\mathit{src}(\rho)$
and $\< p_n, q_n>$ is called target of the run, denoted $\mathit{trg}(\rho)$;
the set of runs with source $\< p, q>$ and target $\< p', q'>$ 
is denoted $\Rho(\< p, q>, \< p', q'>)$.
The word $a_1\ldots a_n \in \Sigma^*$ is called word of the run $\rho$
and denoted $\word(\rho)$.
%The set of all runs
Moreover, we associate a weight value in $\Semiring$ to every run, 
defined by:
\begin{equation} \label{def:weight-run}
\weight(\rho) = \bigotimes_{i = 1}^{n} 
\wei_1(p_{i-1}, a_i, p_i) \otimes
\wei_{10}(q_{i-1}, a_i, \varepsilon, q_i)
\end{equation}

By definition of the weight functions, and 
associativity, commutativity, and distributivity of $\oplus$, $\otimes$,  
it holds that:
\begin{equation}\label{eq:weight-run}
\weight_A(p, s, p') \otimes \weight_T(q, s, \varepsilon, q') = 
\bigoplus_{\begin{array}{c}
		   \scriptstyle\rho \in \Rho(\< p, q>, \< p', q'>)\\[-1ex]
		   %\scriptstyle\mathit{src}(\rho) = \<p , q>\\[-1ex]
		   %\scriptstyle\mathit{trg}(\rho) = \<p' , q'>\\[-1ex]
		   \scriptstyle\word(\rho) = s
		   \end{array}} 
\weight(\rho) 
\end{equation}
%
Using \eqref{def:weight-run} 
and Lemma~\ref{lem:label-th} repetively, 
\eqref{eq:weight-run} implies that:
%
\begin{equation}\label{eq:weight-run-plus}
\begin{array}{l}
\displaystyle\bigoplus_{s \in \Sigma^*} 
  \weight_A(p, s, p') \otimes \weight_T(q, s, \varepsilon, q') =\\
\multicolumn{1}{r}{%
\qquad
\displaystyle\bigoplus_{\begin{array}{c}
		   \scriptstyle\rho \in \Rho(\< p, q>, \< p', q'>)\\[-1ex]
		   \scriptstyle\rho = \< p_0, q_0>, a_1,\< p_1, q_1>,\ldots, a_n, \< p_n, q_n>\\[-1ex]
%		   \scriptstyle\rho = \< p_0, q_0>, a_1,\\[-1ex]
%		   \scriptstyle\< p_1, q_1>,\ldots,\\[-1ex]
%       	   \scriptstyle a_n, \< p_n, q_n>\\[-1ex]
%		   \scriptstyle\< p_0, q_0> = \< p, q>\\[-1ex]
%		   \scriptstyle\< p_n, q_n>  = \< p', q'>	   
		   \end{array}} 
\displaystyle\bigotimes_{i = 1}^{n} 
\displaystyle\bigoplus_\Sigma (\wei_1(p_{i-1}, p_i) \otimes
                  \wei_{10}(q_{i-1}, q_i))
}
\end{array}           %\nonumber
\end{equation}
%
Note that the symbols $a_1, \ldots, a_n \in \Sigma$ in the run $\rho$
are not significant in \eqref{eq:weight-run-plus}.
%
Using a pumping argument, we can show that~\eqref{eq:weight-run-plus}
still holds when restricting $\rho$ to the set $\Rho_0(\< p, q>, \< p', q'>)$
of runs without repetition in the state symbols.
%
Indeed, assume that in 
$\rho = \< p_0, q_0>, a_1,\< p_1, q_1>,\ldots, a_n, \< p_n, q_n>$, 
$\< p_{i_1}, q_{i_1}> = \< p_{i_2}, q_{i_2}>$
for $0 \leq i_1 < i_2 \leq n$.
Then 
$\rho' = 
 \< p_0, q_0>, \ldots, 
 a_{i_1-1},\< p_{i_1-1}, q_{i_1-1}>,
 a_{i_2},\< p_{i_2}, q_{i_2}>, \ldots,  a_n, \< p_n, q_n>$ 
also belongs to   $\Rho\bigl(\< p_0, q_0>, \< p_n, q_n>\bigr)$
and yields a smaller expression (\wrt $\leq_\oplus$) 
in the right-hand-side of \eqref{eq:weight-run-plus}
than~$\rho$.
It follows, by~\eqref{eq:init-wei} and \eqref{eq:iter-wei}, that
for all $b \in \Delta$, 
%
\begin{equation}\label{eq:weiprime1}
%\begin{array}{rcl}
\wei'_1(\< p, q>, b, \< p', q'>) = 
\displaystyle\bigoplus_{s \in \Sigma^*} 
\displaystyle\bigoplus_{\begin{array}{c}
                        \scriptstyle p'' \in P\\[-1ex]
                        \scriptstyle q'' \in Q
                        \end{array}}
\weight_A(p, s, p'') \otimes \weight_T(q, s, \varepsilon, q'') \otimes \psi_1(b)
%\weight_{B_{A, T}}(\< p, q>, s, \< p', q'>)
%\end{array}
\end{equation}
where 
$\psi_1 = \wei_{01}(q'', q') \oplus
\displaystyle{\bigoplus}_\Sigma^1 \bigl(\wei_1(p'', p') 
 \otimes_1 \wei_{11}(q'', q')\bigr)$.
%
% lift to $A(s) \otimes T(s, \varepsilon)$

\medskip
\noindent
We show now by induction on the length of $t \in \Delta^+$, that
\[
\weight_{B_{A, T}}(\<p, q>, t, \<p', q'>) = 
 \displaystyle\bigoplus_{s\in \Sigma^*}
 \weight_{A}(p, s, p') \otimes  \weight_{T}(q, s, t, q')
\]
This permits to conclude, using 
the definition of $\init'$ in~\eqref{eq:initprime},
%and a proof for 
and the definition of $\final'$ 
in~\eqref{eq:init-final}, and~\eqref{eq:iter-final}. 
%$B_{A, T}(t) = \displaystyle\bigoplus_{s\in \Sigma^*} A(s) \otimes T(s, t)$

\noindent
The base case $t \in \Delta$ follows from~\eqref{eq:weiprime1} and 
the distributivity of $\otimes$.

\noindent
For $t = b u$, with $b \in \Delta$ and $u \in \Delta^*$,
by definition of $\weight_A$ and $\weight_T$,
it holds that for all $s \in \Sigma^*$:
\[
\begin{array}{l}
\displaystyle\weight_A(p, s, p') \otimes \weight_T(q, s, t, q') = 
\bigoplus_{s = s_1 s_2}
\displaystyle\bigoplus_{\begin{array}{c}
                        \scriptstyle p'', p''' \in P\\[-1ex]
                        \scriptstyle q'', q''' \in Q
                        \end{array}}
\weight_A(p, s_1, p''') \otimes \weight_A(p''', s_2, p'') \mathop\otimes \\
\hfill\displaystyle
\weight_T(q, s_1, \varepsilon, q'') \otimes
 \left(
 \begin{array}{l}
 \displaystyle
 \bigoplus_{q''' \in Q}
 \wei_{01}(q'', \varepsilon, b, q''') \otimes \weight_T(q''', s_2, u, q')
 \mathop\oplus\\
 \displaystyle
 \bigoplus_{q''' \in Q} \bigoplus_{s_2 = a s'_2}
 \wei_{11}(q'', a, b, q''') \otimes \weight_T(q''', s'_2, u, q')
 \end{array}
 \right)
\end{array}
\]

\noindent
Using~\eqref{eq:weiprime1}, it follows that:
\[
\begin{array}{l}
\displaystyle\bigoplus_{s \in \Sigma^*} 
\displaystyle\weight_A(p, s, p') \otimes \weight_T(q, s, t, q') = \\
\qquad
\displaystyle\bigoplus_{\begin{array}{c}
                        \scriptstyle p'', p''' \in P\\[-1ex]
                        \scriptstyle q'', q''' \in Q
                        \end{array}}
\bigoplus_{s_1 \in \Sigma^*}
\weight_A(p, s_1, p'') \otimes \weight_T(q, s_1, \varepsilon, q'') \otimes \psi_1(b)\\
%\wei'_1(\< p, q>, b, \< p'', q''>) \mathop\otimes \\
\qquad
\otimes\displaystyle\bigoplus_{\scriptstyle s_2 \in \Sigma^*}
\weight_A(p''', s_2, p') \otimes \weight_T(q''', s_2, u, q')
%\displaystyle\bigoplus_{\begin{array}{c}
%                        \scriptstyle s_1 \in \Sigma^*\\[-1ex]
%                        \scriptstyle s_2 \in \Sigma^*
%                        \end{array}}
\end{array}
\]
with 
$\psi_1 = \wei_{01}(q'', q''') \oplus
\displaystyle{\bigoplus}_\Sigma^1 \bigl(\wei_1(p'', p''') 
 \otimes_1 \wei_{11}(q'', q''')\bigr)$.

\noindent
The first term in the right-hand-side is 
$\wei'_1(\< p, q>, b, \< p''', q'''>)$ by \eqref{eq:weiprime1},
and the second term is
$\weight_{B_{A,T}}(\< p''', q'''>, u, \< p', q'>)$ by induction hypothesis.
Hence, by definition, 
\[
\begin{array}{rcl}
\displaystyle\bigoplus_{s \in \Sigma^*} 
\displaystyle\weight_A(p, s, p') \otimes \weight_T(q, s, t, q') & = & \\
\multicolumn{3}{r}{%
\displaystyle\bigoplus_{\begin{array}{c}
                        \scriptstyle p''' \in P\\[-1ex]
                        \scriptstyle q''' \in Q
                        \end{array}}
 \wei'_1(\< p, q>, b, \< p''', q'''>) \otimes
 \weight_{B_{A,T}}(\< p''', q'''>, u, \< p', q'>)}\\
 & = & \weight_{B_{A,T}}(\< p, q>, t, \< p', q'>).\\
\end{array}
\]




%
%
% old version
%
%\noindent
%The entering, leaving and transition functions of $B_{A, T}$ will
%simulate synchronized computations of $A$ and $T$,
%while reading an output word of $\Delta^*$.
%%
%Its state entering functions is defined
%for all $\<p_1, q_1> \in Q'$, %$p_1 \in P$, $q_1 \in Q$
%by $\init'(p_1, q_1) = \init_A(p_1) \otimes \init_T(q_1)$.
%%
%The transition function $\wei'_1$ will roughly perform
%a synchronized product of transitions defined by $\wei_1$,
%$\wei_{01}$ ($T$ reading in output word and not an input word)
%and $\wei_{11}$ ($T$ reading both an input word and an output word).
%%
%Moreover, $\wei'_1$ also needs to simulate transitions
%defined by $\wei_{10}$: $T$ reading in input word and not an output word.
%Since $B_{A, T}$  will read only in the output word, such a transition corresponds
%to an $\varepsilon$-transition of $\SWA$.
%But $\SWA$ have been defined without $\varepsilon$-transitions.
%Therefore, in order to take care of this case, we perform an on-the-fly
%suppression of $\varepsilon$-transition in the $\SWA$ in construction,
%following the algorithm of~\cite{LombardySakarovitch12ciaa}.
%%
%% Initialize state leaving functions i
%%and $\final'(p, q) = \final_A(p) \otimes \final_T(q)$.
%
%%Every transition of $B_{A, T}$ will
%%simulate a sequence of transitions of $T$ performing the following steps:
%%advance in the input word while staying immobile in the output word,
%%and then make one step in the output word (and advance in the input word or not).
%
%\noindent
%Initially, for all $p_1, p_2 \in P$, and $q_1, q_2 \in Q$, let
%\[
%\begin{array}{rcl}
%\wei'_1\bigl( \< p_1, q_1>, \< p_2, q_2>\bigr) & = &
%\wei_1(p_1, p_2) \otimes
%\bigl[
%\wei_{01}(q_1, q_2)
%\oplus
%\displaystyle\bigoplus_{\Sigma}
%\wei_{11}(q_1, q_2)
%\bigr]\\
%\final'(p_1, q_1) & = & \final_A(p_1) \otimes \final_T(q_1)
%\end{array}
%\]
%
%\noindent
%Then, we iterate the following for all $p_1\in P$ and $q_1, q_2 \in Q$:
%for all $p_2\in P$ and $q_3 \in Q$,
%\[
%\begin{array}{rcl}
%\wei'_1\bigl( \< p_1, q_1>, \< p_2, q_3>\bigr) & \opluseq &
%\displaystyle\bigoplus_{\Sigma} \wei_{10}(q_1, q_2)
%\otimes
%\wei'_1\bigl( \< p_1, q_2>, \< p_2, q_3>\bigr)\\
%\final'(p_1, q_1) & \opluseq &
%\displaystyle\bigoplus_{\Sigma} \wei_{10}(q_1, q_2)
%\otimes \final'(p_1, q_2)
%\end{array}
%\]