



\subsection{Application to Automated Music Transcription}
\label{sec:transcription}
Symbolic Automated Music Transcription
and analysis of music performances

\subsubsection{Time Scales}
Real-Time Unit (RTU) = seconds

\noindent 
Musical-Time Unit (MTU) = number of measures

\noindent 
conversion via tempo value

\subsubsection{Representation of Music Performances}
We consider symbolic representations of musical performances, as finite sequences of events.
It corresponds to the concrete case of a MIDI file~\cite{SMF} 
recorded  from an electronic keyboard, 
or the output of a transcription from audio files~\cite{Benetos18AMTsurvey}.
%
For the sake of simplicity, 
we shall only consider here the case of monophonic performances, 
where at most one note is sounding at a time. 
The approach however extends to the polyphonic case.

A music performance is a finite sequence of events in a set~$\Sigma$.
Every event $e \in \Sigma$ has attributes such from a finite domain, 
like a number of key for a note 
or a flag indicating that it is a rest 
(\textsf{ON} and \textsf{OFF} messages in~\cite{SMF})
and a velocity value (0..127 in~\cite{SMF})).
%This representation is similar to the piano roll ~\cite{Muller15fundamentals} chap.1. 
Moreover, it contains a RTU value $\ioi{e}$ (real number) 
representing the time distance to the next event, 
or to the end of performance for the last event,
also called \emph{inter-onset interval}.


\subsubsection{Representation of Music Scores}
Music score are represented as structured words
made of timed %quantified 
events and parenthesized markups,
akin of nested words~\cite{AlurMadhusudan09nested}.

We consider an alphabet $\Delta$, every symbol of which is 
composed of a tag, in a finite set $\Xi$, 
and an MTU (rational) IOI duration value.
%The alphabet $\Delta$ 
It is partitioned into 
$\Delta = \Deltai \uplus \Deltac \uplus \Deltar$, 
like in Section~\ref{section:SWVPA}.
%
\noindent
The symbols of $\Deltai$ represent events:
% (infinite alphabet of internal symbols) made of:
with tags indicating a new note or grace-note (with null IOI), 
a rest or the continuation of the previous note (tie or dot).
%
The elements of $\Deltac \uplus \Deltar$ are matched
markups for describing the structure of the score, 
\ie the hierarchical grouping of events, and also, 
importantly the division of time in measures, tuplets...
%- parentheses for time divisions : tuplets, bars...
(linearization of rhythm trees \cite{jacquemard:hal-01138642}...).
They contain additional info such as tuple number, beaming policy...

\noindent
The duration values of letters of $\Delta$, in MTU (rational), 
can be computed with the markups and tags (\eg grace note has duration 0).

%\noindent
%There are simultaneous events, since grace notes has duration 0. They are ordered.
%
%\noindent
%Finite bound on the number of duration ratio. ?

\begin{example}
...      
\end{example}

\subsubsection{Performance/Score Distance Computation}
\label{app:distance}
We define a distance between performance and score representations
by a $\SWT$ $T = \< Q, \init, \wei, \final >$, over a semiring $\Semiring$.
** detail the elements of $\Semiring$ ....**
%are quadruplets of the form
%$\< t, s, \delta_t, \delta_s>$

Every state of $Q$ contains a 
tempo value in a finite domain (e.g. 30..300 bpm).
This value can be fixed 
or recomputed by the $T$ %transducer 
after reading each event, 
according to a perceptive/cognitive model of tempo 
such as~\cite{LargeJones99tempo}
(also used in the context of score following~\cite{Cont10TPAMI}).
% we wont detail here.


\subsection{Transcription by SW Parsing} %Best-first Search}
We assume a score language defined by a \SWVPA over the semiring 
$\Semiring$ of Section~\ref{app:distance}.

