% !TEX root = main.tex
%
% initial def. of weight for SW-VPA (before revision)


Formally, the computations of the automaton~$A$ are defined
with an intermediate function $\weight_A$, like in Section~\ref{sec:SWT}.
%
A configuration $q[\gamma]$
is composed of a state $q \in Q$
and a stack content $\gamma \in \Gamma^*$,
where $\Gamma = \Deltac \times P$.
Hence, $\weight_A$ is a function from
$[Q \times \Gamma^*] \times \Delta^* \times [Q \times \Gamma^*]$ into~$\Semiring$.
The empty stack is denoted by $\bot$, and the topmost
symbol is the last pushed pair.
%
The recursive definition of $\weight_A$
enumerates each of the six possible cases:
reading $a \in \Deltai$,
%\lydia{intro to func}
or $\call{c} \in \Deltac$, or $\return{r} \in \Deltar$,
for each possible state of the stack (empty or not).
%to add to $u \in {\Delta}^*$.
%\lydia{introduced the 6 cases}
\florent{shorter notation $cp$ for $\< c, p>$ ?}

\begin{align}
%\weight_A\bigl(\config{q}{\bot}, \varepsilon, \config{q}{\bot}) & = \one\label{eq:SWVPA-weight}\\
%\weight_A\bigl(\config{q}{\bot}, \varepsilon, \config{q'}{\gamma}) & = \zero
%\mathrm{~if~} q \neq q'\nonumber\\
%
\weight_A\bigl(\config{q}{\gamma}, \varepsilon, \config{q'}{\gamma'}) & = \one
\mathrm{~if~} q = q', \gamma =\gamma' = \bot
\mathrm{~and~} \zero \mathrm{~otherwise}\label{eq:SWVPA-weight}\\
\weight_A\bigl(\configup{q}{\<{\call{c}}, p>\stackup\gamma}, a\, u,
               \config{q'}{\gamma'}\bigr) & =
 {\displaystyle\bigoplus_{q'' \in Q}} \weii(q, c, p, a, q'')
  \otimes \weight_A\bigl(\configup{q''}{\<{\call{c}}, p> \stackup \gamma}, u,
                         \config{q'}{\gamma'}\bigr)\nonumber\\
%
\weight_A\bigl(\config{q}{\bot}, a\, u,
               \config{q'}{\gamma'}\bigr) & =
  {\displaystyle\bigoplus_{q'' \in Q}} \weiei(q, a, q'')
   \otimes \weight_A\bigl(\config{q''}{\bot}, u, \config{q'}{\gamma'}\bigr)\nonumber\\
%
\weight_A\bigl(\configup{q}{\<{\call{c}}, p>\stackup\gamma}, {\call{c}'}\, u,
               \config{q'}{\gamma'}\bigr) & =
 {\displaystyle\bigoplus_{\begin{array}{c}
                          \scriptstyle q'' \in Q\\[-1ex]
                          \scriptstyle p' \in P
                          \end{array}}}
 \weic\bigl(q, {\call{c}}, p, {\call{c}'}, p', q''\bigr)
 \otimes \weight_A\bigl(\configup{q''}{\<{\call{c}'}, p'>\stackup \<{\call{c}}, p>\stackup \gamma}, u,
                        \config{q'}{\gamma'}\bigr)\nonumber\\[1mm]
%
\weight_A\bigl(\config{q}{\bot}, {\call{c}}\, u,
               \config{q'}{\gamma'}\bigr) & =
 {\displaystyle\bigoplus_{\begin{array}{c}
                          \scriptstyle q'' \in Q\\[-1ex]
                          \scriptstyle p \in P
                          \end{array}}}
  \weiec(q, {\call{c}}, p, q'')
  \otimes \weight_A\bigl(\config{q''}{\<{\call{c}}, p>}, u,
                         \config{q'}{\gamma'}\bigr)\nonumber\\
%
\weight_A\bigl(\configup{q}{\<{\call{c}}, p>\stackup\gamma}, {\return{r}}\, u,
               \config{q'}{\gamma'}\bigr) & =
 {\displaystyle\bigoplus_{q'' \in Q}}
  \weir\bigl(q, {\call{c}}, p, {\return{r}}, q''\bigr)
  \otimes \weight_A\bigl(\config{q''}{\gamma}, u,
                         \config{q'}{\gamma'}\bigr)\nonumber\\
%
\weight_A\bigl(\config{q}{\bot}, {\return{r}}\, u,
               \config{q'}{\gamma'}\bigr) & =
 {\displaystyle\bigoplus_{q'' \in Q}} \weier(q, {\return{r}}, q'')
  \otimes \weight_A\bigl(\config{q''}{\bot}, u,
                         \config{q'}{\gamma'}\bigr)\nonumber
\end{align}
%\lydia{c p to <c, p>}
%
% and ${\call{c}}\, p\stacksep \gamma$
%denotes a stack where the pair made of ${\call{c}} \in \Deltac$ and $p \in P$ is the top symbol
%and $\gamma$ is the rest of stack.

\noindent
The weight associated by $A$ to $t \in \Delta^*$
is defined according to empty stack semantics:
%
\begin{equation}
A(t)  =
{\displaystyle\bigoplus_{q, q' \in Q}} \textstyle
\mathsf{in}(q) \mathop{\otimes}
\weight_A\bigl(\config{q}{\bot}, t, \config{q'}{\bot}\bigr)
\mathop{\otimes} \mathsf{out}(q').
\label{eq:SWVPA-value}
\end{equation}