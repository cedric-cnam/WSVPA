
\begin{proposition} \label{prop:epsilon}
Given a \SWT $T$ over $\Sigma$, $\Delta$, $\Semiring$ and $\bar\Phi$,
and $s \in \Sigma^+$, 
there exists an effectively constructible \SWA 
$A_{T, s}$ over $\Delta$, $\Semiring$ and $\bar\Phi$,
such that for all $t \in \Delta^+$, $A_{T, s}(t) = T(s, t)$.
\end{proposition}
%
\begin{proof}
Let $T = \< Q, \init, \bar{\wei}, \final >$,
with $\bar{\wei} = \< \wei_{10}, \wei_{01}, \wei_{11}>$,
%where the component functions of
%be the synthesized form of 
%$Q \times (\Sigma \cup \{ \varepsilon \}) \times (\Delta \cup \{ \varepsilon \}) \times Q \to \Semiring$,
and let $s = s_1 \ldots s_n$ with $s_1, \ldots, s_n \in \Sigma$ ($n \geq 1$). %$n = |s|$ be the length of $s$.

The state set of $A_{T, s}$ will be $Q' = [1..n+1] \times Q$.
Its state entering weight function is defined by, 
for all $q \in Q$ $\init'(1, q) = \init(q)$ 
and $\init'(i, q) = \zero$ for all $1 < i \leq n+1$. 
Its state leaving weight function is defined by, for all $q \in Q$ 
$\final'(n+1, q) = \final(q)$, 
and $\final'(i, q) = \zero$ for all $1 \leq i <  n+1$.

Every non-null transition of $A_{T, s}$ will
simulate a sequence of transitions of $T$ performing the following steps:
advance in the input word while staying immobile in the output word, 
and then make one step in the output word (and advance in the input word or not).
%
The first steps in the sequence correspond to $\varepsilon$-transitions of automata, 
and their total weight is computed by the following intermediate function 
$\wei'_0 : Q' \times Q' \to \Semiring$, 
defined for all $q, q' \in Q$: % all $1 \leq i < n$ and $1 \leq k \leq n-i$, 
by:
\[
\begin{array}{rcl}
\wei'_0\bigl(\<i, q>, \< i, q>\bigr) & = & \one 
\quad \mathrm{if~} 1 \leq i \leq n+1,\\
%
\wei'_0\bigl(\<i, q>, \< i, q'>\bigr) & = & \zero 
\quad \mathrm{if~} 1 \leq i \leq n+1 \mathrm{~and~} q \neq q',\\
%
\wei'_0\bigl(\<i, q>, \< i+k, q'>\bigr) & = & 
\displaystyle
\bigoplus_{\begin{array}{c}
           \scriptstyle q_0,\ldots, q_k \in Q\\[-2pt]
           \scriptstyle q_0 = q, q_k = q'
           \end{array}} 
\bigotimes_{j=0}^{k-1} \phi_{j, j+1}(s_{i+j})
\quad \begin{array}[t]{l}
      \mathrm{if~} 1 \leq i < n,
      \mathrm{~and~} 1 \leq k \leq n-i,\\
      \mathrm{where~} \phi_{j, j+1} = \wei_{10}(q_j, q_{j+1})
      \end{array}
\end{array}
\]
%where $\bigl[\wei_{10}(q_j, q_{j+1})\bigr]_{s_{i+j}}$ 
%is the partial application $\eta_{s_{i+j}} \in \Phi_{\Delta}$
%for $\eta = \wei_{10}(q_j, q_{j+1}) \in \Phi_{\Sigma, \Delta}$ ($s_{i+j} \in \Sigma$).
%
The function $\wei'_0$ is defined thanks to the closure properties of the label theory~$\bar\Phi$
(Section~\ref{section:symbols}). 
% closed under partial applications 
The sum and product in its definition are finite, 
we shall see below how to compute the first sum in polynomial time.
%The effective computation of $\wei'_0$ does not require enumerating all the state sequences 
%of length $k+1$ in the sum.
%Indeed, it amounts to search short

\noindent
We define now the transition function 
$\wei'_1: Q' \times Q' \to \Phi_\Sigma$
of $A_{T, s}$ as follows,
for $q, q' \in Q$,
$1 \leq i \leq n$, and $0 \leq k \leq n-i$:
\[
\begin{array}{lcl}
\wei'_1\bigl(\< i, q>, \< i, q'>\bigr) & = & \wei_{01}(q, q')\\
%
\wei'_1\bigl(\< i, q>, \< i+k, q'>\bigr) & = & 
\displaystyle
\bigoplus_{\begin{array}{c}
           \scriptstyle q'' \in Q\\[-2pt]
           \scriptstyle i+k < n
           \end{array}} 
\wei'_0\bigl(\<i, q>, \< i+k, q''>\bigr) \otimes \psi %\eta_{s_{i+k}}
\quad \mathrm{where~} \psi = \wei_{01}(q'', q'),\\
 & \oplus & \displaystyle
\bigoplus_{q'' \in Q} \wei'_0\bigl(\<i, q>, \< i+k-1, q''>\bigr) \otimes \eta_{s_{i+k}}
\quad \mathrm{where~}\eta = \wei_{11}(q'', q'),\\
%
\wei'_1\bigl(\< i, q>, \< j, q'>\bigr) & = & \zero  
\quad \mathrm{if~} j < i.
\end{array}      
\]
%
In the second equation, 
$\eta_{s_{i+j}} \in \Phi_{\Delta}$ is a partial application 
for $\eta \in \Phi_{\Sigma, \Delta}$, $s_{i+j} \in \Sigma$, 
defined by $\eta_{s_{i+j}}(y) = \eta(s_{i+j}, y)$ for all $y \in \Sigma$.

\noindent
We can show that the $\SWA$ $A_{T, s} = \< Q', \init', \wei'_1, \final' >$
has the expected property: $\forall t \in \Delta^+, A_{T, s}(t) = T(s, t)$.
\qed
\end{proof}

\noindent
On the quadratic computation of $\wei'_0$. **
by a best path search in the graph with nodes $[i..i+k] \times Q$
and edges labeled in $\Phi_\Sigma$...
ordering = summary.


\noindent
The construction time and size for $A_{T, s}$ are $O(\| T \| . | s |)$,
where the size $\| T \|$ of $T$ is its number of states $|Q|$.

