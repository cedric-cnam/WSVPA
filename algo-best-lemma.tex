% !TEX root = main.tex
%


%Intuitively, 
\begin{lemma}[Correctness]\label{algo-correct}
For all safe path~$\pi$ of~$\G_A$,  %with $\src(\pi) = \<q_0, q_0>$,
%with last element $v$, \\
($i$) if $\last(\pi) = \< q, q'>$, %with $q, q' \in Q$, 
then there exists $u \in \Delta^*$ such that 
$\weight_A(q[\bot], u, q'[\bot]) = \weight(\pi)$,\\
%
($ii$) if $\last(\pi) = \< q, p, q'>$, %with $q, q' \in Q$, $p\in P$, 
then there exists $v \in \Delta^*$ such that \\
${\displaystyle\bigoplus_{c \in \Deltac}}
 \weight_A\bigl(\configup{q}{\< c, p> \stackup \top }, v, \configup{q'}{\<c, p> \stackup \top}\bigr)
 = \weight(\pi)$.
\end{lemma}
%
\begin{proof}
We prove ($i$) and ($ii$) simultaneously by induction on the length $|\pi|$ of the path $\pi$.

\noindent
The base case $|\pi| = 1$ is a direct consequence of the definition of weight of paths of length 1
and the first line of \eqref{eq:SWVPA-weight}.

\noindent
If $\pi = v_0,\ldots, v_n$ with $n \geq 1$, 
let us do a case analysis on the edge $v_{n-1} \to v_n$.
%
We suppose Lemma ~\ref{algo-correct} is true for a safe path $\pi' = v_0,\ldots, v_{n-1}$ and a word $u' \in \Delta^*$. Firstly, let's assume $v_{n-1} = \< q, q''>$ and $last(\pi) = \< q, q'>$. By definition, the edge $\< q, q''> \to \< q, q'>$ is possible only if an internal symbol or a return symbol is read, in order to keep the stack empty:

\begin{itemize}
\item if we read $a \in \Delta_i$ and have an empty stack, then the new $\weight_A$ is computed with the third case in equation~\eqref{eq:SWVPA-weight}.
We have
\begin{align*}
 \weight_A\bigl(\config{q}{\gamma}, u\, a,
               \config{q'}{\bot}\bigr) &=
   \weight_A\bigl(\config{q}{\gamma}, u, \config{q''}{\bot}\bigr)
   \otimes\weiei(q'', a, q')  \\
 &= \weight(\pi')\otimes\weiei(q'', a, q')\\
 &= \weight(\pi)\nonumber 
\end{align*}
with $\pi = (\pi', v_n)$, since we know that $\weight(v_0,\ldots, v_{n-1}) \otimes \eta_A(v_{n-1} \to v_{n}) = \weight(v_0,\ldots, v_n)$ and $\weiei(q'', a, q')$ is indeed a possible edge label $\eta_A$ for the edge $\< q, q''> \to \< q, q'>$ (first line of label definition). Hence in this case the Lemma is true. 

\item Following the same reasoning as before, we find that the Lemma is also true when $a \in \Delta_r$ by using the last equation in \eqref{eq:SWVPA-weight}.
%
\end{itemize}
\noindent
Let's now assume that we have a safe path $\pi' = (v_0,\ldots, v_{n-1})$ such that $v_{n-1} = \< q_1, p, q_2>$ and $last(\pi) = \< q_0, q_3>$. The edge $\< q_1, p, q_2> \to \< q_0, q_3>$ exists if we read a return then try to match with the call on top of the stack which then leaves the stack empty. With the 6th equation in~\eqref{eq:SWVPA-weight}, we can compute the $\weight_A$ for $\<q_0, q_3>$ and then replace the first term with the 5th equation in~\eqref{eq:SWVPA-weight}, such as:
\begin{align*}
\weight_A\bigl(\config{q_0}{\bot}, u\,{\return{r}},
               \config{q_3}{\bot}\bigr) &=
  \weight_A\bigl(\config{q_0}{\bot}, u,
                 \configup{q_2}{\<{\call{c}}, p>\stackup\bot}\bigr)
  \otimes \weir\bigl(q_2, {\call{c}}, p, {\return{r}}, q_3\bigr)\\
  &=
   \weight_A(\config{q_1}{\bot}, u, \config{q_2}{\bot}\bigr)
 \otimes \weiec(q_0, {\call{c}}, p, q_1)
  \otimes \weir\bigl(q_2, {\call{c}}, p, {\return{r}}, q_3\bigr)\\
  &= \weight_A(\pi')\otimes \weiec(q_0, {\call{c}}, p, q_1)
  \otimes \weir\bigl(q_2, {\call{c}}, p, {\return{r}}, q_3\bigr)\\
  &= \weight_A(\pi)
\end{align*}
for $\pi = (\pi', v_n)$. Here the Lemma holds since $\weir\bigl(q_2, {\call{c}}, p, {\return{r}}, q_3\bigr)$ is also part of the edge labels (second line).

Finally, let's assume that $v_{n-1} = \<q_1, p, q_2>$ and $last(\pi) = \<q_0, p, q_3>$ for $p, p' \in Q$. Following the same reasoning as the two previous cases, we can similarly prove that the Lemma is also true.

\end{proof}



\begin{lemma}[Completeness] \label{algo-complete}
For all $q, q' \in Q$, $p\in P$, 
($i$) there exists a safe path~$\pi$ of~$\G_A$ such that
%such that $\src(\pi) = \<q_0, q_0>$ for some $q_0 \in Q$, 
$\last(\pi) = \< q, q'>$,
and $\weight(\pi) = b_\bot(q, q')$, 
($ii$) there exists a safe path $\pi'$ of $\G_A$ 
such that $\last(\pi') = \< q, p, q'>$,
and $\weight(\pi') = \bigoplus_\Deltac b_\top(q, p, q')$.
\end{lemma}
%
\begin{proof}
By associativity, commutativity and distributivity for $\Semiring$,
\eqref{eq:bbot} can be rewritten into the form, unfolding~\eqref{eq:SWVPA-weight}:
%
\begin{equation}\label{eq:bbot-unfold}
b_\bot(q, q') = 
\bigoplus_{t\in \Delta^*} 
\bigoplus_{\begin{array}{c}
           \scriptstyle q_0,\ldots, q_n \in Q\\
           \scriptstyle p_{i},\ldots, p_{k} \in P
           \end{array}}
\bigotimes_{i=1}^n w_i(\tau_i)           
\end{equation}
%
where $n$ is the length of $t$, $k \leq n$,
$q_0 = q$, $q_n = q'$, 
for all $1 \leq i < n$, 
$w_i$ is one of the functions of $\bar\wei$,
$\tau_i$ is a transition of $A$ and
$\src(\tau_i) = q_{i-1}$, $\snd(\tau_i) = q_i$.
%
Since $\Semiring$ is total, there exists finite sequences as above such that
\( b_\bot(q, q') = \bigotimes_{i=1}^n w_i(\tau_i)\).
%
There might exists, for $q$ and $q'$,
several finite sequences~$\bar\tau$ and~$\bar{w}$ as above, 
let us choose arbitrarily one of minimal length~$n$. 
This integer $n$ will be denoted $n_{q, q'}$ in the following.
%We represent a computation of $A$ on $t$ by a sequence $\rho$ of its transitions, 

Similarly, following~\eqref{eq:btop} and~\eqref{eq:SWVPA-weight}, 
$\bigoplus_\Deltac b_\top(q, p, q')$ can be put in the form:
%
\begin{equation}\label{eq:btop-unfold}
{\textstyle\bigoplus_\Deltac} b_\top(q, p, q') = 
{\textstyle\bigoplus_\Deltac} \bigoplus_{t\in \Delta^*} 
\bigoplus_{\begin{array}{c}
           \scriptstyle q_0,\ldots, q_n \in Q\\
           \scriptstyle p_{i},\ldots, p_{k} \in P
           \end{array}}
\bigotimes_{i=1}^n w_i(\tau_i)           
\end{equation}
%
and hence 
\( {\textstyle\bigoplus_\Deltac} b_\top(q, p, q') = \bigotimes_{i=1}^n w_i(\tau_i) \),
using $c \in \Deltac$ that minimize the function in rhs of~\eqref{eq:btop-unfold}.
We denote the smallest $n$ as above by $n_{q, p, q'}$.

\noindent
We show now the existence of a path $\pi$ as expected, %in Lemma~\ref{lem:algo-complete}, 
by simultaneous induction on $n_{q, q'}$ and $n_{q, p, q'}$.

\noindent
The base case, $n = 0$ corresponds to $t = \varepsilon$. 
In this case, by~\eqref{eq:SWVPA-weight}, $b_\top(q, p, q') = \zero$,
$b_\bot(q, q) = \one$ and $b_\bot(q, q') = \zero$ if $q \neq q'$.

\noindent
For $n = n_{q, q'} > 0$, let $\bar\tau$ and~$\bar{w}$
be the sequences associated to $q$ and $q'$ as above.
Let us do a case analysis of $w_n$.
... \qed
\end{proof}