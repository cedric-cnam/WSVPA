% !TEX root = main.tex
%
% Semiring basics
%
We shall consider semirings for the weight values of our language models.
\florent{The results are established for a general class of semirings. They can be instantiated for concrete cases}
%
A \emph{semiring} $\< \Semiring, \oplus, \zero, \otimes, \one>$ 
is a structure with a domain~$\Semiring$,
equipped with two associative
binary operators~$\oplus$ and $\otimes$,
with respective neutral elements $\zero$ and $\one$, and such that:
%$\< \mathbb{S}, \oplus, \zero>$ is a commutative monoid
%$\< \mathbb{S}, \otimes, \one>$ is a monoid
\begin{itemize}
\item $\oplus$ is commutative: 
 $\< \Semiring, \oplus, \zero>$ is a commutative monoid 
   and $\< \Semiring, \otimes, \one>$ a monoid,
\item $\otimes$ distributes over~$\oplus$:  $\forall x, y, z \in \mathbb{S}$,
$x \otimes (y \oplus z) = (x \otimes y) \oplus (x \otimes z)$, 
and $(x \oplus y) \otimes z = (x \otimes z) \oplus (y \otimes z)$,
\item $\zero$ is absorbing for~$\otimes$: 
$\forall x\in \mathbb{S}$, $\zero \otimes x = x \otimes \zero = \zero$.
\end{itemize}
%Components of a semiring~$\Semiring$ may be subscripted by~$\Semiring$ when needed.
%We simply write $x \in \Semiring$ to mean $x \in \mathbb{S}$.
%
Intuitively, in the models presented in this paper, 
$\oplus$ selects an optimal value from two given values, 
in order to handle non-determinism, 
and $\otimes$ combines two values into a single value, 
in a chaining of transitions.
%and let $\< \Semiring, \oplus, \zero, \otimes, \one>$ be a {semiring}, 

\medskip%\noindent
A semiring $\Semiring$ is \emph{commutative} if $\otimes$ is commutative.
It is \emph{idempotent} if for each $x \in \dom(\Semiring)$, $x \oplus x = x$.
%
Every idempotent semiring~$\Semiring$ induces 
a partial ordering~$\leq_\oplus$ 
called the \emph{natural ordering} of~$\Semiring$~\cite{Mohri02semiring} 
and defined,  by: 
%implicitly defined by the semiring $\Semiring$ 
for all $x$ and $y$,
$x \leq_\oplus y \;\mbox{iff}\; x \oplus y = x$.
%(see~\cite{Mohri02semiring} for the proof that it is an ordering).
%
The natural ordering is sometimes defined in the opposite direction~\cite{DrosteKuich09semirings};
We follow here the direction  %follows \cite{Mohri02semiring}, and 
that coincides with the usual ordering on the Tropical semiring \emph{min-plus} 
(Figure~\ref{fig:semirings}).
%
\noindent
An idempotent semiring $\Semiring$~is called \emph{total} if
it~$\leq_\oplus$ is total
\ie when for all $x, y \in \Semiring$, either $x \oplus y = x$ or $x \oplus y = y$.

\begin{lemma}[Monotony, \cite{Mohri02semiring}] \label{lem:monotonic}
Let $\< \Semiring, \oplus, \zero, \otimes, \one>$ be an idempotent semiring.
For all $x, y, z  \in \Semiring$,  
if $x \leq_\oplus y$ then
$x \oplus z \leq_\oplus y \oplus z$,
$x \otimes z \leq_\oplus y \otimes z$
and $z \otimes x \leq_\oplus z \otimes y$.
\end{lemma}   
When the property of Lemma~\ref{lem:monotonic} holds, 
%We say in this case that 
$\Semiring$ is called  \emph{monotonic}. % \wrt $\leq_\oplus$.
%A semiring $\Semiring$ 
%is \emph{monotonic} \wrt a partial ordering~$\leq$ 
%iff for all $x, y, z  \in \Semiring$,  $x \leq y$ implies
%$x \oplus z \leq y \oplus z$,
%$x \otimes z \leq y \otimes z$,
%and $z \otimes x \leq z \otimes y$.
%
Another important semiring property in the context of optimization
is {superiority}~\cite{Huang08advanceddynamic}, 
which corresponds to the 
\emph{non-negative weights} condition in shortest-path algorithms~\cite{Dijkstra59anote}.
Intuitively, it means that combining elements with $\otimes$ always increase their weight. 
Formally, it is defined as the property ($i$) below. %of the following lemma.

\begin{lemma}[Superiority, Boundedness]
\label{lem:superior}\label{lem:bounded}
Let $\< \Semiring, \oplus, \zero, \otimes, \one>$ be an idempotent semiring.
The two following statements are equivalent:
\begin{itemize}
\item [$i.$] for all $x, y \in \Semiring$,  
$x \leq_\oplus x \otimes y$ and 
$y \leq_\oplus x \otimes y$
\item[$ii.$] for all $x \in \Semiring$,  $\one \oplus x = \one$.
\end{itemize}
\end{lemma}
%
\begin{proof} %(Lemma~\ref{lem:bounded})
$(ii) \Rightarrow (i)$ : 
$x \oplus (x \otimes y) = x \otimes (\one \oplus y) = x$, 
by distributivity of~$\otimes$ over~$\oplus$. 
Hence $x \leq_\oplus x \otimes y$.
Similarly, $y \oplus (x \otimes y) = (\one \oplus x) \otimes y = y$, 
hence $y \leq_\oplus x \otimes y$.
%
$(i) \Rightarrow (ii)$ :
%$(i)$ implies that $\one \leq_\oplus x$ for all $x \in \Semiring$, 
by the second inequality of ($i$), with $y = \one$, 
$\one \leq_\oplus x \otimes \one = x$, \ie, 
by definition of $\leq_\oplus$, $\one \oplus x = \one$.
%\qed
\end{proof}

In~\cite{Huang08advanceddynamic}, when the property~$(i)$ holds,  
$\Semiring$ is called \emph{superior} \wrt the ordering~$\leq_\oplus$.
We have seen in the proof of Lemma~\ref{lem:bounded} that it implies that 
$\one \leq_\oplus x$ for all $x \in \Semiring$.
Similarly, by the first inequality of ($i$) with $y = \zero$,  
$x \leq_\oplus x \otimes \zero = \zero$.
%
Hence, in a superior semiring, % \wrt~$\leq$, 
it holds that %$\one \leq \zero$
for all $x \in \Semiring$, $\one \leq_\oplus x \leq_\oplus \zero$.
%
Intuitively, from an optimization point of view,
it means that $\one$ is the best value, and $\zero$ the worst.
%** superior implies $\Semiring$ bounded~\cite{Mohri02semiring} see **
%
In~\cite{Mohri02semiring}, 
$\Semiring$ with the property ($ii$) of Lemma~\ref{lem:bounded}  
is called \emph{bounded} -- we shall use this term in the rest of the paper. 
% \emph{negative boundedness} 
It implies that, when looking for a best path in a graph whose edges
are weighted by values of $\Semiring$, the loops can be safely avoided
(because, for all $x \in \Semiring$ and $n \geq 1$, 
 $x \oplus x^n = x \otimes (\one \oplus x^{n-1}) = x$).


\begin{lemma}
Every bounded semiring is idempotent.
\end{lemma}
\begin{proof}
By boundedness, $\one \oplus \one = \one$, 
and idempotency follows by multiplying
both sides by $x$ and distributing. 
%\qed
\end{proof}

%\medskip
\noindent
We shall need below infinite sums with~$\oplus$.
A semiring~$\Semiring$ is called \emph{complete}~\cite{Droste09handbook} 
%(\cite{Droste09handbook} chapter 1) %\cite{Kuich97semirings}
if it has an %infinite sum 
operation $\bigoplus_{i \in I} x_i$
for every family
$(x_i)_{i \in I}$ %$\{ x_i \mid i \in I \}$
of elements of $\dom(\Semiring)$ over an index set $I \subset \mathbb{N}$, such that:
%the following holds: %properties
\begin{description}
\item[$i.$]
\emph{infinite sums extend finite sums:}\\
$\displaystyle\bigoplus_{i \in \emptyset} x_i = \zero$,\quad 
      $\forall j\in \mathbb{N}, \displaystyle\bigoplus_{i \in \{ j \}} x_i = x_j$,
      $\forall j, k\in \mathbb{N}, j\neq k, 
      \displaystyle\bigoplus_{i \in \{ j, k \}} x_i = x_j \oplus x_k$,
%
\item[$ii.$]
\emph{associativity and commutativity:}\\
for all $I \subseteq \mathbb{N}$
and all partition $(I_{j})_{j \in J}$ of $I$, %\subseteq \mathbb{N}$, 
\(
\displaystyle
\bigoplus_{j \in J}\bigoplus_{i \in I_j} x_i = 
\bigoplus_{i \in I} x_i
\),
%
\item[$iii.$] 
\emph{distributivity of product over infinite sum:}\\
for all $I \subseteq \mathbb{N}$,
\(
\displaystyle
\bigoplus_{i \in I} (x \otimes y_i) = x \otimes \bigoplus_{i\in I} y_i\), and
\(
\displaystyle
\bigoplus_{i \in I} (x_i \otimes y) = (\bigoplus_{i \in I} x_i ) \otimes y\).
\end{description}



%\begin{example}
%Figure~\ref{fig:semirings} presents examples of semirings interesting in practice 
%and enjoying the above properties.
%\end{example}


\begin{figure}[t]
\begin{center}
\begin{tabular}{|c|c|C{2em}|C{2em}|C{2em}|C{2em}|}
\hline
        & domain & $\oplus$ & $\otimes$ & $\zero$  & $\one$\\ %& nat. ordering\\
\hline\hline
Boolean & $\{\bot, \top\}$ & $\vee$ & $\wedge$ & $\bot$ & $\top$\\ %& $\top \leq_\oplus \bot$  \\
\hline
Counting & $\mathbb{N}$ & $+$ & $\times$ & 0 & 1 \\
\hline
Viterbi & $[0, 1] \subset \mathbb{R}$ & $\mathit{max}$ & $\times$ & 0 & 1\\ % & $x \leq_\oplus y \iff x \ge y$  \\
\hline
Tropical min-plus & $\mathbb{R}_+ \cup \{ \infty\}$ & $\mathit{min}$ & $+$ & $\infty$ & $0$\\ % & $x \leq_\oplus y \iff x \leq y$   \\
\hline
%MaxPlus & $\mathbb{Q} \cup \{ -\infty\}$ & $\mathsf{max}$ & $+$ & $-\infty$ & $0$ \\
%\hline
%Word lang. & $2^{\Sigma^*}$ & $\cup$ & $\cdot$ & $\emptyset$ & $\{ \varepsilon \}$ \\
%\hline
\end{tabular}
\end{center}
\vskip-1em
\caption{Some commutative, bounded, total and complete semirings.}
\label{fig:semirings}
\end{figure}
\florent{results of this paper: for semirings commutative, bounded, total and complete}

%%% semiring properties used in paper
% - commutative
% - bounded (implies idempotent and superior)
% - total
% - complete




