% !TEX root = main.tex
%
% Proof of Proposition closure swA by swT
%
Let $T = \< Q, \init_T, \bar{\wei}, \final_T >$,
where $\bar{\wei}$ contains $\wei_{10}$, $\wei_{01}$, and $\wei_{11}$,
from $Q \times Q$ into respectively
$\Phi_{\Sigma}$, $\Phi_{\Delta}$, and $\Phi_{\Sigma, \Delta}$,
and let $A = \< P, \init_A, \wei_1, \final_A >$
with $\wei_1: Q \times Q \to \Phi_{\Sigma}$.
The state set of $B_{A, T}$ will be $Q' = P \times Q$.

The entering, leaving and transition functions of $B_{A, T}$ will
simulate synchronized computations of $A$ and $T$,
while reading an output word of $t \in \Delta^*$, 
and some input word $s \in \Sigma^*$.
%
Its state entering functions is defined
for all $\<p_1, q_1> \in Q'$, %$p_1 \in P$, $q_1 \in Q$
by:
%
\begin{equation} \label{eq:initprime}
\init'\bigl(\< p_1, q_1>\bigr) = \init_A(p_1) \otimes \init_T(q_1).
\end{equation}
%
The transition function $\wei'_1$ will roughly perform
a synchronized product of transitions defined by $\wei_1$,
$\wei_{01}$ ($T$ reading in output word and not in input word)
and $\wei_{11}$ ($T$ reading both in input word and in output word).
%
Moreover, $\wei'_1$ also needs to simulate transitions
defined by $\wei_{10}$: $T$ reading in input word and not in output word.
Since $B_{A, T}$  will read only in the output word, such a transition corresponds
to an $\varepsilon$-transition of $\SWA$.
But $\SWA$ have been defined without $\varepsilon$-transitions.
Therefore, in order to take care of this case, we perform an on-the-fly
elimination of $\varepsilon$-transition during the construction of the $\SWA$,
following Algorithm~1 of~\cite{LombardySakarovitch12ciaa}.
%
% Initialize state leaving functions i
%and $\final'(p, q) = \final_A(p) \otimes \final_T(q)$.

%Every transition of $B_{A, T}$ will
%simulate a sequence of transitions of $T$ performing the following steps:
%advance in the input word while staying immobile in the output word,
%and then make one step in the output word (and advance in the input word or not).

\noindent
The transition function $\wei'_1$ is constructed iteratively.

\noindent
Initially, for all $p_1, p_2 \in P$, and $q_1, q_2 \in Q$, let
%
\begin{align}
%\begin{array}{rcl}
\wei'_1\bigl( \< p_1, q_1>, \< p_2, q_2>\bigr) & = 
\bigl({\displaystyle\bigoplus_{p_1 = p_2}} \wei_{01}(q_1, q_2)\;\bigr)
\oplus
{\bigoplus}_\Sigma^1 \bigl(\wei_1(p_1, p_2) \otimes_1 \wei_{11}(q_1, q_2)\bigr)
\label{eq:init-wei}\\
%
\final'\bigl(\<p_1, q_1>\bigr) & =  \final_A(p_1) \otimes \final_T(q_1)
\label{eq:init-final}
%\end{array}
\end{align}
We recall that by convention, 
${\displaystyle\bigoplus_{p_1 = p_2}} \wei_{01}(q_1, q_2)$
is equal to $\zero$ if $p_1 \neq p_2$.


\noindent
Then, we iterate the following updates for all $p_1, p_2, p_3\in P$ 
and $q_1, q_2, q_3 \in Q$:
%
\begin{align}
%\begin{array}{rcl}
\wei'_1\bigl( \< p_1, q_1>, \< p_3, q_3>\bigr) & \opluseq 
%\displaystyle\bigoplus_{\Sigma} 
\bigoplus_\Sigma \bigl(\wei_{1}(p_1, p_2) \otimes \wei_{10}(q_1, q_2)\bigr)
\otimes \wei'_1\bigl( \< p_2, q_2>, \< p_3, q_3>\bigr) \label{eq:iter-wei}\\
%
\final'\bigl(\<p_2, q_2>\bigr) & \opluseq 
%\displaystyle\bigoplus_{\Sigma} 
\bigoplus_\Sigma \bigl(\wei_{1}(p_1, p_2)\otimes \wei_{10}(q_1, q_2)\bigr)
\otimes \final'(p_1, q_1) \label{eq:iter-final}
%\end{array}
\end{align}
%
In both cases of updates of $\wei'_1$ and $\final'$ during the iteration, 
$\wei_{1}(p_1, p_2) \otimes \wei_{10}(q_1, q_2)$ 
is the weight of an $\varepsilon$-transition.
It corresponds to the reading, by $A$ and $T$, 
of a symbol $a$ in the input word $s$ without moving in the output word,
i.e. the synchronization of 
a transition $\wei_{1}(p_1, a, p_2)$ of $A$ and 
a transition $\wei_{10}(q_1, a, \varepsilon, q_2)$ of~$T$.

The iteration stops if it does not change the value of $\wei'_1$ and $\final'$.
By hypothesis and Lemma~\ref{lem:idempotent}, 
$\Semiring$ is idempotent. 
Therefore, the construction of $B_{A, T}$ will stop after 
at most $|P|^2 . |Q|^2$ iterations.

\smallskip
Let us now show that 
$B_{A, T}(t) = \displaystyle\bigoplus_{s\in \Sigma^*} A(s) \otimes T(s, t)$
for all $t\in \Delta^+$.

\noindent
We call \emph{path} from $\< p_0, q_0>$ to $\< p_n, q_n>$ 
a finite sequence of the form:
$\pi = \< p_0, q_0>$, $a_1, \< p_1, q_1>$,\ldots, $a_n, \< p_n, q_n>$
where $\< p_i, q_i> \in Q'$ for all $0 \leq i \leq n$
and $a_j \in \Sigma$ for all $1 \leq j \leq n$.
The state $\< p_0, q_0>$ is called source of the path, denoted $\mathit{src}(\pi)$
and $\< p_n, q_n>$ is called target of the path, denoted $\mathit{trg}(\pi)$;
the set of paths with source $\< p, q>$ and target $\< p', q'>$ 
is denoted $\Pi(\< p, q>, \< p', q'>)$.
The word $a_1\ldots a_n \in \Sigma^*$ is called word of the path $\pi$
and denoted $\word(\pi)$.
%The set of all paths
Moreover, we associate a weight value in $\Semiring$ to every path, 
defined by:
\begin{equation} \label{def:weight-path}
\weight(\pi) = \bigotimes_{i = 1}^{n} 
\wei_1(p_{i-1}, a_i, p_i) \otimes
\wei_{10}(q_{i-1}, a_i, \varepsilon, q_i)
\end{equation}

By definition of the weight functions, and 
associativity, commutativity, and distributivity of $\oplus$, $\otimes$,  
it holds that:
\begin{equation}\label{eq:weight-path}
\weight_A(p, s, p') \otimes \weight_T(q, s, \varepsilon, q') = 
\bigoplus_{\begin{array}{c}
		   \scriptstyle\pi \in \Pi(\< p, q>, \< p', q'>)\\[-1ex]
		   %\scriptstyle\mathit{src}(\pi) = \<p , q>\\[-1ex]
		   %\scriptstyle\mathit{trg}(\pi) = \<p' , q'>\\[-1ex]
		   \scriptstyle\word(\pi) = s
		   \end{array}} 
\weight(\pi) 
\end{equation}
%
Using \eqref{def:weight-path} 
and Lemma~\ref{lem:label-th} repetively, 
\eqref{eq:weight-path} implies:
%
\begin{equation}\label{eq:weight-path-plus}
\begin{array}{l}
\displaystyle\bigoplus_{s \in \Sigma^*} 
  \weight_A(p, s, p') \otimes \weight_T(q, s, \varepsilon, q') =\\
\multicolumn{1}{r}{%
\qquad
\displaystyle\bigoplus_{\begin{array}{c}
		   \scriptstyle\pi \in \Pi(\< p, q>, \< p', q'>)\\[-1ex]
		   \scriptstyle\pi = \< p_0, q_0>, a_1,\< p_1, q_1>,\ldots, a_n, \< p_n, q_n>\\[-1ex]
%		   \scriptstyle\pi = \< p_0, q_0>, a_1,\\[-1ex]
%		   \scriptstyle\< p_1, q_1>,\ldots,\\[-1ex]
%       	   \scriptstyle a_n, \< p_n, q_n>\\[-1ex]
%		   \scriptstyle\< p_0, q_0> = \< p, q>\\[-1ex]
%		   \scriptstyle\< p_n, q_n>  = \< p', q'>	   
		   \end{array}} 
\displaystyle\bigotimes_{i = 1}^{n} 
\displaystyle\bigoplus_\Sigma (\wei_1(p_{i-1}, p_i) \otimes
                  \wei_{10}(q_{i-1}, q_i))
}
\end{array}           %\nonumber
\end{equation}
%
Note that the symbols $a_1, \ldots, a_n \in \Sigma$ in the path $\pi$
are not significant in \eqref{eq:weight-path-plus}.
%
Using a pumping argument, we can show that~\eqref{eq:weight-path-plus}
still holds when restricting $\pi$ to the set $\Pi_0(\< p, q>, \< p', q'>)$
of paths without repetition in the state symbols.
%
Indeed, assume that in 
$\pi = \< p_0, q_0>, a_1,\< p_1, q_1>,\ldots, a_n, \< p_n, q_n>$, 
$\< p_{i_1}, q_{i_1}> = \< p_{i_2}, q_{i_2}>$
for $0 \leq i_1 < i_2 \leq n$.
Then 
$\pi' = 
 \< p_0, q_0>, \ldots, 
 a_{i_1-1},\< p_{i_1-1}, q_{i_1-1}>,
 a_{i_2},\< p_{i_2}, q_{i_2}>, \ldots,  a_n, \< p_n, q_n>$ 
also belongs to   $\Pi\bigl(\< p_0, q_0>, \< p_n, q_n>\bigr)$
and yields a smaller expression (\wrt $\leq_\oplus$) 
in the right-hand-side of \eqref{eq:weight-path-plus}
than~$\pi$.
It follows, by~\eqref{eq:init-wei} and \eqref{eq:iter-wei}, that
for all $b \in \Delta$, 
%
\begin{equation}\label{eq:weiprime1}
%\begin{array}{rcl}
\wei'_1(\< p, q>, b, \< p', q'>) = 
\displaystyle\bigoplus_{s \in \Sigma^*} 
\displaystyle\bigoplus_{\begin{array}{c}
                        \scriptstyle p'' \in P\\[-1ex]
                        \scriptstyle q'' \in Q
                        \end{array}}
\weight_A(p, s, p'') \otimes \weight_T(q, s, \varepsilon, q'') \otimes \psi_1(b)
%\weight_{B_{A, T}}(\< p, q>, s, \< p', q'>)
%\end{array}
\end{equation}
where 
$\psi_1 = \wei_{01}(q'', q') \oplus
\displaystyle{\bigoplus}_\Sigma^1 \bigl(\wei_1(p'', p') 
 \otimes_1 \wei_{11}(q'', q')\bigr)$.
%
% lift to $A(s) \otimes T(s, \varepsilon)$

\medskip
We show now by induction on the length of $t \in \Delta^+$, that
\[
\weight_{B_{A, T}}(\<p, q>, t, \<p', q'>) = 
 \displaystyle\bigoplus_{s\in \Sigma^*}
 \weight_{A}(p, s, p') \otimes  \weight_{A}(q, s, t, q')
\]
This permits to conclude, using 
the definition of $\init'$ in~\eqref{eq:initprime},
%and a proof for 
and the definition of $\final'$ 
in~\eqref{eq:init-final}, and~\eqref{eq:iter-final}. 
%$B_{A, T}(t) = \displaystyle\bigoplus_{s\in \Sigma^*} A(s) \otimes T(s, t)$

\noindent
The base case $t \in \Delta$ follows from~\eqref{eq:weiprime1} and 
the distributivity of $\otimes$.

\noindent
For $t = b u$, with $b \in \Delta$ and $u \in \Delta^*$,
by definition of $\weight_A$ and $\weight_T$,
it holds that for all $s \in \Sigma^*$:
\[
\begin{array}{l}
\displaystyle\weight_A(p, s, p') \otimes \weight_T(q, s, t, q') = 
\bigoplus_{s = s_1 s_2}
\displaystyle\bigoplus_{\begin{array}{c}
                        \scriptstyle p'', p''' \in P\\[-1ex]
                        \scriptstyle q'', q''' \in Q
                        \end{array}}
\weight_A(p, s_1, p''') \otimes \weight_A(p''', s_2, p'') \mathop\otimes \\
\hfill\displaystyle
\weight_T(q, s_1, \varepsilon, q'') \otimes
 \left(
 \begin{array}{l}
 \displaystyle
 \bigoplus_{q''' \in Q}
 \wei_{01}(q'', \varepsilon, b, q''') \otimes \weight_T(q''', s_2, u, q')
 \mathop\oplus\\
 \displaystyle
 \bigoplus_{q''' \in Q} \bigoplus_{s_2 = a s'_2}
 \wei_{11}(q'', a, b, q''') \otimes \weight_T(q''', s'_2, u, q')
 \end{array}
 \right)
\end{array}
\]

\noindent
It follows that:
\[
\begin{array}{l}
\displaystyle\bigoplus_{s \in \Sigma^*} 
\displaystyle\weight_A(p, s, p') \otimes \weight_T(q, s, t, q') = \\
\qquad
\displaystyle\bigoplus_{\begin{array}{c}
                        \scriptstyle p'', p''' \in P\\[-1ex]
                        \scriptstyle q'', q''' \in Q
                        \end{array}}
\bigoplus_{s_1 \in \Sigma^*}
\weight_A(p, s_1, p'') \otimes \weight_T(q, s_1, \varepsilon, q'') \otimes \psi_1(b)\\
%\wei'_1(\< p, q>, b, \< p'', q''>) \mathop\otimes \\
\qquad
\otimes\displaystyle\bigoplus_{\scriptstyle s_2 \in \Sigma^*}
\weight_A(p''', s_2, p') \otimes \weight_T(q''', s_2, u, q')
%\displaystyle\bigoplus_{\begin{array}{c}
%                        \scriptstyle s_1 \in \Sigma^*\\[-1ex]
%                        \scriptstyle s_2 \in \Sigma^*
%                        \end{array}}
\end{array}
\]
with 
$\psi_1 = \wei_{01}(q'', q''') \oplus
\displaystyle{\bigoplus}_\Sigma^1 \bigl(\wei_1(p'', p''') 
 \otimes_1 \wei_{11}(q'', q''')\bigr)$.

\noindent
The first term in the right-hand-side is 
$\wei'_1(\< p, q>, b, \< p''', q'''>)$ by \eqref{eq:weiprime1},
and the second term is
$\weight_{B_{A,T}}(\< p''', q'''>, u, \< p', q'>)$ by induction hypothesis.
Hence, by definition, 
\[
\begin{array}{rcl}
\displaystyle\bigoplus_{s \in \Sigma^*} 
\displaystyle\weight_A(p, s, p') \otimes \weight_T(q, s, t, q') & = & \\
\multicolumn{3}{r}{%
\displaystyle\bigoplus_{\begin{array}{c}
                        \scriptstyle p''' \in P\\[-1ex]
                        \scriptstyle q''' \in Q
                        \end{array}}
 \wei'_1(\< p, q>, b, \< p''', q'''>) \otimes
 \weight_{B_{A,T}}(\< p''', q'''>, u, \< p', q'>)}\\
 & = & \weight_{B_{A,T}}(\< p, q>, t, \< p', q'>).\\
\end{array}
\]




%
%
% old version
%
%\noindent
%The entering, leaving and transition functions of $B_{A, T}$ will
%simulate synchronized computations of $A$ and $T$,
%while reading an output word of $\Delta^*$.
%%
%Its state entering functions is defined
%for all $\<p_1, q_1> \in Q'$, %$p_1 \in P$, $q_1 \in Q$
%by $\init'(p_1, q_1) = \init_A(p_1) \otimes \init_T(q_1)$.
%%
%The transition function $\wei'_1$ will roughly perform
%a synchronized product of transitions defined by $\wei_1$,
%$\wei_{01}$ ($T$ reading in output word and not an input word)
%and $\wei_{11}$ ($T$ reading both an input word and an output word).
%%
%Moreover, $\wei'_1$ also needs to simulate transitions
%defined by $\wei_{10}$: $T$ reading in input word and not an output word.
%Since $B_{A, T}$  will read only in the output word, such a transition corresponds
%to an $\varepsilon$-transition of $\SWA$.
%But $\SWA$ have been defined without $\varepsilon$-transitions.
%Therefore, in order to take care of this case, we perform an on-the-fly
%suppression of $\varepsilon$-transition in the $\SWA$ in construction,
%following the algorithm of~\cite{LombardySakarovitch12ciaa}.
%%
%% Initialize state leaving functions i
%%and $\final'(p, q) = \final_A(p) \otimes \final_T(q)$.
%
%%Every transition of $B_{A, T}$ will
%%simulate a sequence of transitions of $T$ performing the following steps:
%%advance in the input word while staying immobile in the output word,
%%and then make one step in the output word (and advance in the input word or not).
%
%\noindent
%Initially, for all $p_1, p_2 \in P$, and $q_1, q_2 \in Q$, let
%\[
%\begin{array}{rcl}
%\wei'_1\bigl( \< p_1, q_1>, \< p_2, q_2>\bigr) & = &
%\wei_1(p_1, p_2) \otimes
%\bigl[
%\wei_{01}(q_1, q_2)
%\oplus
%\displaystyle\bigoplus_{\Sigma}
%\wei_{11}(q_1, q_2)
%\bigr]\\
%\final'(p_1, q_1) & = & \final_A(p_1) \otimes \final_T(q_1)
%\end{array}
%\]
%
%\noindent
%Then, we iterate the following for all $p_1\in P$ and $q_1, q_2 \in Q$:
%for all $p_2\in P$ and $q_3 \in Q$,
%\[
%\begin{array}{rcl}
%\wei'_1\bigl( \< p_1, q_1>, \< p_2, q_3>\bigr) & \opluseq &
%\displaystyle\bigoplus_{\Sigma} \wei_{10}(q_1, q_2)
%\otimes
%\wei'_1\bigl( \< p_1, q_2>, \< p_2, q_3>\bigr)\\
%\final'(p_1, q_1) & \opluseq &
%\displaystyle\bigoplus_{\Sigma} \wei_{10}(q_1, q_2)
%\otimes \final'(p_1, q_2)
%\end{array}
%\]