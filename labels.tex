% !TEX root = main.tex
%
% Label Theory
%
We shall now define the functions labeling the transitions of SW automata and transducers,
generalizing the Boolean algebras of~\cite{dAntoniVeanes17CAV} 
from Boolean to other semiring domains.
%
We consider \emph{alphabets}, which are countable sets of symbols 
denoted $\Sigma$, $\Delta$,...
%Let $\< \Semiring, \oplus, \zero, \otimes, \one>$ be a commutative, complete semiring.
%
\noindent 
Given a semiring $\< \Semiring, \oplus, \zero, \otimes, \one>$, 
a \emph{label theory} over $\Semiring$
is a tuplet $\bar\Phi$ of recursively enumerable sets denoted
%$\Phi_\epsilon \subseteq \Semiring$, % containing constant functions valued in $\Semiring$, 
$\Phi_\Sigma$, %and $\Phi_\Delta$, 
containing unary functions of type $\Sigma \to \Semiring$, %resp. $\Delta \to \Semiring$, 
and $\Phi_{\Sigma, \Delta}$, containing binary functions $\Sigma \times \Delta \to \Semiring$, 
and such that:

\noindent -- 
for all $\Phi_{\Sigma, \Delta} \in \bar\Phi$, we have
$\Phi_{\Sigma} \in \bar\Phi$ and $\Phi_{\Delta} \in \bar\Phi$

\noindent -- 
every $\Phi_{\Sigma}$ contains all the constant functions from $\Sigma$ into $\Semiring$, 
 
\noindent -- 
for all $\alpha \in \Semiring$ and $\phi \in \Phi_\Sigma$,
      $\alpha \otimes \phi : x \mapsto \alpha \otimes \phi(x)$, 
      and $\phi \otimes \alpha : x \mapsto \phi(x) \otimes \alpha$\\
\phantom{--} belong to $\Phi_\Sigma$, and similarly for $\oplus$ 
      and for $\Phi_{\Sigma, \Delta}$

\noindent -- 
for all $\phi, \phi' \in \Phi_\Sigma$,
$\phi \otimes \phi': x \mapsto \phi(x) \otimes \phi'(x)$ belongs to $\Phi_\Sigma$

\noindent -- 
for all $\eta, \eta' \in \Phi_{\Sigma, \Delta}$
$\eta \otimes \eta': x, y \mapsto \eta(x, y) \otimes \eta'(x, y)$ belongs to $\Phi_{\Sigma, \Delta}$

\noindent -- 
for all $\phi \in \Phi_\Sigma$ and $\eta \in \Phi_{\Sigma, \Delta}$,
$\phi \otimes_1 \eta: x, y \mapsto \phi(x) \otimes \eta(x, y)$ and\\
\phantom{--} $\eta \otimes_1 \phi: x, y \mapsto \eta(x, y) \otimes \phi(x)$
belong to $\Phi_{\Sigma, \Delta}$

\noindent -- 
for all $\psi \in \Phi_\Delta$ and $\eta \in \Phi_{\Sigma, \Delta}$,
$\psi \otimes_2 \eta: x, y \mapsto \psi(y) \otimes \eta(x, y)$ and\\
\phantom{--} $\eta \otimes_2 \psi: x, y \mapsto \eta(x, y) \otimes \psi(y)$ 
belong to $\Phi_{\Sigma, \Delta}$

\noindent -- 
similar closures hold for $\oplus$

\noindent -- 
the partial applications $\eta \in \Phi_{\Sigma, \Delta}$
and $\eta_a: y \mapsto \eta(a, y)$ for $a \in \Sigma$ %and $y \in \Delta$
and\\ 
\phantom{--} $\eta_b: x \mapsto \eta(x, b)$ for $b \in \Delta$ %and $x \in \Sigma$, 
belong respectively to~$\Phi_\Delta$ and~$\Phi_\Sigma$.
\marginpar{\tiny partial appli needed?}

\medskip\noindent
In what follows, we shall omit the subscripts in 
$\otimes_1$, $\otimes_2$, $\oplus_1$, $\oplus_2$
when there is no ambiguity, 
and keep them only for the special case $\Sigma = \Delta$,
\ie $\eta \in \Phi_{\Sigma, \Sigma}$. % for~$\otimes_1$ above,
%and $\eta \in \Phi_{\Delta, \Delta}$ for~$\otimes_2$. 

%Moreover, these sets are required to be closed under the 
%operators~$\oplus$ and~$\otimes$ of~$\Semiring$:
%for all $\phi, \phi' \in \Phi_\Sigma$,
%$\psi, \psi' \in \Phi_\Delta$, 
%and $\eta, \eta' \in \Phi_{\Sigma, \Delta}$, %the function
%%
%\begin{center}
%\begin{tabular}{cclll}
%$\phi \otimes \phi'$ & : & $x \mapsto \phi(x) \otimes \phi'(x)$ & belongs to $\Phi_\Sigma$,\\
%$\psi \otimes \psi'$ & : & $y \mapsto \psi(y) \otimes \psi'(y)$ & belongs to $\Phi_\Delta$,\\
%$\phi \otimes \eta$\;  & : & $x, y \mapsto \phi(x) \otimes \eta(x, y)$ & belongs to $\Phi_{\Sigma, \Delta}$,\\
%$\eta \otimes \psi$  & : & $x, y \mapsto \eta(x, y) \otimes \psi(y)$ & belongs to $\Phi_{\Sigma, \Delta}$,\\
%$\eta \otimes \eta'$ & : & $x, y \mapsto \eta(x, y) \otimes \eta'(x, y)$ & belongs to $\Phi_{\Sigma, \Delta}$, &
%\multicolumn{1}{r}{and similarly for $\oplus$.}\\ %the same also holds for the binary sum operator $\oplus$.
%\end{tabular}
%\end{center}
%
%Finally, it is also required 
%% that the codomain of every function of $\Phi_\Sigma$ and $\Phi_\Delta$ 
%% is a subset of $\Phi_\epsilon$, and
%that the partial applications of a function $\eta \in \Phi_{\Sigma, \Delta}$, 
%resp.  $\eta_a: y \mapsto f(a, y)$ for $a \in \Sigma$ and $y \in \Delta$
%and  $\eta_b: x \mapsto f(x, b)$ for $b \in \Delta$ and $x \in \Sigma$, 
%belong resp. to~$\Phi_\Sigma$ and~$\Phi_\Delta$.

\noindent
When the semiring $\Semiring$ is complete, 
let us consider the following operators on the functions of a label theory
(we use overloading to simplify notations):
\[
\begin{array}{ll}
\bigoplus_\Sigma : \Phi_\Sigma \to \Semiring,\ 
  \phi \mapsto \displaystyle\bigoplus_{a \in \Sigma} \phi(a)\\
\bigoplus^1_\Sigma : 
  \Phi_{\Sigma,\Delta} \to \Phi_\Delta,\ 
  \eta \mapsto \bigl( y \mapsto \displaystyle\bigoplus_{a \in \Sigma} \eta(a, y) \bigr) &
\bigoplus^2_\Delta : 
  \Phi_{\Sigma,\Delta} \to \Phi_\Sigma,\ 
  \eta \mapsto \bigl( x \mapsto \displaystyle\bigoplus_{b \in \Delta} \eta(x, b) \bigr)\\
\end{array}
\]
%
Similarly as in the case of the notation of product and sum of functions above, 
the superscripts in $\bigoplus^1_\Sigma$ and $\bigoplus^2_\Sigma$
shall be reserved to the ambiguous case of $\Phi_{\Sigma,\Sigma}$,
in order to to distinguish between the first and the second argument.
%
\begin{definition}\label{def:label-th-complete}
A label theory~$\bar\Phi$ is \emph{complete} when 
its underlying semiring~$\Semiring$ is complete, and
for all $\Phi_{\Sigma, \Delta} \in \bar\Phi$ 
and all $\eta \in \Phi_{\Sigma, \Delta}$,
$\bigoplus_\Sigma \eta \in \Phi_{\Delta}$ and 
$\bigoplus_\Delta \eta \in \Phi_{\Sigma}$.
\end{definition}

\noindent 
The following facts are immediate.
%
\begin{lemma}
For $\bar\Phi$ complete %label theory for all 
$\alpha \in \Semiring$,
$\phi, \phi' \in \Phi_{\Sigma}$,
$\psi \in \Phi_{\Delta}$, and
$\eta \in \Phi_{\Sigma, \Delta}$:
%
\begin{enumerate}
\item[$i.$]
\( \bigoplus_{\Sigma}\bigoplus_{\Delta} \eta = \bigoplus_{\Delta}\bigoplus_{\Sigma} \eta \)
\item[$ii.$] 
\( \alpha \otimes \bigoplus_{\Sigma} \phi = \bigoplus_{\Sigma} (\alpha \otimes \phi) \) and
\( \bigl( \bigoplus_{\Sigma} \phi \bigr) \otimes\alpha = \bigoplus_{\Sigma} (\phi \otimes \alpha) \),
and similarly for~$\oplus$
\item[$iii.$]
\( \bigl(\bigoplus_{\Sigma} \phi\bigr) \oplus \bigl(\bigoplus_{\Sigma} \phi'\bigr) 
   = \bigoplus_{\Sigma} (\phi \oplus \phi') \) and
\( \bigl(\bigoplus_{\Sigma} \phi\bigr) \otimes \bigl(\bigoplus_{\Sigma} \phi'\bigr) 
   = \bigoplus_{\Sigma} (\phi \otimes \phi') \)
\item[$iv.$] 
\( \bigl(\bigoplus_{\Delta} \eta\bigr) \oplus \bigl(\bigoplus_{\Delta} \eta' \bigr) = 
 \bigoplus_{\Delta} (\eta \oplus \eta') \), and
\( \bigl(\bigoplus_{\Delta} \eta\bigr) \otimes \bigl(\bigoplus_{\Delta} \eta' \bigr) = 
 \bigoplus_{\Delta} (\eta \otimes \eta') \)
\item[$v.$] 
%\( \phi \oplus \bigl(\bigoplus_{\Delta} \eta\bigr) = \bigoplus_{\Delta} (\phi \oplus \eta) \),
\( \phi \otimes \bigl(\bigoplus_{\Delta} \eta\bigr) = \bigoplus_{\Delta} (\phi \otimes \eta) \), and
\( \bigl(\bigoplus_{\Delta} \eta\bigr) \otimes \phi = \bigoplus_{\Delta} (\eta \otimes \phi) \),
and similarly for~$\oplus$
\item[$vi.$] 
%\( \psi \oplus \bigl(\bigoplus_{\Sigma} \eta\bigr) = \bigoplus_{\Sigma} (\psi \oplus \eta) \),
\( \psi \otimes \bigl(\bigoplus_{\Sigma} \eta\bigr) = \bigoplus_{\Sigma} (\psi \otimes \eta) \), and
\( \bigl(\bigoplus_{\Sigma} \eta\bigr) \otimes \psi = \bigoplus_{\Sigma} (\eta \otimes \psi) \),
and similarly for~$\oplus$
\end{enumerate}
\end{lemma}

%we call \emph{summary} of a function
%$\phi \in \Phi_\Sigma$,
%resp. $\eta \in \Phi_{\Sigma, \Delta}$,
%the value $\bigoplus_{a \in \Sigma} \phi(a)$, 
%resp. $\bigoplus_{a \in \Sigma} \bigoplus_{b \in \Delta} \eta(a, b)$.
%By definition of infinite sums in complete semirings, 
%a summary of $\phi \oplus \phi'$, $\alpha \otimes \phi$ and $\phi \otimes \alpha$
%can be computed from $\alpha \in \Semiring$ and summaries of $\phi$ and $\phi'$ in $\Phi_{\Sigma}$, 
%using the operators of $\Semiring$, 
%and the same holds for $\Phi_{\Delta}$ and $\Phi_{\Sigma, \Delta}$. 

\noindent 
Intuitively, the operators $\bigoplus_\Sigma$ 
return global minimum, \wrt $\leq_\oplus$, of functions of~$\bar\Phi$. 
%
A label theory is called \emph{effective} when 
for all $\phi \in \Phi_\Sigma$ and $\eta \in \Phi_{\Sigma, \Delta}$, 
$\bigoplus_{\Sigma} \phi$, 
$\bigoplus_{\Sigma} \eta$, and
$\bigoplus_{\Delta} \eta$
can be effectively computed from $\phi$ and $\eta$.

%there is an oracle returning, in constant time,
%$\bigoplus_{\Sigma} \phi$, 
%$\bigoplus_{\Sigma} \eta$, and
%$\bigoplus_{\Delta} \eta$
%and one symbol where the function reaches this minimum,
%denoted $\bigominus_\Sigma \phi$

\begin{definition}\label{def:label-th-convex}
Let $\Omega$ be an alphabet 
A function $\phi \in \Phi_\Sigma$ in a label theory over a complete semiring $\Semiring$
is called $k$-\emph{convex}, for a natural number $k$, iff 
$\mathrm{card}\{ a \in \Sigma \mid \phi(a) = \bigoplus_{\Sigma} \phi \} \leq k$.
\end{definition}
A label theory is $k$-convex if all its functions are $k$-convex.




