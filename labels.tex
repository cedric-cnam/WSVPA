% !TEX root = main.tex
%
% Label Theory
%
We shall now define the functions labeling the transitions of SW automata and transducers,
generalizing the Boolean algebras of~\cite{dAntoniVeanes17CAV} 
from Boolean to other semiring domains.
%
We consider \emph{alphabets}, which are countable sets of symbols 
denoted $\Sigma$, $\Delta$,...
%Let $\< \Semiring, \oplus, \zero, \otimes, \one>$ be a commutative, complete semiring.
%
\noindent 
Given a semiring $\< \Semiring, \oplus, \zero, \otimes, \one>$, 
a \emph{label theory} over $\Semiring$
is a tuplet $\bar\Phi$ of recursively enumerable sets denoted
%$\Phi_\epsilon \subseteq \Semiring$, % containing constant functions valued in $\Semiring$, 
$\Phi_\Sigma$, %and $\Phi_\Delta$, 
containing unary functions of type $\Sigma \to \Semiring$, %resp. $\Delta \to \Semiring$, 
and $\Phi_{\Sigma, \Delta}$, containing binary functions $\Sigma \times \Delta \to \Semiring$, 
and such that:

\noindent -- 
for all $\Phi_{\Sigma, \Delta} \in \bar\Phi$, we have
$\Phi_{\Sigma} \in \bar\Phi$ and $\Phi_{\Delta} \in \bar\Phi$

\noindent -- 
every $\Phi_{\Sigma}$ contains all the constant functions from $\Sigma$ into $\Semiring$, 
 
\noindent -- 
for all $\alpha \in \Semiring$ and $\phi \in \Phi_\Sigma$,
      $\alpha \otimes \phi : x \mapsto \alpha \otimes \phi(x)$, 
      and $\phi \otimes \alpha : x \mapsto \phi(x) \otimes \alpha$\\
\phantom{--} belong to $\Phi_\Sigma$, and similarly for $\oplus$ 
      and for $\Phi_{\Sigma, \Delta}$

\noindent -- 
for all $\phi, \phi' \in \Phi_\Sigma$,
$\phi \otimes \phi': x \mapsto \phi(x) \otimes \phi'(x)$ belongs to $\Phi_\Sigma$

\noindent -- 
for all $\eta, \eta' \in \Phi_{\Sigma, \Delta}$
$\eta \otimes \eta': x, y \mapsto \eta(x, y) \otimes \eta'(x, y)$ belongs to $\Phi_{\Sigma, \Delta}$

\noindent -- 
for all $\phi \in \Phi_\Sigma$ and $\eta \in \Phi_{\Sigma, \Delta}$,
$\phi \otimes \eta: x, y \mapsto \phi(x) \otimes \eta(x, y)$ belongs to $\Phi_{\Sigma, \Delta}$

\noindent -- 
for all $\psi \in \Phi_\Delta$ and $\eta \in \Phi_{\Sigma, \Delta}$,
$\eta \otimes \psi: x, y \mapsto \eta(x, y) \otimes \psi(y)$ belongs to $\Phi_{\Sigma, \Delta}$

\noindent -- 
similar closures hold for $\oplus$

\noindent -- 
the partial applications $\eta \in \Phi_{\Sigma, \Delta}$
and $\eta_a: y \mapsto \eta(a, y)$ for $a \in \Sigma$ %and $y \in \Delta$
and\\ 
\phantom{--} $\eta_b: x \mapsto \eta(x, b)$ for $b \in \Delta$ %and $x \in \Sigma$, 
belong respectively to~$\Phi_\Delta$ and~$\Phi_\Sigma$.


%Moreover, these sets are required to be closed under the 
%operators~$\oplus$ and~$\otimes$ of~$\Semiring$:
%for all $\phi, \phi' \in \Phi_\Sigma$,
%$\psi, \psi' \in \Phi_\Delta$, 
%and $\eta, \eta' \in \Phi_{\Sigma, \Delta}$, %the function
%%
%\begin{center}
%\begin{tabular}{cclll}
%$\phi \otimes \phi'$ & : & $x \mapsto \phi(x) \otimes \phi'(x)$ & belongs to $\Phi_\Sigma$,\\
%$\psi \otimes \psi'$ & : & $y \mapsto \psi(y) \otimes \psi'(y)$ & belongs to $\Phi_\Delta$,\\
%$\phi \otimes \eta$\;  & : & $x, y \mapsto \phi(x) \otimes \eta(x, y)$ & belongs to $\Phi_{\Sigma, \Delta}$,\\
%$\eta \otimes \psi$  & : & $x, y \mapsto \eta(x, y) \otimes \psi(y)$ & belongs to $\Phi_{\Sigma, \Delta}$,\\
%$\eta \otimes \eta'$ & : & $x, y \mapsto \eta(x, y) \otimes \eta'(x, y)$ & belongs to $\Phi_{\Sigma, \Delta}$, &
%\multicolumn{1}{r}{and similarly for $\oplus$.}\\ %the same also holds for the binary sum operator $\oplus$.
%\end{tabular}
%\end{center}
%
%Finally, it is also required 
%% that the codomain of every function of $\Phi_\Sigma$ and $\Phi_\Delta$ 
%% is a subset of $\Phi_\epsilon$, and
%that the partial applications of a function $\eta \in \Phi_{\Sigma, \Delta}$, 
%resp.  $\eta_a: y \mapsto f(a, y)$ for $a \in \Sigma$ and $y \in \Delta$
%and  $\eta_b: x \mapsto f(x, b)$ for $b \in \Delta$ and $x \in \Sigma$, 
%belong resp. to~$\Phi_\Sigma$ and~$\Phi_\Delta$.

\medskip\noindent
When~$\Semiring$ is complete, 
we call \emph{summary} of a function
$\phi \in \Phi_\Sigma$,
resp. $\eta \in \Phi_{\Sigma, \Delta}$,
the value $\bigoplus_{a \in \Sigma} \phi(a)$, 
resp. $\bigoplus_{a \in \Sigma} \bigoplus_{b \in \Delta} \eta(a, b)$.
By definition of infinite sums in complete semirings, 
a summary of $\phi \oplus \phi'$, $\alpha \otimes \phi$ and $\phi \otimes \alpha$
can be computed from $\alpha \in \Semiring$ and summaries of $\phi$ and $\phi'$ in $\Phi_{\Sigma}$, using the operators of $\Semiring$, 
and the same holds for $\Phi_{\Delta}$ and $\Phi_{\Sigma, \Delta}$. 
%
A label theory is called \emph{effective} when 
summaries of $\phi \otimes \phi'$, 
$\phi \oplus \eta$, $\phi \otimes \eta$, $\eta \oplus \psi$, $\eta \otimes \psi$, 
and of the partial applications $\eta_a$ and $\eta_b$, 
can be effectively computed, using the operators of $\Semiring$, 
from summaries of 
$\phi, \phi' \in \Phi_{\Sigma}$, $\psi \in \Phi_{\Delta}$, 
and $\eta \in \Phi_{\Sigma, \Delta}$. 

