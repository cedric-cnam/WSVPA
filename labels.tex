% !TEX root = main.tex
%
% Label Theory
%
We shall now define the functions labeling the transitions of SW automata and transducers,
generalizing the Boolean algebras of~\cite{dAntoniVeanes17CAV} 
from Boolean to other semiring domains.
%
We consider \emph{alphabets}, which are countable sets of symbols 
denoted $\Sigma$, $\Delta$,...
%Let $\< \Semiring, \oplus, \zero, \otimes, \one>$ be a commutative, complete semiring.
%
\noindent 
\philippe{OK, donc c'est là que les fonctions d'étiquettes prennent en argument l'input de la règle. Je ne sais
pas dans quelle mesure il faut donner un peu d'explications pour faciliter la compréhension du formalisme.}
Given a semiring $\< \Semiring, \oplus, \zero, \otimes, \one>$, 
a \emph{label theory} over $\Semiring$
is a set $\bar\Phi$ of recursively enumerable sets denoted
%$\Phi_\epsilon \subseteq \Semiring$, % containing constant functions valued in $\Semiring$, 
$\Phi_\Sigma$, %and $\Phi_\Delta$, 
containing unary functions of type $\Sigma \to \Semiring$, %resp. $\Delta \to \Semiring$, 
or $\Phi_{\Sigma, \Delta}$, containing binary functions $\Sigma \times \Delta \to \Semiring$, 
and such that:

\noindent -- 
for all $\Phi_{\Sigma, \Delta} \in \bar\Phi$, we have
$\Phi_{\Sigma} \in \bar\Phi$ and $\Phi_{\Delta} \in \bar\Phi$

\noindent -- 
every $\Phi_{\Sigma}\in \bar\Phi$ contains all the constant functions from $\Sigma$ into $\Semiring$, 
 
\noindent -- 
for all $\alpha \in \Semiring$ and $\phi \in \Phi_\Sigma$,
      $\alpha \otimes \phi : x \mapsto \alpha \otimes \phi(x)$, 
      and $\phi \otimes \alpha : x \mapsto \phi(x) \otimes \alpha$\\
\phantom{--} belong to $\Phi_\Sigma$, and similarly for $\oplus$ 
      and for $\Phi_{\Sigma, \Delta}$

\noindent -- 
for all $\phi, \phi' \in \Phi_\Sigma$,
$\phi \otimes \phi': x \mapsto \phi(x) \otimes \phi'(x)$ belongs to $\Phi_\Sigma$

\noindent -- 
for all $\eta, \eta' \in \Phi_{\Sigma, \Delta}$
$\eta \otimes \eta': x, y \mapsto \eta(x, y) \otimes \eta'(x, y)$ belongs to $\Phi_{\Sigma, \Delta}$

\noindent -- 
for all $\phi \in \Phi_\Sigma$ and $\eta \in \Phi_{\Sigma, \Delta}$,
$\phi \otimes_1 \eta: x, y \mapsto \phi(x) \otimes \eta(x, y)$ and\\
\phantom{--} $\eta \otimes_1 \phi: x, y \mapsto \eta(x, y) \otimes \phi(x)$
belong to $\Phi_{\Sigma, \Delta}$

\noindent -- 
for all $\psi \in \Phi_\Delta$ and $\eta \in \Phi_{\Sigma, \Delta}$,
$\psi \otimes_2 \eta: x, y \mapsto \psi(y) \otimes \eta(x, y)$ and\\
\phantom{--} $\eta \otimes_2 \psi: x, y \mapsto \eta(x, y) \otimes \psi(y)$ 
belong to $\Phi_{\Sigma, \Delta}$

\noindent -- 
similar closures hold for $\oplus$.

%\noindent -- 
%the partial applications $\eta \in \Phi_{\Sigma, \Delta}$
%and $\eta_a: y \mapsto \eta(a, y)$ for $a \in \Sigma$ %and $y \in \Delta$
%and\\ 
%\phantom{--} $\eta_b: x \mapsto \eta(x, b)$ for $b \in \Delta$ %and $x \in \Sigma$, 
%belong respectively to~$\Phi_\Delta$ and~$\Phi_\Sigma$.
\florent{partial application is needed?}

%Moreover, these sets are required to be closed under the 
%operators~$\oplus$ and~$\otimes$ of~$\Semiring$:
%for all $\phi, \phi' \in \Phi_\Sigma$,
%$\psi, \psi' \in \Phi_\Delta$, 
%and $\eta, \eta' \in \Phi_{\Sigma, \Delta}$, %the function
%%
%\begin{center}
%\begin{tabular}{cclll}
%$\phi \otimes \phi'$ & : & $x \mapsto \phi(x) \otimes \phi'(x)$ & belongs to $\Phi_\Sigma$,\\
%$\psi \otimes \psi'$ & : & $y \mapsto \psi(y) \otimes \psi'(y)$ & belongs to $\Phi_\Delta$,\\
%$\phi \otimes \eta$\;  & : & $x, y \mapsto \phi(x) \otimes \eta(x, y)$ & belongs to $\Phi_{\Sigma, \Delta}$,\\
%$\eta \otimes \psi$  & : & $x, y \mapsto \eta(x, y) \otimes \psi(y)$ & belongs to $\Phi_{\Sigma, \Delta}$,\\
%$\eta \otimes \eta'$ & : & $x, y \mapsto \eta(x, y) \otimes \eta'(x, y)$ & belongs to $\Phi_{\Sigma, \Delta}$, &
%\multicolumn{1}{r}{and similarly for $\oplus$.}\\ %the same also holds for the binary sum operator $\oplus$.
%\end{tabular}
%\end{center}
%
%Finally, it is also required 
%% that the codomain of every function of $\Phi_\Sigma$ and $\Phi_\Delta$ 
%% is a subset of $\Phi_\epsilon$, and
%that the partial applications of a function $\eta \in \Phi_{\Sigma, \Delta}$, 
%resp.  $\eta_a: y \mapsto f(a, y)$ for $a \in \Sigma$ and $y \in \Delta$
%and  $\eta_b: x \mapsto f(x, b)$ for $b \in \Delta$ and $x \in \Sigma$, 
%belong resp. to~$\Phi_\Sigma$ and~$\Phi_\Delta$.
%
\noindent 
Intuitively, the operators $\bigoplus_\Sigma$ 
return global minimum, \wrt $\leq_\oplus$, of functions of~$\Phi_\Sigma$. 
%
When the semiring $\Semiring$ is complete, we 
consider the following operators on the functions of~$\bar\Phi$. % a label theory.
%(we use overloading to simplify notations):
\[
\begin{array}{ll}
\bigoplus_\Sigma : \Phi_\Sigma \to \Semiring,\ 
  \phi \mapsto \displaystyle\bigoplus_{a \in \Sigma} \phi(a)\\
\bigoplus^1_\Sigma : 
  \Phi_{\Sigma,\Delta} \to \Phi_\Delta,\ 
  \eta \mapsto \bigl( y \mapsto \displaystyle\bigoplus_{a \in \Sigma} \eta(a, y) \bigr) &
\bigoplus^2_\Delta : 
  \Phi_{\Sigma,\Delta} \to \Phi_\Sigma,\ 
  \eta \mapsto \bigl( x \mapsto \displaystyle\bigoplus_{b \in \Delta} \eta(x, b) \bigr)\\
\end{array}
\]
%
\medskip\noindent
In what follows, we might omit the sub- and superscripts in 
$\otimes_1$, $\bigoplus^1_\Sigma$..., 
%$\otimes_2$, $\oplus_1$, $\oplus_2$
when there is no ambiguity.
We shall keep them only for the special case $\Sigma = \Delta$,
\ie $\eta \in \Phi_{\Sigma, \Sigma}$, % for~$\otimes_1$ above,
%and $\eta \in \Phi_{\Delta, \Delta}$ for~$\otimes_2$. 
%Similarly as for the above product and sum of functions, 
%the superscripts in $\bigoplus^1_\Sigma$ and $\bigoplus^2_\Sigma$
%shall be reserved to the ambiguous case of $\Phi_{\Sigma,\Sigma}$,
in order to be able to distinguish between the first and the second argument.
%
\begin{definition}\label{def:label-th-complete}
A label theory~$\bar\Phi$ is \emph{complete} when 
the underlying semiring~$\Semiring$ is complete, and
for all $\Phi_{\Sigma, \Delta} \in \bar\Phi$ 
and all $\eta \in \Phi_{\Sigma, \Delta}$,
$\bigoplus^1_\Sigma \eta \in \Phi_{\Delta}$ and 
$\bigoplus^2_\Delta \eta \in \Phi_{\Sigma}$.
\end{definition}
%
\florent{notion of diagram of functions akin BDD for transitions in practice}

\noindent 
The following facts are immediate.
\florent{mv appendix?}
%
\begin{lemma}
For $\bar\Phi$ complete %label theory for all 
$\alpha \in \Semiring$,
$\phi, \phi' \in \Phi_{\Sigma}$,
$\psi \in \Phi_{\Delta}$, and
$\eta \in \Phi_{\Sigma, \Delta}$:
%
\begin{enumerate}
\item[$i.$]
\( \bigoplus_{\Sigma}\bigoplus^2_{\Delta} \eta = \bigoplus_{\Delta}\bigoplus^1_{\Sigma} \eta \)
\item[$ii.$] 
\( \alpha \otimes \bigoplus_{\Sigma} \phi = \bigoplus_{\Sigma} (\alpha \otimes \phi) \) and
\( \bigl( \bigoplus_{\Sigma} \phi \bigr) \otimes\alpha = \bigoplus_{\Sigma} (\phi \otimes \alpha) \),
and similarly for~$\oplus$
\item[$iii.$]
\( \bigl(\bigoplus_{\Sigma} \phi\bigr) \oplus \bigl(\bigoplus_{\Sigma} \phi'\bigr) 
   = \bigoplus_{\Sigma} (\phi \oplus \phi') \) and
\( \bigl(\bigoplus_{\Sigma} \phi\bigr) \otimes \bigl(\bigoplus_{\Sigma} \phi'\bigr) 
   = \bigoplus_{\Sigma} (\phi \otimes \phi') \)
\item[$iv.$] 
\( \bigl(\bigoplus^2_{\Delta} \eta\bigr) \oplus \bigl(\bigoplus^2_{\Delta} \eta' \bigr) = 
 \bigoplus^2_{\Delta} (\eta \oplus \eta') \), and
\( \bigl(\bigoplus^2_{\Delta} \eta\bigr) \otimes \bigl(\bigoplus^2_{\Delta} \eta' \bigr) = 
 \bigoplus^2_{\Delta} (\eta \otimes \eta') \)
\item[$v.$] 
%\( \phi \oplus \bigl(\bigoplus_{\Delta} \eta\bigr) = \bigoplus_{\Delta} (\phi \oplus \eta) \),
\( \phi \otimes \bigl(\bigoplus^2_{\Delta} \eta\bigr) = \bigoplus_{\Delta} (\phi \otimes_1 \eta) \), and
\( \bigl(\bigoplus^2_{\Delta} \eta\bigr) \otimes \phi = \bigoplus_{\Delta} (\eta \otimes_1 \phi) \),
and similarly for~$\oplus$
\item[$vi.$] 
%\( \psi \oplus \bigl(\bigoplus_{\Sigma} \eta\bigr) = \bigoplus_{\Sigma} (\psi \oplus \eta) \),
\( \psi \otimes \bigl(\bigoplus^1_{\Sigma} \eta\bigr) = \bigoplus_{\Sigma} (\psi \otimes_2 \eta) \), and
\( \bigl(\bigoplus^1_{\Sigma} \eta\bigr) \otimes \psi = \bigoplus_{\Sigma} (\eta \otimes_2 \psi) \),
and similarly for~$\oplus$
\end{enumerate}
\end{lemma}

%we call \emph{summary} of a function
%$\phi \in \Phi_\Sigma$,
%resp. $\eta \in \Phi_{\Sigma, \Delta}$,
%the value $\bigoplus_{a \in \Sigma} \phi(a)$, 
%resp. $\bigoplus_{a \in \Sigma} \bigoplus_{b \in \Delta} \eta(a, b)$.
%By definition of infinite sums in complete semirings, 
%a summary of $\phi \oplus \phi'$, $\alpha \otimes \phi$ and $\phi \otimes \alpha$
%can be computed from $\alpha \in \Semiring$ and summaries of $\phi$ and $\phi'$ in $\Phi_{\Sigma}$, 
%using the operators of $\Semiring$, 
%and the same holds for $\Phi_{\Delta}$ and $\Phi_{\Sigma, \Delta}$. 


\philippe{Je trouve qu'il y a beaucoup de notions à retenir (complete, effective) et ça devient
difficile pour un lecteur non spécialiste. Est-ce que tout est nécessaire (je ne sais plus qui m'avait dit: 
un concept en plus, un point en moins.}

\noindent
A label theory is called \emph{effective} when 
for all $\phi \in \Phi_\Sigma$ and $\eta \in \Phi_{\Sigma, \Delta}$, 
$\bigoplus_{\Sigma} \phi$, 
$\bigoplus_{\Delta}\bigoplus_{\Sigma} \eta$, and
$\bigoplus_{\Sigma}\bigoplus_{\Delta} \eta$
can be effectively computed from $\phi$ and $\eta$.
\florent{$\exists$ oracle returning ...  in worst time complexity $T$.}

%there is an oracle returning, in constant time,
%$\bigoplus_{\Sigma} \phi$, 
%$\bigoplus_{\Sigma} \eta$, and
%$\bigoplus_{\Delta} \eta$
%and one symbol where the function reaches this minimum,
%denoted $\bigominus_\Sigma \phi$

%\begin{definition}\label{def:label-th-convex}
%Let $\Omega$ be an alphabet 
%A function $\phi \in \Phi_\Sigma$ in a label theory over a complete semiring $\Semiring$
%is called $k$-\emph{convex}, for a natural number $k$, iff 
%$\mathrm{card}\{ a \in \Sigma \mid \phi(a) = \bigoplus_{\Sigma} \phi \} \leq k$.
%\end{definition}
%A label theory is $k$-convex if all its functions are $k$-convex.


Concretely, in one of the language models defined below, 
we consider a finite number of base functions $\phi, \eta$ of the underlying label theory,
labelling transitions, and combine them with the above operators for construction of 
other models.
The combinations might be represented by dags (diagrams) whose leaves are labeled
by base functions and inner nodes by operators.
% we can compute the value of a diagram in time...



