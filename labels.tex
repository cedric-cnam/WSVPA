% !TEX root = main.tex
%
% Label Theory
%
\subsection{Label Theories} 
%\noindent
We define the functions labeling the transitions of SW automata and transducers,
generalizing the Boolean algebras of~\cite{dAntoniVeanes17CAV}.
%from Boolean to other semiring domains.
%
We consider \emph{alphabets}, which are non-empty countable 
%\reviews{1) non-empty}
sets of symbols
denoted by $\Sigma$, $\Delta$,...
Moreover, $\Sigma^*$ is the set of finite sequences (\emph{words}) over
$\Sigma$, $\varepsilon$ the empty word, $\Sigma^+ = \Sigma^* \setminus \{ \varepsilon \}$,
and $u v$ denotes the concatenation of $u, v \in \Sigma^*$.
%where $au$, and $bv$, denote the concatenation
%of the symbol $a \in \Sigma$ (resp. $b \in \Delta$)
%with a word $u \in \Sigma^*$ (resp. $v \in \Delta^*$).
%\lydia{added $u$ and $v$ def}

Given a semiring $\< \Semiring, \oplus, \zero, \otimes, \one>$,
a \emph{label theory} $\bar\Phi$ over~$\Semiring$
is an indexed family of %recursively enumerable %$\Phi_\epsilon \subseteq \Semiring$, % containing constant functions valued in $\Semiring$,
sets denoted by
%\reviews{1) $\bar\Phi$ is a $(\Sigma \cup \Sigma \times \Delta)$-indexed family...}
$\Phi_\Sigma$, %and $\Phi_\Delta$,
containing unary functions of type $\Sigma \to \Semiring$, %resp. $\Delta \to \Semiring$,
or $\Phi_{\Sigma, \Delta}$, containing binary functions $\Sigma \times \Delta \to \Semiring$,
and such that:
%\florent{unary for \SWA (weight depends on input symbol) and binary for transducers and VPA (weight depends on input symbol AND output or stack symbol)}
\begin{itemize}
\item %\noindent --
     for all $\Phi_{\Sigma, \Delta} \in \bar\Phi$, we have
     $\Phi_{\Sigma} \in \bar\Phi$ and ${\Phi_{\Delta} \in \bar\Phi}$.
%
\item %\noindent --
    all $\Phi_{\Sigma}, \Phi_{\Sigma, \Delta}\in \bar\Phi$ contain all the constant functions of 
    $\Sigma \to \Semiring$, resp. ${\Sigma \times \Delta  \to \Semiring}$.% into $\Semiring$,
%\reviews{3) contradicts $\Sigma$ countable}
%\reviews{1) $\forall \Sigma$...}
%
\item %\noindent --
      for all $\Phi_{\Sigma} \in \bar\Phi$, 
      for all $\phi \in \Phi_\Sigma$, and $\alpha \in \Semiring$,
      $\alpha \otimes \phi : x \mapsto \alpha \otimes \phi(x)$,
      and $\phi \otimes \alpha : x \mapsto \phi(x) \otimes \alpha$,  %\phantom{--} 
      belong to $\Phi_\Sigma$, and similarly for $\oplus$
      and for $\Phi_{\Sigma, \Delta}$.
%
\item %\noindent --
      for all $\Phi_{\Sigma} \in \bar\Phi$, 
      for all $\phi, \phi' \in \Phi_\Sigma$,
      $\phi \otimes \phi': x \mapsto \phi(x) \otimes \phi'(x)$ belongs to $\Phi_\Sigma$.
%
\item %\noindent --
	  for all $\Phi_{\Sigma, \Delta} \in \bar\Phi$,
      for all $\eta, \eta' \in \Phi_{\Sigma, \Delta}$,
      $\eta \otimes \eta': x, y \mapsto \eta(x, y) \otimes \eta'(x, y)$ belongs to $\Phi_{\Sigma, \Delta}$.
%
\item %\noindent --
      for all $\Phi_{\Sigma}, \Phi_{\Sigma, \Delta} \in \bar\Phi$,
      for all $\phi \in \Phi_\Sigma$ and $\eta \in \Phi_{\Sigma, \Delta}$,
      $\phi \otimes_1 \eta: x, y \mapsto \phi(x) \otimes \eta(x, y)$ and
      $\eta \otimes_1 \phi: x, y \mapsto \eta(x, y) \otimes \phi(x)$
      belong to $\Phi_{\Sigma, \Delta}$.
%
\item %\noindent --
      for all $\Phi_{\Delta}, \Phi_{\Sigma, \Delta} \in \bar\Phi$,
      for all $\psi \in \Phi_\Delta$ and $\eta \in \Phi_{\Sigma, \Delta}$,
      $\psi \otimes_2 \eta: x, y \mapsto \psi(y) \otimes \eta(x, y)$ and
      $\eta \otimes_2 \psi: x, y \mapsto \eta(x, y) \otimes \psi(y)$
      belong to $\Phi_{\Sigma, \Delta}$.
%
\item %\noindent --
      similar closures hold for $\oplus$.
\end{itemize}

%\noindent --
%the partial applications $\eta \in \Phi_{\Sigma, \Delta}$
%and $\eta_a: y \mapsto \eta(a, y)$ for $a \in \Sigma$ %and $y \in \Delta$
%and\\
%\phantom{--} $\eta_b: x \mapsto \eta(x, b)$ for $b \in \Delta$ %and $x \in \Sigma$,
%belong respectively to~$\Phi_\Delta$ and~$\Phi_\Sigma$.
\florent{partial application is needed?}

%Moreover, these sets are required to be closed under the
%operators~$\oplus$ and~$\otimes$ of~$\Semiring$:
%for all $\phi, \phi' \in \Phi_\Sigma$,
%$\psi, \psi' \in \Phi_\Delta$,
%and $\eta, \eta' \in \Phi_{\Sigma, \Delta}$, %the function
%%
%\begin{center}
%\begin{tabular}{cclll}
%$\phi \otimes \phi'$ & : & $x \mapsto \phi(x) \otimes \phi'(x)$ & belongs to $\Phi_\Sigma$,\\
%$\psi \otimes \psi'$ & : & $y \mapsto \psi(y) \otimes \psi'(y)$ & belongs to $\Phi_\Delta$,\\
%$\phi \otimes \eta$\;  & : & $x, y \mapsto \phi(x) \otimes \eta(x, y)$ & belongs to $\Phi_{\Sigma, \Delta}$,\\
%$\eta \otimes \psi$  & : & $x, y \mapsto \eta(x, y) \otimes \psi(y)$ & belongs to $\Phi_{\Sigma, \Delta}$,\\
%$\eta \otimes \eta'$ & : & $x, y \mapsto \eta(x, y) \otimes \eta'(x, y)$ & belongs to $\Phi_{\Sigma, \Delta}$, &
%\multicolumn{1}{r}{and similarly for $\oplus$.}\\ %the same also holds for the binary sum operator $\oplus$.
%\end{tabular}
%\end{center}
%
%Finally, it is also required
%% that the codomain of every function of $\Phi_\Sigma$ and $\Phi_\Delta$
%% is a subset of $\Phi_\epsilon$, and
%that the partial applications of a function $\eta \in \Phi_{\Sigma, \Delta}$,
%resp.  $\eta_a: y \mapsto f(a, y)$ for $a \in \Sigma$ and $y \in \Delta$
%and  $\eta_b: x \mapsto f(x, b)$ for $b \in \Delta$ and $x \in \Sigma$,
%belong resp. to~$\Phi_\Sigma$ and~$\Phi_\Delta$.
%
\noindent
When the semiring $\Semiring$ is complete, we consider moreover 
the following operators on the functions of~$\bar\Phi$. % a label theory.
%(we use overloading to simplify notations):
\[
\begin{array}{ll}
\bigoplus_\Sigma : \Phi_\Sigma \to \Semiring,\
  \phi \mapsto \displaystyle\bigoplus_{a \in \Sigma} \phi(a)\\
%  
\bigoplus^1_\Sigma :
  \Phi_{\Sigma,\Delta} \to \Phi_\Delta,\
  \eta \mapsto \bigl( y \mapsto \displaystyle\bigoplus_{a \in \Sigma} \eta(a, y) \bigr) &
%
\bigoplus^2_\Delta :
  \Phi_{\Sigma,\Delta} \to \Phi_\Sigma,\
  \eta \mapsto \bigl( x \mapsto \displaystyle\bigoplus_{b \in \Delta} \eta(x, b) \bigr)\\
\end{array}
\]
Intuitively, $\bigoplus_\Sigma$
returns the global minimum, \wrt $\leq_\oplus$, of a function $\phi$ of~$\Phi_\Sigma$,
and $\bigoplus^1_\Sigma$, $\bigoplus^2_\Delta$ return partial minimums of 
a function $\phi$ of~$\Phi_{\Sigma,\Delta}$.

%\medskip\noindent
%In what follows, we might omit the sub- and superscripts in
%\philippe{On peut simplifier la notation et supprimer cette discussion?}
%$\otimes_1$, $\bigoplus^1_\Sigma$...,
%%$\otimes_2$, $\oplus_1$, $\oplus_2$
%when there is no ambiguity.
%We shall keep them only for the special case $\Sigma = \Delta$,
%\ie $\eta \in \Phi_{\Sigma, \Sigma}$, % for~$\otimes_1$ above,
%%and $\eta \in \Phi_{\Delta, \Delta}$ for~$\otimes_2$.
%%Similarly as for the above product and sum of functions,
%%the superscripts in $\bigoplus^1_\Sigma$ and $\bigoplus^2_\Sigma$
%%shall be reserved to the ambiguous case of $\Phi_{\Sigma,\Sigma}$,
%in order to be able to distinguish between the first and the second argument.
%
%\begin{definition}\label{def:label-th-complete}
%A label theory~$\bar\Phi$ is \emph{complete} when

\noindent
We assume that when the underlying semiring~$\Semiring$ is complete:
\begin{itemize}
\item 
	  for all $\Phi_{\Sigma, \Delta} \in \bar\Phi$
	  and all $\eta \in \Phi_{\Sigma, \Delta}$,
	  $\bigoplus^1_\Sigma \eta \in \Phi_{\Delta}$ and
	  $\bigoplus^2_\Delta \eta \in \Phi_{\Sigma}$.
\end{itemize}


\begin{example}\label{distance-time}
We return to Example~\ref{ex:running}.
Let $\Deltai$ be the subset of $\Delta$ without markup symbols.
In order to align the input in $\Sigma^*$ %sequence
with a music score, % in $\Delta^*$, 
we must account for
the expressive timing of human performance that
results in small time shifts between an input event of~$\Sigma$ and the corresponding
notation event in~$\Deltai$.
These shifts can be weighted as the time distance between both,
computed in the tropical semiring by $\delta \in \Phi_{\Sigma, \Deltai}$,
defined as follows:
\[
%\mbox{for~all}\,
%\mu\ts{\tau} \in \Sigma,
%\nu\ts{\tau'} \in \Deltai,
\delta(\mu\ts{\tau}, \nu\ts{\tau'}) =
\left\{
\begin{array}{ll}
   | \tau' - \tau | & \mathrm{if}\  \nu \rm{\ corresponds\ to\ } \mu,\\
   \zero  & \mathrm{otherwise}
\end{array}
\right.
\]
%The performance is
%$I = [ \mu_1\ts{0.07}, \mu_2, 0.72>, \<\mu_3, 0.91>, \<\mu_4, 1.05>, \<\mu_5, 1.36>, \<\mu_6, 1.71>]$,
%and the (linearized) score is
%$[\<\nu_1,0>, \<\nu_2,\frac{3}{4}>, \<\nu_3,\frac{7}{8}>, \<\nu_4,1>, \<\nu_5,\frac{4}{3}>, \<\nu_6\frac{5}{3}>]$
%Assuming the pairwise correspondence of MIDI symbols
%$\mu_i$ and notation symbols $\nu_i$, for $i \in [1, 6]$,
\florent{$I$ et $O$ deja donnés dans ex.\ref{ex:running}}
The distance between $I$ and $O$ is the  aggregation with $\otimes$
of the pairwise differences between the
timestamps. In the tropical semiring, this yields
$|0.07 - 0| + |0.72 - \frac{3}{4}| + |0.91- \frac{7}{8} | +
|1.05-1| + |1.36-\frac{4}{3}| + |1.71-\frac{5}{3}|= 0.255$.
\endex
\end{example}


% !TEX root = main.tex
%
% Properties of Label Theory

The following facts are immediate consequences of the definitions of 
the operators on the functions of labels theories. % in Section~\ref{sec:prelim}.

\begin{lemma} \label{lem:label-th}
For a complete label theory $\bar\Phi$, 
for all $\Phi_{\Sigma}, \Phi_{\Delta}, \Phi_{\Sigma, \Delta} \in \bar\Phi$, 
$\alpha \in \Semiring$,
$\phi, \phi' \in \Phi_{\Sigma}$,
$\psi \in \Phi_{\Delta}$, and
$\eta \in \Phi_{\Sigma, \Delta}$,
it holds that:
%
\begin{enumerate}
\item[$i.$]
\( \bigoplus_{\Sigma}\bigoplus^2_{\Delta} \eta = \bigoplus_{\Delta}\bigoplus^1_{\Sigma} \eta \).
\item[$ii.$]
\( \alpha \otimes \bigoplus_{\Sigma} \phi = \bigoplus_{\Sigma} (\alpha \otimes \phi) \) and
\( \bigl( \bigoplus_{\Sigma} \phi \bigr) \otimes\alpha = \bigoplus_{\Sigma} (\phi \otimes \alpha) \),
and similarly for~$\oplus$.
\item[$iii.$]
\( \bigl(\bigoplus_{\Sigma} \phi\bigr) \oplus \bigl(\bigoplus_{\Sigma} \phi'\bigr)
   = \bigoplus_{\Sigma} (\phi \oplus \phi') \) and
\( \bigl(\bigoplus_{\Sigma} \phi\bigr) \otimes \bigl(\bigoplus_{\Sigma} \phi'\bigr)
   = \bigoplus_{\Sigma} (\phi \otimes \phi') \).
\item[$iv.$]
\( \bigl(\bigoplus^2_{\Delta} \eta\bigr) \oplus \bigl(\bigoplus^2_{\Delta} \eta' \bigr) =
 \bigoplus^2_{\Delta} (\eta \oplus \eta') \), and
\( \bigl(\bigoplus^2_{\Delta} \eta\bigr) \otimes \bigl(\bigoplus^2_{\Delta} \eta' \bigr) =
 \bigoplus^2_{\Delta} (\eta \otimes \eta') \).
\item[$v.$]
%\( \phi \oplus \bigl(\bigoplus_{\Delta} \eta\bigr) = \bigoplus_{\Delta} (\phi \oplus \eta) \),
\( \phi \otimes \bigl(\bigoplus^2_{\Delta} \eta\bigr) = \bigoplus_{\Delta} (\phi \otimes_1 \eta) \), and
\( \bigl(\bigoplus^2_{\Delta} \eta\bigr) \otimes \phi = \bigoplus_{\Delta} (\eta \otimes_1 \phi) \),\\
and similarly for~$\oplus$.
\item[$vi.$]
%\( \psi \oplus \bigl(\bigoplus_{\Sigma} \eta\bigr) = \bigoplus_{\Sigma} (\psi \oplus \eta) \),
\( \psi \otimes \bigl(\bigoplus^1_{\Sigma} \eta\bigr) = \bigoplus_{\Sigma} (\psi \otimes_2 \eta) \), and
\( \bigl(\bigoplus^1_{\Sigma} \eta\bigr) \otimes \psi = \bigoplus_{\Sigma} (\eta \otimes_2 \psi) \),\\
and similarly for~$\oplus$.
\end{enumerate}
\end{lemma}




\noindent 
The following property of label theories will be useful in 
the following results, in order to ensure 
the computability of the above infinite sum operators.
%
\begin{definition}\label{def:effective}
A label theory $\bar\Phi$ is called \emph{effective} when
for all $\Phi_{\Sigma}, \Phi_{\Sigma, \Delta} \in \bar\Phi$, 
$\phi \in \Phi_\Sigma$ and $\eta \in \Phi_{\Sigma, \Delta}$,
$\bigoplus_{\Sigma} \phi$,
$\bigoplus_{\Delta}\bigoplus^1_{\Sigma} \eta$, and
$\bigoplus_{\Sigma}\bigoplus^2_{\Delta} \eta$
can be effectively computed from $\phi$ and $\eta$,
as well as one symbol, at least, 
reaching each of these bounds.
%the number of symbols reaching these bounds is finite
\end{definition}

The effectiveness is a strong restriction on label theories.
It is however not unrealistic in the context of the problems for languages models considered below, 
namely the combination of automata, best-search and symbolic weighted parsing.
%
In fact, every instance of such problems comes with a finite number of automata, each one containing a finite number of functions in a label theory, in their transitions. 
We may assume that the global minimums $\bigoplus_{\Sigma} \phi$,
$\bigoplus_{\Delta}\bigoplus^1_{\Sigma} \eta$, and
$\bigoplus_{\Sigma}\bigoplus^2_{\Delta} \eta$
of all these functions are known.
%
Then, the other functions considered when solving the problems are obtained by combination with the above operators.  
These operators preserve effectiveness, as long as the semiring is monotonic. 
In practice, the combinations may be represented by structures like 
Algebraic Decision Diagrams~\cite{Bahar97ADD}.

%we call \emph{summary} of a function
%$\phi \in \Phi_\Sigma$,
%resp. $\eta \in \Phi_{\Sigma, \Delta}$,
%the value $\bigoplus_{a \in \Sigma} \phi(a)$,
%resp. $\bigoplus_{a \in \Sigma} \bigoplus_{b \in \Delta} \eta(a, b)$.
%By definition of infinite sums in complete semirings,
%a summary of $\phi \oplus \phi'$, $\alpha \otimes \phi$ and $\phi \otimes \alpha$
%can be computed from $\alpha \in \Semiring$ and summaries of $\phi$ and $\phi'$ in $\Phi_{\Sigma}$,
%using the operators of $\Semiring$,
%and the same holds for $\Phi_{\Delta}$ and $\Phi_{\Sigma, \Delta}$.


%there is an oracle returning, in constant time,
%$\bigoplus_{\Sigma} \phi$,
%$\bigoplus_{\Sigma} \eta$, and
%$\bigoplus_{\Delta} \eta$
%and one symbol where the function reaches this minimum,
%denoted $\bigominus_\Sigma \phi$

%\begin{definition}\label{def:label-th-convex}
%Let $\Omega$ be an alphabet
%A function $\phi \in \Phi_\Sigma$ in a label theory over a complete semiring $\Semiring$
%is called $k$-\emph{convex}, for a natural number $k$, iff
%$\mathrm{card}\{ a \in \Sigma \mid \phi(a) = \bigoplus_{\Sigma} \phi \} \leq k$.
%\end{definition}
%A label theory is $k$-convex if all its functions are $k$-convex.

%s.t. for all $\phi \in \Phii$,
%$\psi \in \Phir$,
%and $\eta \in \Phicr$,
%$\displaystyle\bigoplus_{a \in \Omegai} \phi(a)$
%$\displaystyle\bigoplus_{r \in \Omegar} \phi(r)$ and
%$\displaystyle\bigoplus_{{\call{c}} \in \Omegac}
%\displaystyle\bigoplus_{{\return{r}} \in \Omegar} \eta({\call{c}}, {\return{r}})$
%are computable...
%
% total ?
% monotonic and superior writ natural ordering
%Regarding the infinite sum operator, note that
%$\bigoplus_{x \in \Phi_\Omega} \phi(x)$,
%$\bigoplus_{y \in \Phi_\Delta} \psi(y)$, and
%... exist and in $\Semiring$.



%Concretely, in one of the language models defined below,
%we consider a finite number of base functions $\phi, \eta$ of the underlying label theory,
%labelling transitions, and combine them with the above operators for construction of
%other models.
%The combinations might be represented by dags (diagrams) whose leaves are labeled
%by base functions and inner nodes by operators.
% we can compute the value of a diagram in time...
